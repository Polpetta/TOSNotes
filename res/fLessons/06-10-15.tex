\section{Lezione 06-10-15}

\subsection{Introduzione al software libero}

Definizione: Il software libero\footnote{Libero \`e diverso da gratuito: è una questione di libertà, non di prezzo.} \`e software che garantisce le seguenti quattro libert\`a fondamentali:

\begin{enumerate}

\item (\textbf{Libertà 0}) Eseguire il programma per qualsiasi scopo: un programma libero non pu\`o mai imporre:

  \begin{itemize}

  \item Restrizioni in termini di tempo
  \item Restrizioni in termini di scopo
  \item Limitazioni di area geografica $\to$ questo pu\`o essere un caso speciale

  \end{itemize}

\item (\textbf{Libertà 1}) Studiare come funziona il programma e adattarlo alle proprie necessit\`a.
Nessuna restrizione sulla modifica e comprensione del tipo:

  \begin{itemize}

  \item Richiedere l'acquisto di licenze speciali
  \item Richiedere le firma di NDA
  \item Impedire l'accesso al codice sorgente

  \end{itemize}

\item (\textbf{Libertà 2}) Redistribuire copie in modo da aiutare il prossimo. Il software libero non proibisce di prestare la propria copia ad una persona o darne una copia, \underline{nemmeno dietro un pagamento di un compenso}

  \item (\textbf{Libertà 3}) Migliorare il programma e distribuire pubblicamente i miglioramenti, in modo tale che tutta la comunit\`a ne tragga beneficio. Infatti migliorare il programma e distribuire i miglioramenti permette a chi non ha il tempo o le capacita' per risolvere un problema di accedere indirettamente alla liber\`a di modifica $\to$ anche questo pu\`o essere dietro compenso.

\end{enumerate}

\subsubsection{Importanza del software libero}

\begin{itemize}

\item Riduzione dei costi

\item Trasparenze $\to$ soprattutto per gli enti statali

\item Nessun lock-in (si veda il caso di XFree86)

\item Sicurezza e affidabilit\`a\footnote{Nota: da un punto di vista pratico la qualit\`a del software libero tende a essere pari con il miglior software proprietario. Il problema \`e che \`e possibile che siano presenti bug di sicurezza non noti agli sviluppatori che possono essere sfruttati a scopo malevolo, senza che venga segnalato alla comunit\`a}

\end{itemize}

Il software libero, diversamente da quello proprietario, \`e un'infrastruttura e cambia il modo in cui si fa impresa (si veda Openerp, CUPS, Android). Quello che si sta capendo \`e che le molte menti creative senza lavorare per una certa azienda possono dare comunque una mano a quel determinato software, oltre che, dall'iterazione con gli utenti si hanno nuove idee, perch\`e viene gestito il progetto insieme.

\subsection{Albori del diritto d'autore}

\subsubsection{Venezia e i ``Privilegi''}

Nella Venezia del 1500 c.a. erano presenti i Privilegi. I privilegi si riferivano alla tecnologia utilizzata. I privilegi erano garantiti agli stampatori, ma non agli autori (rigurardo la pubblicazione e alla stampa di documenti-libri)

Nel 1517-1537 si va verso il privilegio d'autore. Si afferma infatti il ruolo dei privilegi sui libri comuni, con una limitazione della durata dei privilegi fino a 10 anni. Tutto ci\`o porta alla richieste di opere originali, con lo spostamento delle protezione delle opere: si cominciano a proteggere anche le modifiche\footnote{Inizialmente le opere modificate-rielaborate e ripubblicate erano considerate diverse dall'opera iniziale, e non una sua ``derivata''}.

Le cooperazioni veneziane attive dal tredicesimo secolo, adibite alle professioni artigiane proteggono scrupolosamente le conoscenze artigiane:
\begin{itemize}

\item Restrizioni sui movimenti degli artigiani
\item Conoscenze trasmesse esclusivamente per via orale
\item Meccanismi per limitare la crescita delle singole attivit\`a

\end{itemize}

Con tutti i controlli nasce l'idea del diritto immateriale. Tutto il riserbo sulle conoscenze gli fanno acquisire importanza. Nel 1474 si hanno i privilegi d'autore (di cui accennato prima) a Venezia, nel tentativo di portare ordine:

\begin{itemize}

\item Non creava percorsi obbligati, si limitava a codificare le politiche relative ai privilegi

\item Mirata principalmente agli inventori e non alle corporazioni

\end{itemize}

Con l'avvento dell'umanismo si ha un aumento di persone che tentano di ``scappare'' dalle corporazioni e ottenere dei provilegi: si ha una maggiore attenzione alla necessit\`a di proteggere il proprio lavoro. Molti dotti cominciano a preoccuparsi a problemi pratici, con un assottigliamento dei confini tra conoscienza e capacit\`a meccaniche. Da qui vengono scritti i primi libri in cui si cerca di avere una maggiore importanza alla protezione delle conoscienze, e comincia a cambiare il modo in cui esse vengono trasmesse.


\subsubsection{Inghilterra e Copyright}

Con la prima stamperia nel 1476 (in ritardo ripetto a Venezia) i problemi di copia non autorizzate hanno una ridotta importanza. Con l'introduzione della censura da parte della Corona, e nel 1557 viene dato il controllo esclusivo sulla stampa dei libri alla Stationers Company, posta sotto il controllo della Camera Stellata. Agendo in tale maniera gli autori vengono estromessi dai propri lavori.

Nel 1640 si ha l'abolizione della Camera Stellata e delle limitazioni sulla
stampa. Data l'esplosione dei giornali e delle libera stampa si ha che dal
1643-1694 viene instaurata di nuovo la censura da parte della Corona. Cos\`i
facendo, dal 1695-1704 si hanno 13 tentativi di restaurare dei controlli
censori. Nel 1710 con l'editto di Anna si ha lo scopo del Copyright visto come
stimolazione della cultura, proteggendo le creazioni intangibili, e non pi\`u le
copie come prima.

La licenza passa agli autori e non alle stamperie con una protezione di 14 anni e con la richiesta di estensione di 14 anni dopo la scadenza.

\subsubsection{Colonie Inglesi e USA}

In America nel 1638 reverendo Glover porta una macchina da stampa. Le autorit\`a del Massachusetts (dove in particolare si trovava la macchina) contribuiscono alla sua manutenzione. Nel 1672 viene dato il primo privilegio di stampa al signor Usher di stampare le leggi della colonia.


Il primo diritto d'autore viene concesso nel 1781: Andrew Law scrive una collezione di melodie, e per bloccare la competizione di altre collezioni simili chiede un monopolio.

Nel 1783 una petizione di John Ledyard cambia la situzione. Dopo aver scritto la
stesura del suo libro chiede il privilegio di stampa e dopo essergli stato concesso dalle
autorit\`a, viene creata una \textit{Connecticut Copyright Statute}, stabilendo
una legislazione generale (per quella regione). Sette anni dopo, nel 1790 si ha
il Copyright Act, che da le prime regolamentazioni sul Copyright a livello federale.


\subsubsection{Copyright moderno}
Con la diffusione del Copyright, si ha che le nazioni non riconoscono il Copyright di altri stati: nasce quindi l'esigenza di una legislazione uniforme. Nel 1883 si ha la Convenzione di Berna in cui le varie leggi vengono standardizzate. Inoltre viene resa automatica la tutela, senza nessuna registrazione. Il Copyright copriva solo gli oggetti tangibili e l'accordo includeva 165 paesi. Nel 1908 si deve affrontare White-Smith Music in una causa contro Apollo, primo caso legato a macchinari ``computazionali''. Si decide che la copia di musica su tubi per pianole non \`e una copia nel senso del Copyright Act. Si ha l'intervento del Congresso che afferma che i lavori derivati che possono essere percepiti solo attraverso l'uso i una macchina sono soggetti al Copyright. Si affronta il problema istituendo una legge ad hoc.

Nel 1950 con la nascita dei computer non si sente la necessit\`a del Copyright. Il software \`e piuttosto pensato principalmente come complemento alla macchina: poche esigenze di una protezione, e manca una legge sul Copyright software e i contratti che non si applicano a terzi\footnote{Data la mancanza di una legge sul copyright software se si volevano delle restrizioni nell'utilizzo del software da parte degli utenti si era soliti stipulare un contratto.}. Inoltre il Copyright Act del 1909 imponeva la registrazione del materiale protetto ma non era chiaro cosa implicasse per il software.
C'\`e il desiderio delle case software di proteggere il software sviluppato, e nel 1976 con il nuovo Copyright Act si ha che il software diviene proteggibile da Copyright. Una commissione viene incaricata di indagare sul grado di protezione dato al software. La commissione racomand\`o una definizione di software, e di permettere la copia o l'adattamento del software a patto che servisse a scopo di backup e fossero necessari per l'utilizzo del software. Viene deciso che le copie esatte potevano essere vendute a costo che venissero trasferite completamente al nuovo proprietario. Le copie modificate potevano inoltre essere usate solo dal legittimo proprietario.

Nel 1980 il congresso approva i suggerimenti della commisione. Ma si ha che il termine ``legittimo possessore di una copia'' viene sostituito da ``proprietario di una copia''. Ci\`o porter\`a al caso giudiziario MAI vs Plack.

Nel 1990 viene limitato l'uso di piena vendita.

\subsubsection{Diritto in Italia oggi}

Diritto esclusivo dell'autore su:
\begin{itemize}

\item Ridistribuzione
\item Modifica
\item Adattamento
\item Traduzione

\end{itemize}

Tale diritto\footnote{In Italia il diritto si suddivide in due tipi: Morale e Economico. Solamente il diritto Economico pu\`o essere trasferito/rinunciabile.} \`e:
\begin{itemize}

\item Rinunciabile
\item Trasferibile

\end{itemize}

Licenze software: ad esempio le licenze di Windows. Le licenze GPL non vogliono fungere da contratto. La licenza GNU vuole differire dalle solite licenze software.
