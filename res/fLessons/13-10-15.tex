\section{Lezione 13-10-15}

\subsection{GNU}

Nota: argomenti trattati nel libro \textit{Heroes of the computer: Cap. 1,2 ed Epilogo}

Gli hack\footnote{In informatica: hack \`e un esercizio di codice che dimostra l'inventiva dell'autore, creato anche per il piacere dstesso dello scriverlo} del mit sono scherzi anonimi che dimostrano la creativit\`a della persona che li eseguono. Son presenti fino dagli anni 50.

\subsubsection{Gli albori}

Nel 1950-1960 la culla hacker nasce ai laboratori del MIT tra i ragazzi del Tech Model Railboard Club. Il primo gruppo lo si ha dal sottogruppo Signal \& Power, che si occupava della gestione della circuiteria e segnali dei treni. Questa complessit\`a rasentava quella del software. Il Corso di intelligenza arificiale di McCarthy del 1959 facilit\`o la formazione di questo gruppo, permettendo di cominciare a programmare le prime macchine. I primi hack si ebbero sull'IBM 704, tramite la programmazione non interattiva\footnote{Queste macchine sono i primi esempi di macchine commerciali e si programmavano con le schede perforate e il loro accesso per l'utilizzo era molto ristretto.}. Con l'IBM 407\footnote{Macchinetta di servizio per la perforazione delle schede} e determinate modifiche era possibile usare il macchinario per poter programmare. La vera programmazione si vede per\`o con il TX0\footnote{La programmazione in assembler ammetteva 4 istruzioni}, dove la politica di utilizzo era un po' pi\`u ``rilassata''.
Caratteristiche principali della cultura hacker:
\begin{itemize}

\item Giocosit\`a
\item Condivisione degli sforzi di sviluppo (programmare sulle macchine di quel tempo richiedeva una notevole mole di lavoro)
\item Politica di apertura agli esterni (ovviamente se si era capaci). Peter Deutsch, ad esempio, si avvicin\`o alla comunit\`a del MIT molto presto.
\item Lavoro notturno\footnote{Permetteva un utilizzo maggiore delle macchine perch\`e la notte si aveva una minore affluenza di personale}

\end{itemize}

Nel 1961 arriva il PDP-1, primo computer concentrato sull'utente piuttosto che sull'ottimizzazione delle risorse, offrendo un approccio interattivo e inserendo nuovi tool per lo sviluppo. La progettazione \`e assegnata a Gurley (ex membro del MIT), che si basa sul TX-0 e TX-2. Ci\`o porta a un rafforzamento della comunit\`a hacker, grazie anche alla nascita di ARPANET.

\subsubsection{Etica hacker}

\begin{itemize}
  
\item L'accesso al computer, e a tutto ci\`o da cui si pu\`o imparare dovrebbe essere illimitato e totale. L'apprendimento avveniva non per basi teoriche, ma per basi pratiche (si voleva che il computer venisse ``aperto'' e studiato). Il ``Midnight Requisitioning Committee'' era un comitato non ufficiale che si occupava di requisire componentistica elettronica principalmente inutilizzata per poterla riusare.

\item L'informazione dovrebbe essere libera: senza l'informazione \`e impossibile capire e quindi migliorare i sistemi, con l'idea che l'informazione dovrebbe essere come il flusso di bit in un computer. Era inconcepibile quindi che un software fosse proprietario.

\item ``Non fidarti dell'autorit\`a e promuovi la decentralizzazione''. Secondo la comunit\`a hacker, l'autorit\`a porta con s\`e la burocrazia, che promuove regole arbitrarie che mirano solo alla propria perpetuazione.

\item Gli hacker dovrrebbero essere giudicati per il loro valore, non sulla base di fattori come razza, religione, sesso o posizione sociale. L'unica cosa che conta \`e quanto l'hacker pu\`o contribuire all'avanzamento dello stato dell'hacking.

\item \`E possibile creare arte e bellezza su un computer. Infatti i computer di quel tempo disponevano di pochissime risorse, e l'eleganza nella programmazione \`e vista come bellezza.

\end{itemize}
 
\subsubsection{Incompatible Timesharing System}

Con le macchine non-timesharing si avevano code lunghe per l'utilizzo del computer, causando problemi. Il progetto MAC puntava alla costruzione di una rete come quella elettrica destinata a distribuire potenza di calcolo.

Il timesharing era mal visto dagli hacker, che gli ricordava il Multics o CTSS (Compatible Time Sharing System), in quanto non si poteva avere controllo totale della macchina. Alcuni programmi che gli hacker sviluppavano avevano veramente bisogno di tutte le risorse del computer disponibili. Era quindi necessario aver bisogno di un accesso totale, e si arriv\`o a un compromesso: timesharing durante il giorno e single mode durante la notte, e per questo era necessario uno sviluppo di un SO in timesharing ispirato all'etica hacker: ITS\footnote{Incompatible Timesharing System}.

L'incompatible timesharing system aveva le seguenti funzionalit\`a:
\begin{itemize}

\item Utenti multipli ma anche programmi multipli per ogni utente
\item No password e assenza di sistema di permessi
\item File personali per ogni utente, ma disponibili a tutti
\item Strumenti collaborativi (es: possibilit\`a di switchare terminale e passare al terminale di un altro utente)
\item Fede negli utenti

\end{itemize}

L'ITS era quindi un sistema vivo che cresceva  con i suoi utilizzatori.

Con l'avvento del nuovo sistema, il PDP-10, si ha una crisi: viene infatti proposto di utilizzare TENEX, un nuovo OS al posto di ITS. Questo porta nel 1968 alle manifestazioni contro il laboratorio, con conseguenti atriti.

\subsubsection{Stallman - L'ultimo vero hacker}

Richard Stallman nasce a New York nel 1953. Ha le prime esperienze con i computer all'IBM New York Scientific Center ed entra nel laboratorio del MIT attorno agli anni '70, quando il laboratorio si sta avvicinando alla fine. Non era un informatico, ma era interessato all'informatica in generale. Nel 1971 entra a Harward, ma si interessa di pi\`u al MIT, dove viene assunto da Russel Noftsker come programmatore di sistema. Stallman quindi si avvicina alla cultura hacker, e lavora insieme a Richard Greenblatt e Bill Gosper\footnote{Pi\`u incline nella risoluzione di problemi ambito ``lato matematico''.}.

\paragraph*{Emacs} (Editing Macros) Dalla telescrivente si hanno dei passaggi verso i primi monitor con una riga. I primi programmi per gli editor sono Expensive Typewriter e TECO\footnote{Type Editor and Corrector}, che fungevano da insieme di strumenti a cui si lavorava applicando un insieme di comandi al testo. Questi tool erano scomodi da usare, quindi Stallman decise di cercare un altro tipo di software altrove. Non trovandolo, fece un insieme di MACRO per TECO, che rendeva possibile:
\begin{itemize}
  
\item Editare il testo \textit{real time}
\item Permettere il \textit{random access editing}
\item Consentire l'aggiunta di ulteriori MACRO

\end{itemize}

La nascita di Emacs si ha dal caos dovuto alle troppe MACRO nate. Steele propone di generare un ordine nell'universo delle MACRO. Inizialmente impone una clausola che imponeva che ogni modifica a Emacs fosse inviata allo sviluppatore principale, in modo tale che, se fosse stata un'ottima idea, sarebbe stato possibile integrarla in Emacs e renderla disponibile a tutti. Questa clausola permise alle persone di lavorare insieme agli altri in una community ma impose una limitazione alla libert\`a di sviluppo.

\subsection{La crisi del movimento Hacker del MIT}

\paragraph*{Le prime incursioni} Con l'avvento delle password, Stallman cerca di convincere le altre persone a limitarne l'utilizzo, e a permettere agli altri utenti di utilizzare i file di tutti. Questo porta all'intervento del ministero della Difesa, che lo costringe all'uso di password. Ci\`o aliena la comunit\`a vicino a Stallman, anche a causa dell'introduzione dello \textit{sciopero del software}:
Stallman rifuita di fornire allo staff del laboratorio le ultime versioni di Emacs fino a quando non avessero eliminato il sistema di sicurezza nel laboratorio. La time bomb di Scribe fa prendere a Stallman la scelta di opporsi alle restrizioni sull'utilizzo del software.

\subsubsection{Cambiamento di mentalit\`a}

Nel 70-80 si ha una frammentazione della cultura hacker (nel 1976 si ha il copyright act), causata dal fatto che gli Hacker originari abbandonano il MIT per lavorare-aprire la propria azienda. Si ha un cambiamento dei visitatori al MIT, e con essi cominciano a essere presenti i primi programmi protetti da copyright nel laboratorio di intelligenza artificiale.

Questo a causa di un cambiamento degli equilibri di forze tra gli hacker e gli studenti-professori, che hanno un modo completamente diverso di vedere il software (ovvero come uno strumento).

\paragraph*{La lisp machine} La nascita della lisp\footnote{Linguaggio al tempo ad alto livello, molto potente ma macchinoso} machine causa la crisi finale. Dall'idea avuta da Greenblatt, viene costruita una macchina concepita per funzionare in sintonia con Lisp, che ebbe successo. Con i copiosi fondi del progetto, al MIT vennero prodotte 32 macchine, che si voleva far comunicare in rete per favorirne la condivisione. Ci\`o porta all'idea di creare un'azienda ``hacker friendly'' per la produzione di lisp machine. Da ci\`o si ha lo scontro con Russel Noftsker, che propone di creare un'azienda ``per azioni'' e renderla prettamente commerciale. Dato lo scontro, viene raggiunto un accordo: a Greenblatt viene dato un anno di tempo per creare un'azienda per la vendita di lisp machines, che riesce a far partire entro l'anno (Lisp Machine Inc., LMI). Nonostante ci\`o viene creata un'altra azienda, la Symbolics. Ci\`o caus\`o:
\begin{itemize}

\item Svuotamento del MIT
\item LMI e soprattutto Symbolics attingono pesantemente dal MIT
\item 1982: Symbolics rende le proprie modifiche al SO delle lisp machines proprietario, causando la vendetta di Stallman, che si mise a replicare tutte le funzionalit\`a e a donarle a LMI.
\item La comunit\`a hacker s'indebolisce, perch\`e minoritaria rispetto a studenti e professori, e diventa difficile sostenere dentro il laboratorio un sistema proprio interno.

\end{itemize}

\paragraph*{Cambiamento del PDP} Con il cambiamento del PDP tutto il software relativo diventa obsoleto. Ci\`o apre un fronte:
\begin{itemize}

\item Continuare a utilizzare ITS $\to$ diventato vecchio e poco sicuro
\item Usare Twenex, software proprietario derivato da Tenex (verr\`a scelto quest'ultimo)

\end{itemize}

\subsubsection{Nasce GNU}

GNU nasce grazie al caso della stampante Xerox, che veniva venduta con un software sotto la media e che tendeva a far inceppare spesso la stampante. Il software dato era proprietario, e i sorgenti sotto NDA\footnote{Atto di non divulgazione}. Stallman non poteva quindi migliorare il driver della stampante, e con ci\`o lasci\`o il MIT e and\`o a fondare il progetto \textit{GNU} nel 1983.

\paragraph*{L'appello} Nel 1983, su net.unix-wizard Stallman lancia dopo il giorno del ringraziamento un nuovo software Unix compatibile, che chiam\`o GNU\footnote{Gnu is Not Unix}. Unix venne scelto come base perch\`e:
\begin{itemize}

\item AT\&T, la proprietaria di Unix, non poteva venderlo causa della sua posizione dominante nella telefonia
\item Familiarit\`a con il codice sorgente
\item Portabilit\`a: Unix era stato sviluppato in un linguaggio creato appositamente per lui: il C.
\item Modularit\`a

\end{itemize}

Avendo bisogno di un compilatore, si mise a lavorare su Pastel, un compilatore libero. Purtroppo la struttura troppo pesante impediva a Pastel di lavorare su macchine poco potenti, e con ci\`o si mise alla stesura di GNU Emacs partendo dal codice di Gosling, reimplementando alcune funzionalit\`a sotto le minaccie legali di Unipress. Nel 1985 Stallman rilascia GNU Emacs, ponendo il problema di quale licenza usare per quel programma.

\paragraph*{GPL} Inizialmente, la licenza di GNU Emacs era ispirata dalle note di ``copyright'' sulle email. Inoltre, la diffusione del software in modo centralizzato, a differenza che con la commune. Per suggerimento di Gilmore si ha un cambio nome: si ha quindi la nascita della \textbf{GNU general public license}. GNU gpl v1 viene distribuita con il rilascio di gdb, con l'idea di unificare in un'unica licenza tutto il software GNU.

\paragraph*{L'incontro con BSD} Con la rottura del monopolio, AT\&T comincia a focalizzarsi sullo sviluppo di Unix a scopi commerciali. BSD era una distribuzione accademica derivata da Unix con vari contributi esterni ma richiedeva comunque il pagamento di una licenza AT\&T perch\`e non tutto il codice era di Berkeley. Nel 1984-1985 Stallman convince Bostic a sviluppare una distribuzione completamente libera.

Negli anni 80-90 Bruce Perens rilascia electric fence\footnote{Una libreria scritta in C} sotto GPL. Rich Morin fonda Prime Time Freeware, un'azienda che ricercava software open source nella rete e si occupava di rivendere i nastri.

La Cygnus era un'azienda che aveva cominciato a lavorare su gcc\footnote{L'idea era quella di contribuire a gcc e rivenderlo.}, facendone il porting al National Semiconductor's 32032. Data la modularit\`a di gcc, implement\`o il supporto a C++ e si occup\`o di portare il compilatore per sistemi embedded. Alla fine del 1990 aveva fatto $725.000$ \$ in supporto e contratti. L'azienda \`e stata ora venduta alla Red Hat.

\subsubsection{Espansione del progetto GNU}

Nel 1987 si ha la nascita di Libc:
\begin{itemize}

\item 1990 fork Linux
\item 1997 fork abbandonato

\end{itemize}

\paragraph*{GNU Hurd} Nel 1986 si ha il tentativo di basarsi su TRIX. Ma ci\`o causava il non funzionamento su macchine standard, e ci\`o portava a un numero troppo elevato di cambiamenti. Si prova quindi a basarsi sul codice BSD, ma si ha poca cooperazione da parte degli sviluppatori e si preferisce un approccio pi\`u ambizioso: basarsi su un microkernel usando \textit{Mach}. Nel 1990 iniziano i lavori sul kernel, ma si incontrano difficolt\`a di sviluppo, aggravata dalla poca attenzione dovuta dall'avvento di Linux. Attualmente Mach supporta i driver linux, supporta X, iceweasel e forse debian render\`a ufficiale una prossima release.

\subsection{BSD}

Testo di riferimento: \href{http://www.groklaw.net/staticpages/index.php?page=20051013231901859}{\textit{The Daemon, The GNU and the Penguin}}

\subsubsection{DARPA}

Nel 1957 si ha il lancio dello Sputnik1 da parte dell'URSS nello spazio. Gli americani, nel 1958 fondano l'ARPA (poi rinominata DARPA), \footnote{il cui scopo era lo sviluppo di nuove tecnologie per scopi militari} in cui veniva sviluppato MULTICS.

Nel 1963 nasce il progetto MAC\footnote{Multiple Access Computer, Machine Aided Cognitions}, sovvenzionato con due milioni di dollari dal DARPA. Gli obiettivi iniziali del progetto MAC erano quelli di rendere possibile l'affitto di potenza computazionale con la creazione di sistemi affidabili come quelli per la distribuzione di energia elettrica.

\paragraph*{Multics} Primo sistema operativo Hight Availability. \begin{itemize}

\item Sviluppato da MIT, General Eletric e Bell Labs
\item Struttura modulare: possibile aumentare le prestazioni del sistema semplicemente aggiungendo una ulteriore unit\`a (CPU, memoria, storage etc)
\item Riconfigurazione on-line
\item Linkaggio dinamico, filesystem gerarchico etc

\end{itemize}

Multics caus\`o un fallimento commerciale per via dell'estrema complessit\`a del sistema.

\paragraph*{Unix} Nel 1969 AT\&T si toglie dal progetto Multics. Dennis Richie e Ken Thompson desiderano continuare la ricerca sulle idee di Multics. Viene creata quindi una versione pi\`u leggera, compatibile con macchine pi\`u piccole, chiamata \textit{UNIX}. Nel 1972 Unix viene riscritto in C, per avere una maggiore portabilit\`a e maggiore facilit\`a di sviluppo. Nell'Ottobre del 1973 un articolo fa espandere la popolarit\`a di Unix.
Con il monopolio telefonico di AT\&T Unix viene distribuito liberamente, con la possibilit\`a di far liberamente delle modifiche (con codice sorgente). Nel 1977 John Lion pubblica il codice sorgente commentato di Unix, che causa un incremento dell'insegnamento di UNIX all'universit\`a, nonostante il tentativo di AT\&T di bloccare ci\`o. Conseguentemente, due anni dopo, AT\&T annuncia una restrizione sulla redistribuzione di Unix, che implicava sia una restrizione a livello commerciale sia una restrizione sulla possibilit\`a didattica nell'utilizzo di Unix.

Nel 1973 John Fabris dell'universit\`a di Berkeley assiste al talk su Unix al SOSP\footnote{Symposiuom on Operating System}, e decide di fare un dual boot in una sua macchina. Nel 1975 vengono acquistate altre due nuove macchine Unix. Berkeley ha bisogno di un certo supporto ai suoi sistemi, e nel 1975 Chuck Haley e Bill Joy arrivano a Berkeley, cominciando lo viluppo di un compilatore e editor per Pascal. Quello che diede una spinta decisiva a BSD fu che DARPA decide di muovrere la loro DARPANET su UNIX. Nel 1980 i fondi di DARPA vengono usati per il miglioramento di BSD UNIX, che rilascia 4BSD. In tutto questo, AT\&T annuncia di voler commercializzare Unix ed entra in conflitto con BSD. Nel 1983 Unix viene completamente commercializzato, con licenze sui sorgenti molto costose, in particolare per il TCP. In tutto ci\`o BSD cerca di liberare la dipendenza da AT\&T, che viene coronata con NET2. BSD386 (poi rinominato BSDI), distribuzione completa basata su NET2 e a stampo commerciale nasce in quell'anno, facendo scaturire una causa tra Unix System Laboratories (USL), BSDI e Berkeley, finita con la vittoria di BSDI.
