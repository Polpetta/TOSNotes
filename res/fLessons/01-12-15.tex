\section{Lezione del 01-12-15}
\graphicspath{ {res/data/01-12-15/} }

\paragraph*{Collezioni RDF}

I container RDF non sono ``bloccabili''. Quindi, se necessario c'\`e il bisogno dell'uso delle collezioni. Le collezioni vengono viste come liste singolarmente linkate, e si accede tramite le seguenti voci: \textit{rdf:first, rdf:rest, rdf:nil}

\paragraph*{Reification}

I reification permettono di esprimere predicati che parlano ad altri predicati. Per esprimere ci\`o esistono dei predicati standard da applicare ai nodi anonimi:
\begin{itemize}

\item rdf:Statement
\item rdf:subject
\item rdf:predicate
\item rdf:object

\end{itemize}

\subsubsection{RDFa}

Il linguaggio RDFa \`e un linguaggio di annotazione rdf per XHTML, compatibile con i namespace.

\paragraph*{Flusso RDFa}Inizialmente il soggetto corrente \`e il documento corrente, avendo esso stesso un url. Esegue una scansione depth-first (lavorando in maniera ricorsiva), cercando un tag che specifica il predicato. Se \`e presente un oggetto e un soggetto, ottiene una tripla completa, aggiungendolo al grafo. Il soggetto iniziale \`e sempre il tag ``base''. \newline

RDFa supporta una \href{http.//www.w3.org/2011/rdfa-context/rdfa-1}{lista di prefissi iniziali predefiniti}.

\newpage

\paragraph*{Property}Il tag property setta l'oggetto di una propriet\`a. Ad esempio il codice:
\begin{verbatim}

<p>Il contenuto di questo sito e' licenziato sotto
<a property="cc:license" 
href="http://creativecommons.org/licenses/by/3.0/">CC-BY</a>
</p>
\end{verbatim}

Viene schematizzata nella seguente maniera:

\begin{figure}[h]
  \centering
  \includegraphics[scale=0.6]{01-12-15-01}
\end{figure}

\paragraph*{Content}Specifica il contenuto da usarsi a fini RDFa quando il contenuto del tag corrente \textbf{non \`e desiderabile} o \textbf{non \` disponibile}. Ci\`o avviene quando si vuole sostituire il contenuto non appropriato con un alternativo per qualche ragione (es. data americana con data europea).

\paragraph*{Vocab}Setta un vocabolare di default per tutti i discendenti. Pu\`o essere mescolato con full uris, e pu\`o essere settato su diversi livelli dell'albero xhtml. Utile soprattutto quando si hanno molte propriet\`a dello stesso vocabolario.

\paragraph*{Prefix}Permette di utilizzare velocemente pi\`u di un prefisso. Questo sistema permette di essere mescolato con vocab.
Ad esempio:
\begin{verbatim}
<body prefix="p1: http://www.p1.org/ p2: http://www.p2.org">
<span property="p1:property">properieta' 1</span>
[...]
</body>
\end{verbatim}

\paragraph*{About}L'attributo about permette di settare il soggetto, che diventer\`a il soggetto corrente all'interno del tag
Es:
\begin{verbatim}
<div about="/barbecue">
[...]
</div>
\end{verbatim}

Mescolando about e property si ha il setting contemporaneo di soggetto e predicato. \`E importante notare che l'ordine con cui gli attributi vengono inseriti \`e importante. Ad esempio:
\begin{verbatim}
<p>frammento di codice rilasciato sotto
<span about="#code" property="license">
cc-by
</span>
</p>
\end{verbatim}

L'about permette non solo di settare il soggetto, ma di settare il soggetto in modo anonimo. Ad esempio:
\begin{verbatim}

<link about="[_:n1]" rel="foaf:mbox" href="mailto:john@ex.org" />
\end{verbatim}

\paragraph*{Typeof}
Typeof pu\`o essere usato in due situazioni:
\begin{itemize}
  
\item Settare un tipo per un soggetto $\to$ funziona solamente inserendo anche l'attributo about
\item Creare un nodo anonimo usato come base per i predicati $\to$ esempio:
\begin{verbatim}
  <div typeof="foaf:Person">
    <p propery="foaf:name">Alice</p>
    <p>
      Email
      <a rel="foaf:mbox" href="mailto:alice@example.com">
        alice@example.com</a>
    </p>
    <p>Phone: <a rel="foaf:phone"
      href="tel:0444123456">0444123456</a></p>
  </div>

\end{verbatim}

\end{itemize}

\paragraph*{Rel}L'attributo rel collega l'oggetto all'attributo resource, href o src o ai nodi contenuti. Rel e Property presentano delle differenze tra loro:
\begin{itemize}

\item Il Rel non si collega al contenuto interno del tag e nemmeno al valore di content
\item Rel supporta il \textit{chaining}
  
\end{itemize}

Il chaining \`e uno strumento molto potente, che pu\`o concatenare oggetti esterni e soggetti interni tra loro. Ad esempio:
\begin{verbatim}
<a about="http://www.debian.org" rel="dc:creator"
href="http://www.ian.org">
<span property="foaf:name">Ian Murdock</span>
</a>
\end{verbatim}

Come si pu\`o vedere, lo span di foaf:name \`e si lega a www.ian.org e a Ian Murdock: ovvero l'oggetto della proposizione esterna diventa soggetto di quella interna.

\`E anche possibile che il soggetto della proposizione interna diventi oggetto di quella esterna. Ad esempio:
\begin{verbatim}
<div about="#me" rel="foaf:knows">
  <div about="http://www.w3.org/People/Ivan/#me"
   property="foaf:name" content="Ivan Herman" />
</div>
\end{verbatim}

\begin{figure}[h]
  \centering
  \includegraphics[scale=0.5]{01-12-15-02}
  \caption{Schema dell'esempio sopra riportato}
\end{figure}

\`E anche possibile il \textit{chaining con intermediario anonimo}. Se non viene specificato n\'e un oggetto per la proposizione esterna n\'e un soggetto per quella interna, il collegamento \`e fornito da un nodo anonimo creato automaticamente.

\paragraph*{Datatype}Datatype pu\`e venire usato per esprimere il tipo di un letterale.
Per esempio \`e possibile dichiarare l'et\`a di un sito.

\subsubsection{RDF Schema}
Descrive il vocabolario usabili in RDF. Si dividono in vari elementi.

\paragraph*{Classi}Collezione di elementi, dette istanza collegate agli elementi attraverso rdf:type. \`E possibile creare gerarchie gerarchiche. Si ddice che una classe A sia sottoclasse di B se ogni istanza di A \`e anche istanza di B.

\paragraph*{Propriet\`a}Connessioni tra i vari elementi. \`E possibile definire delle \textbf{sottoproriet\`a}, in modo che se \`e presente una propriet\`a allora vale anche un'altra. Inoltre \`e possibile definire \textbf{range} e \textbf{domain}.
