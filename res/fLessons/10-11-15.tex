\section{Lezione del 10-11-15}

\subsubsection{Affero GPL v3}

Creata dalla Affero inc. per regolare i diritti sui servizi con codice coperto da GPL v3. La Affero GPL v3 \`e stato adottata dalla GNU.

\subsubsection{Mozilla Public License}

Originariamente nata dalla Mozilla per il rilascio del codice di Netscape, essa voleva rendere possibile l'aggiunta di software proprietario ``di contorno'' (come per esempio dei plugin), e quindi fu creata una licenza (inizialmente chiamata NPL) MPL che fu una via di mezzo tra la GPL iniziale e una BSD.

Con la Mozilla Public License \`e possibile avere una integrazione con le licenze proprietarie. Questo in quanto inizialmente voleva essere protetto il codice originario di Netscape. Con la Mozilla Public License viene consegnata una licenza brevettuale per i brevetti che sono necessari per l'utilizzo di quella versione del software.

\subsubsection{Perl Artistic License}

La Perl Artistic License non \`e pensata per diffondere, come nel caso della GPL, il software libero, ma \`e pensata per permettere all'autore del proprio software un controllo ``artistico'' sulla sua opera.

In questa licenza viene privilegiato il copyright holder. Il software in s\`e non necessita alcun pagamento, e il destinatario ha la possibilit\`a di distribuire il software agli stessi termini.

\paragraph*{Redistribuzione dei sorgenti}Il codice \`e redistribuibile mantenendo la nota di licenza originale, ma \`e possibile incorporare codice che sia o sotto dominio pubblico o sia scritto dallo stesso autore che detiene il copyright. Per la redistribuzione dei sorgenti modificabili \`e possibile rendere nel dominio pubblico le modifiche, o renderle disponibili gratuitamente. \`E possibile distribuire localmente il codice (per esempio per una propria azienda) con la licenza che si desidera. Se invece si vuole distribuire al pubblico, bisogna allora rinominare tutti i file modificati, in modo che sia chiaro che quella versione \`e stata modificata e non sia possibile confondersi con la versione originale.

\paragraph*{Redistribuzione dei binari}\`E possibile redistribuire direttamente i binari, a patto che siano disponibili anche i sorgenti originari con delle istruzioni su come prelevare la versione standard. Altres\`i \`e possibile distribuire insieme ai binari modificati anche i sorgenti modificati, oppure \`e necessario fare la rinominazione degli eseguibili e documentare le differenze.

\paragraph*{Uso commerciale}Non \`e possibile usare questa licenza per software con scopi commerciali, in quanto la vendita del software \`e vietata. \`E permesso ricevere un pagamento per il supporto e \`e possibile l'integrazione in superpacchetti commerciali. L'uso del software deve essere isolato dall'utente.

\subsection{Creative Commons}

Testi di riferimenti: \textit{Viral Spiral, How the Commoners Built a Digital Republic of Their Own - David Bollier} e \textit{Creative Commons: a user guide - Simone Aliprandi}

\subsubsection{GNU Free Documentation License}

Nata come una licenza sulla documentazione del software, in quanto problema molto sentito e non trattato dalla GPL (alcune tematiche riguardo alla trattazione dei documenti non vengono considerate).

Obiettivi della GNU Free Documentation License:
\begin{itemize}

\item Libert\`a di modifica
\item Tutela dei diritti morali dell'autore $\to$ per gli autori \`e possibile che ci siano parti importanti da non modificare, quindi \`e stata introdotta la possibilit\`a di inserire alcune sezioni non varianti, che possono essere tutte quelle sezioni che non sono l'argomento principale dell'opera.
\item Problema gestione delle copie non trasparenti $\to$ viene regolato il modo in cui viene gestita la redistribuzione per grandi quantit\`a 
\item Copia in grande quantit\`a e non

\end{itemize} 

La licenza suddivide il documento in varie parti:
\begin{itemize}
  
\item Documento
\item Titolo
\item Sezioni secondarie e invarianti
\item Testi di copertina
\item Storia del documento $\to$ changelog all'interno dell'opera
\item Licenza
  
\end{itemize}

La GNU FDL impone delle restrizioni sulle redistribuzioni senza modifica:
\begin{itemize}

\item Mantenimento della licenza
\item No misure tecnologiche di restrizione
\item Esibizioni pubbliche permesse
\item Redistribuzioni voluminose:
  \begin{itemize}

  \item Obbligo di identificarsi come editore
  \item Obbligo di mantenere titolo e testi di copertina
  \item Obbligo di distribuire il sorgente

  \end{itemize}

\end{itemize}

Se invece vengono applicate delle modifiche \`e necessario:
\begin{itemize}

\item Modifica del titolo
\item Indicazione degli autori delle modifiche e del documento
\item Rimozione dei ``Riconoscimenti''
\item Preservazione della sezione delle dediche
\item Aggiornamento della sezione cronologia
\item Preservazione degli invarianti\footnote{Anche le traduzioni vengono considerate delle modifiche}
\item Preservazione della versione trasparente
\item Preservazione e/o aggiunta dei testi di copertina

\end{itemize}

\`E presente una clausola speciale, creata per Wikipedia, che sotto specifiche condizioni permettevano di convertire la licenza sotto CC-BY-SA (la licenza Creative Commons pi\`u vicina alla FDL). Questa clausola, temporanea, \`e ora scaduta.

Nei documenti ha senso trattare i vari tipi di unioni (a differenza della GPL), ed \`e possibile mantenere solo una licenza nel documento.
Nella traduzione dei documenti \`e possibile tradurre anche la licenza, ma \`e necessario mantenerne una versione in inglese.

La GNU FDL \`e un ``ponte'' tra le licenze software e le licenze sui documenti.

\subsubsection{Transizione alle Creative Commons}

Creata da un insieme di persone e dopo molti anni.

\paragraph*{Background}Negli anni '60 si ha un aumento pervasivo del copyright, fino al culmine degli anni '80 quando il giornale Nation prese un estratto dal libro del presidente Ford, causando una causa legale vinta da Ford. Negli anni successivi ci furono i cambiamenti dei mezzi d'informazione, coinvolgendo nella copia anche le persone, causando una maggiore sensibilizzazione del copyright al grande pubblico.

\paragraph*{Inizio della crisi}Nel 1990 si ha la nascita di EFF, associazione molto attiva che si contrapponeva alle stringenti regole del copyright dettate dalle lobby. Si ha anche la nascita della National Information Infrastructure creata da parte di Clinton che cercava di portare ordine su Internet, in cui vennero proposte tutte una serie di modifiche restrittive nei confronti del pubblico. Questo caus\`o una forte reazione, con la pubblicazione di un articolo nel 1994 dal The Economy of Ideas di Barlow, accumunando molte persone. Tutto ci\`o sfoci\`o nel 1995 nella creazione della Digital Future Coalition, che si contrapponeva al DMCA.
Nel 1996 si ha la Dichiarazione d'indipendenza del cyberspazio, in cui viene dichiarata l'internazionalit\`a dei contenuti del web e dove viene affermato che gli stati non dovrebbero averne influenza.
Nel 1999 si ha la conferenza ``Private censorship/perfect choice''.
Alla fine degli anni novanta si valorizza il dominio pubblico da parte di queste comunit\`a createsi, che considerano i lavori di dominio pubblico una base per la nascita di futuri progetti.
