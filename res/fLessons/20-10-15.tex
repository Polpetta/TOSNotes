\section{Lezione del 20-10-15}

USL perde la causa, ma si ha un rallentamento della diffusione di BSD e nel 1994 si ha la definitiva chiusura del processo tra Berkeley e Novell. Nello stesso anno viene rilasciato 4.4BSD-Lite, una versione di BSD completamente libera\footnote{Ci furono 2 release, una con funzionalit\`a in pi\`u in cui erano presenti sorgenti AT\&T in cui bisognava pagare licenze aggiuntive}.

\subsection{Linux}

Materiale di riferimento: \textit{The Deamon, the GNU and the Penguin: a history of Free and Open Source}; Peter Salus.

\subsubsection{Minix}

\paragraph*{Premesse} John Lions pubblica il codice sorgente commentato di UNIX. Nel 1978-1979 vengono bloccati i commentari di John Lions, e si ha un aumento dei costi delle licenze e restrizioni sull'insegnamento in classe. Questo causa l'interruzione dell'insegnamento di UNIX in molte universit\`a.

\paragraph*{Minix} Andrew Tanenbaum, docente di Computer Science, decide di scrivere un proprio sistema operativo sulle traccie di Unix: Minix. Minix \`e V7 compatibile, completo di compilatore ed editor. Era pensato per scopi didattici, era rilasciato sotto licenza permissiva e non libera.

\paragraph*{Nascita di Linux} Nel 1990 l'universit\`a di Helsinki installa un MicroVax con Ultrix. Linus studia il libro di Tenenbaum, ma non possiede un computer con Unix per mettere in pratica ciò che ha imparato. Segue il corso di "C e Unix". Nel 1991 Linus viene portato alla conferenza di Stallman e nello stesso anno compra un PC e comincia a scriverci un emulatore di terminale a partire da Minix. Sempre nello stesso anno Linus rilascia Linux 0.0.1\footnote{All'inizio il nome non si chiamava Linux, ma Freax e veniva distribuito sotto licenza proprietaria, in quando non permetteva l'utilizzo commerciale}. Nell'anno successivo, con Linux 0.12\footnote{Si ha la prima richiesta di una nuova feature da parte di un ragazzo che aveva a disposizione un computer con poca RAM. Cosí Torvalds implementa il paging.} Linus cambia la licenza adottando la licenza GPL, dando la possibilit\`a a tutti a contribuire, comprese aziende come SUSE. Si ha anche la prima distribuzione Linux: Linux Pro con Yggdrasil\footnote{Un database inizilamente solo per UNIX}. Si ha anche il diverbio tra Tanembaum e Linus. Nel 1994 si arriva alla versione di Linux 1.0.

Nel 1993 Mark Ewing fonda RedHat, una distribuzione che viene rilevata da Bob Young e diventa RedHat Software Inc. Nel 1995 nasce RedHat Linux, che diventa la pi\`u diffusa distribuzione Linux, ora a pagamento. Vengono effettuati anche i primi porting di Linux a DEC Alpha e SUN SPARC, mentre nel 1996 si ha il supporto per multiprocessore. Nel 2011 si arriva a Linux 3.0 e nel 2015 Linux 4.0.

\subsubsection{Debian}

Distro guidata dalla nuova generazione hacker di Linus. Linux come kernel stava prendendo suo vigore, creando una comunit\`a a sé stante allontanandosi dal software libero. Un motivo di ci\`o fu la causa che si ebbe nel 1988 tra Apple e Microsoft, che impegn\`o lo GNU nella LPF\footnote{League of programming freedom, orgranizzazione che si oppene ai brevetti software e ai copyright delle interfacce utenti}, impedendo che passasse il principio di copyright nella interfaccia grafica. Murdock allora per riavvicinare la comunit\`a Linux a GNU annuncia la sua intenzione di fare una distribuzione completamente libera. Avendo possibilit\`a di focalizzarsi sul progetto, nel 1996 viene rilasciata la versione 1.1, dopo la versione erronea 1.0 del 1995. Con Bruce Perens come nuovo DPL vengono scritte le Debian Free Software Guidelines (DFSG), dando un insieme di caratteristiche che le licenze dovevano avere per essere considerate conformi agli standard del parco software Debian. Su queste guide furono date le future definizioni di Open Source. Venne anche creato l'Open Hardware Certification Program, programma con il quale le aziende potevano certificarsi per essere compatibili con Linux. Il Software in the Public Interest\footnote{Organizzazione non a scopo di lucro di supporto alla creazione e diffusione di software libero.} era un ombrello legale tramite cui la Debian diventa un'entit\`a in cui venne riconosciuta dai governi.

\paragraph*{Caratteristiche} Debian presenta il patto sociale, dove si ha un'attenzione maniacale alla qualit\`a, in cui si adotta solo software libero (secondo la DFSG) e dove gli sviluppi e decisioni sono presi in maniera comunitaria. Data l'impossibilit\`a di adottare solamente software libero, la Debian ha messo a disposizione (anche se non supporta ufficilamente) software non-free (ovvero costituito da software proprietario) e contrib (software libero che per funzionare ha bisogno di software proprietario). Diversi tipi di distribuzioni:
\begin{itemize}
  
\item Stable: versione rilasciata per il pubblico utilizzo.
\item Testing: entrano tutti i pacchetti che dopo 15 giorni di permanenza in unstable non hanno bug critici.
\item Unstable: versione per sviluppatori altamente instabile.

\end{itemize}

\subsection{Open Source}

Libro: \textit{Opensources: voices from the Open Source Revolution}

Nel 1997 Raymond scrive il libro \textit{The Cathedral and the Bazaar}, dove viene analizzato dal punto di vista architetturale Linux, che dall'esterno sembra destinato al collasso in quanto i collaboratori sono dei terzi che quindi hanno la possibilit\`a di sviluppare e dirottare il progetto (per questo che viene associato al Bazaar). Analizzado tutto ci\`o si riesce a dettare delle linee guida sullo sviluppo di una comunit\`a open source:
\begin{itemize}

\item Ogni progetto deve partire da un ``prurito'' del programmatore
\item I buoni programmatori sanno cosa scrivere. I grandi programmatori sanno cosa riscrivere
\item Tratta i tuoi utenti come co-sviluppatori
\item Rilascia presto e spesso e ascolta i tuoi utenti
\item Dato un sufficiente numero di beta tester ogni problema verr\`a identificato e risolto
\item Riconoscere le buone idee degli utenti \`e importante come averne di proprie

\end{itemize}

Grazie a questi principi, nel 1998 netscape (il primo browser grafico) annuncia la volont\`a di liberare il proprio codice sorgente, rendendo Raymond una celebrit\`a. Dopo il rilascio si cerca una strategia di lungo termine per mettere in vista l'Open Source, e si decide di lanciare una campagna di marketing, denominata ``Open Source'' sostituendo la nomenclatura ``Free Software''. Torvalds appoggi\`o l'iniziativa Open Source, con l'idea di usare DFSG come definizione di Open Source, e nel 1998 nasce la Open Source Initiative (1998) fondata da Raymond e Perens, causando il coinvolgimento di persone e di programmi. Stretegie del movimento open Source:
\begin{itemize}

\item Approccio top-down
\item Puntare su Linux come dimostrazione $\to$ in quanto \`e molto software libero, e Linux aveva il fatto che poteva affascinare molte persone, dimostrando la forza del software Open Souce.
\item Puntare sulle Fortune 500 $\to$ i volontari tentano di puntare sulle piccole aziendine
\item Puntare sui media che influenzano le grosse aziende
\item Educare gli hacker sulle tattiche di promozione da seguire $\to$ troppa campagna portava ad avere effetti negativi
\item Usare un programma di certificazione $\to$ vengono rilasciate delle certificazioni per poter dichiarare se un software \`e open source o meno.

\end{itemize}

Il 7 marzo 1998 al Free Software Summit una ventina di leader del movimento appoggiano l'iniziativa, e si cerca di spingere altre aziende a seguire l'esempio di Netscape (Corel annuncia computer basati su Linux seguita da Oracle e Informix che annunciano il porting su linux).

I documenti Halloween sono dei documenti rilasciati erroneamente dalla Microsoft dove vengono sottolineati tutti i pericoli strategici che potevano venir dal movimento Open Source, e il fatto che colossi come Micosoft fossero preoccupati sottoline\`o l'importanza che questo movimento aveva preso.

\paragraph*{OSI} Il movimento Open Source prese una strategia diversa dal software libero, ovvero partendo da una serie di principi cercare una strategia per aumentare la visibilit\`a e raggiungere persone e aziende prima non possibile. La filosofia consisteva in:
\begin{itemize}

\item Licenze libere e permissive $\to$ per poter contribuire allo sviluppo di software in maniera comunitaria
\item Costruzione di una comunit\`a attorno al software $\to$ ci\`o permette alla comunit\`a di comprendere determinate scelte, facendo sentire la comunit\`a parte del processo decisionale
\item Trasparenza del processo di sviluppo $\to$ senza la possibilit\`a di modificarlo non \`e possibile interagire con il software e con la comunit\`a di utenti
\item Codice sorgente liberamente disponibile
\item Codice sorgente liberamente modificabile
\item Libera redistribuzione $\to$ insieme di linee guida che dovevano incanalare queste idee di base, costruite in modo di essere uno strumento di visione del software. Il movimento Open Source pubblica la sua definizione quasi come fosse un manifesto, spiegando il perch\`e di ogni punto. Imponendo la libera redistribuzione, si elimina la tentazione di rinunciare a importanti guadagni a lungo termine in cambio di un guadagno materiale a breve termine, ottenuto con il controllo delle vendite
\item Vincoli su altro software: la licenza non deve porre restrizioni su altro software distribuito insieme al software licenziato
\item Neutralit\`a rispetto alle tecnologie: la licenza non deve contenere clausole che dipendano o si basino su particolari tecnologie o tipi di interfacce

\end{itemize}

\paragraph*{Perl artistic license} La Perl artistic license sancisce che non \`e redistribuibile il software per scopi commerciali, mentre pu\`o essere incluso in una distribuzione pi\`u grande per essere venduto commercialmente. I software open source bloccano la redistribuzione di software commercialmente, ma permettono la redistribuzione dello stesso software ``wrapperizzato'' all'interno di qualcosa di pi\`u grande.

