\section{Lezione del 12-01-16}
\subsection{SVC - Conf}
Tre file per le varie configurazioni:
\paragraph*{svnserve.conf}
In questo file vengono specificati gli utenti e i loro privilegi all'interno del repository. Comandi:
\begin{itemize}
\item anon-access -> livelli di accesso per utenti anonimi;
\item auth-access -> livelli di accesso per utenti autenticati;
\item password-db -> locazone file passwd;
\item realm -> se due repository hanno lo stesso realm allora entrambi condividono le password;
\item authz -> locazione file authz;
\end{itemize}
Opzioni di svn:
\begin{verbatim}
svnserve -d -r PATH/RADICE/DOVE/SONO/ALLOCATI/I/REPOSITORY
\end{verbatim}
In questo modo non \`e necessario inserire la path assoluta del server per ottenere il repository. Inoltre si evita un grosso problema di sicurezza. 


\paragraph*{passwd}
In questo file vengono specificate le password ed il relativo utente. E' da inserire nel formato: \newline
\begin{verbatim}
UTENTE = PASSWORD
\end{verbatim}

\paragraph*{authz}
In questo file sono specificati in maggior dettaglio i livelli di accesso dei vari utenti ed i vari repository. \newline
E' da inserire nel formato: \newline
\begin{verbatim}
[path/del/repository]
UTENTE = LIVELLI DI ACCESSO [rw|vuoto]
@GRUPPO = LIVELLI DI ACCESSO [rw|vuoto]

Note: quanto il livello di accesso \`e vuoto l'utente non ha alcun livello di accesso.
\end{verbatim}

\subsection{SVN - Revisioni miste}
SVN ti la possibilit\`a di aggiornare solo parte del repository. Per farlo basta andare nella directory in cui si \`e interessati e aggiornare oppure creare un nuovo file e committare. \newline
Per verificare che la revisione \`e mista basta guardare la revisione di partenza di tutti file, se ci sono alcune versioni diverse allora si \`e in una revisione mista. \newline
Le revisioni miste portano alcuni problemi:
\begin{itemize}
\item svn log mostra solo le revisioni di partenza e non le modifiche aggiunte. \`E sempre necessario aggiornare per vedere tutte le modifiche;
\item non \`e possibile applicare operazioni di merge;
\end{itemize}

\subsection{SNV - Comandi utili}
HEAD,BASE,PREV \newline
Specifica intervallo FROM:TO \newline

\paragraph{Comando: svn info REPO}
Offre una serie d'informazioni riguardanti il repository selezionato.
\paragraph{Comando: svn st -u}
Mostra tutte le modifiche applicate rispetto al server.
\paragraph{Comando: svn st -v}
Mostra le informazioni sull'ultima revisione.
\paragraph{Comando: svn up -r REVISIONE}
Specifica a quale revisione, determinato da REVISIONE, aggiornare il proprio repository.
\paragraph{Comando: svn log HEAD [-l NUMERO]}
Indica di visualizzare l'ultima revisione. In questo modo si ha in cima tutti i commit fatti nelle ultime revisioni. \`E specificando -l NUMERO si stampano i primi NUMERO revisioni svolte.
\paragraph{Comando: svn diff -r PREV:BASE FILE}
Indica quali sono le difference tra il FILE nella versione corrente e quella precedente.
\paragraph{Comando: svn log -r ''{yyyy-mm-dd hh:mm}'':HEAD}
Mostra i commit svolti dalla data segnato nel formato yyyy-mm-dd e quello corrente. \`E possibile scegliere la data, l'ora o tutte e due. Nel caso si sposta la data nella parte del TO, se non si specifica l'ora viene settata automaticamente alle 23:59, perci\`o bisogna stare attendi a specificare il giorno.

