\section{Lezione del 15-12-15}
\graphicspath{ {res/data/15-12-15/} }


\paragraph*{Dublin Core}Questo viene importato con il prefisso ``dc'', e presenta le principali voci:
\begin{itemize}
\item[Contenuto]
  \begin{itemize}
    
  \item Title: Titolo dell'opera
  \item Subject: Topic dell'opera
  \item Description: Abstract
  \item Source: Opera originaria
    
  \end{itemize}

\item[Propriet\`a Intellettuale]
  \begin{itemize}

  \item Creator: autore
  \item Contributor: ulteriore contributore
  \item Publisher: per chi \`e responsabile per la pubblicazione della risorsa $\to$ per una collana pu\`o essere il rispettivo pubblicatore
    \item Rights: un link alla licenza
    
  \end{itemize}
  
\end{itemize}

I predicati double core possono essere usati per specificare la licenza, ma pu\`o essere possibile usare un altra vocabolario pi\`u specifico.

\paragraph*{CCrel} \`E un vocabolario creato per parlare delle licenze creato dai membri della creative commons. Permette di specificare le propriet\`a del lavoro e della licenza.
Voci:
\begin{itemize}

\item cc:attributionName
\item cc:attributionUrl $\to$ permette di citare direttamente la licenza di tutto il documento, semplificando il suo ritrovamento
  
\end{itemize}

Le altre voci vengono riprese da altri vocabolari, come ad esempio Double Core o xhtml.
