\section{Lezione del 24-11-15}

Nel 1998 viene approvato il Sonny Bono Copyright Extension Act\footnote{Detto anche Mickey Mouse Protection Act}, che come sostenitore aveva la Disney, dove l'estensione del copyright veniva estesa di ulteriori 20 anni. Questa estensione colp\`i gli interessi di tantissime opere, come il caso di Eldricht Press di Eric Eldred che ospitava un sito web di libri con il copyiright scaduto.

%\subsection{Lessing}

\paragraph*{Lessig}Politico in Erba derivante da una famiglia repubblicana, dopo il college a Cambridge divent\`o liberale e si interess\`o di Giurisprudenza, incominciandosi a interessare dell'effetto Eldred, dove una stessa legge era stata interpretata in maniera diversa in base a pressioni sociali. Si interess\`o dell'impatto dell'architettura di internet sulla legge, dove siti grandi visitati da molti utenti finivano per ottenere un controllo sugli utenti.
Negli ultimi anni si assiste a una fusione tra il mondo reale e il mondo cybernetico.
Con il patto Lessig-Eldred si ha la denuncia dell'incostituzionalit\`a del SBCEA. Si ebbe l'apertura del sito \textit{openlaw}, dove persone avevano la possibilit\`a di confrontarsi e trasmettere dei valori sul modo di vedere su ci\`o che doveva essere il modo di vedere la diffusione dell'informazione. La causa fu persa 7-2 alla corte suprema nel 2003.

\subsection{Creative Commons}

Creata dall'idea di Eldred: creare una copyright conservancy. Dopo la fine della causa con Microsoft Eldred comincia a farlo fin dall'inizio dei principi, in quanto bisogna vedere l'apertura del copyright per tutte le opere, in quanto le case produttrici dopo pochi anni smettevano di vendere le opere in quanto non c'era pi\`u un vero guadagno. Era inoltre necessario abbandonare i principi sul fair use\footnote{Utilizzo dell'opere senza vincoli. Sono presenti dei paletti (\`e possibile per una piccola citazione che non sia contro l'autore stesso etc..).}. Era necessario creare un movimento tecnico-legale, senza esserne a conoscenza si veniva a creare l'inizio della \textit{Creative Commons}. Gi\`a nel 1999 era nato il \textit{Copyright Commons}, una newsletter mensile sul caso Eldred vs Reno. Da qui si ha l'idea per la creazione della compagna Counter Copyright (CC), per dare la possibilit\`a di creare opere e distribuirle anche sotto Public Domain, anche se non era possibile in quanto ogni opera \`e sempre sotto un dominio.

Nel 1999 si ha anche la nascita di Napster, un sito in cui scaricare musica illegale che fece moltissimo successo. Lessig capisce che c'\`e moltissima gente che ha bisogno di accedere ai contenuti liberamente. Da qui nasce nel 2000 dall'idea di Habelson: una fondazione che accetti donazioni di opere, anche se scartata in quanto troppa problematica. Da ci\`o si ha nel 2001 la fondazione della Creative Commons\footnote{Il nome deriva dalla tragedia di Garret Hardin ``tragedia dei commons''.}, dove era possibile adottare una licenza comune per tutte le opere che volevano essere diffuse liberamente. Gli obiettivi del progetto sono:
\begin{itemize}

\item Promuovere la diffusione di un modello ``some rights reserved''
\item Tutelare il marchio ``Creative Commons''
\item Creazione di  appropriati strumenti legali e tecnologici

\end{itemize}
Le creative commons si pongono a met\`a tra il pubblico dominio e tra il copyright assoluto.

Creative Commons non \`e un ente pubblico, n\`e un organismo per la raccolta dei diritti d'autore e non offre servizi di consulenza legale.

\paragraph*{Licenze CC}Le licenze CC sono pensate per opere creative non software. Sono disponibili per 52 sistemi giuridici, e hanno come caratteristiche comuni sulla preservazione delle note di licenza, sul permesso di copia e distribuzione dell'opera e sull'esibizione dell'opera e l'uso di restrizioni tecnologiche vietato.
\`E possibile inserire un digital code che segna la licenza che permette di facilitare la ricerca nei motori come Google, rendendo la loro visibili\`a pi\`u rapida. Ogni licenza \`e identificata dal suo commons deed:
\begin{itemize}
  
\item Attribuzione
\item No opere derivate
\item Non commerciale
\item Condividi allo stesso modo

\end{itemize}

Tutte le combinazioni formano le licenze Creative Commons. \`E da notare che ``Attribuzione'' \`e presente in tutte le licenze e che non esiste ``No-Derivate'' e ``Share-alike'' nella stessa licenza in quanto non avrebbe molto senso.
Le licenze possibili sono:
\begin{itemize}

\item Attribution
\item Attribution-SA $\to$ per certi versi la pi\`u simile alla GNU GPL.
\item Attribution-ND $\to$ non \`e permesso creare opere derivate, ma \`e sempre possibile la distribuzione (sempre sotto la stessa licenza)
\item Attribution-NC $\to$ non sar\`a mai possibile la ridistribuzione a scopo commerciale. \`E anche possibile cambiare licenza basta che rimanga la clausola NC.
\item Attribution-NC-SA $\to$ \`e necessario riconoscere l'autore, non \`e possibile utilizzarla per scopi commerciali e \`e necessario condividerla allo stesso modo
\item Attribution-NC-ND $\to$ non \`e possibile usarla per scopi commerciali senza alcuna modifica. Questo si usa per opere che vogliono avere un'ampia diffusione e veniva utilizzato recentemente per i temi Wordpress.

\end{itemize}

\subsubsection{Obblighi delle licenze CC}

Una raccolta di opere \`e da considerarsi come un lavoro collettivo. La distribuzione della licenza \`e possibile attraverso un URL. Per eseguire una citazione su un'opera protetta da CC \`e necessario specificare:
\begin{itemize}

\item Autore originario
\item Titolo dell'opera originaria
\item Se possibile URL associato all'opera
\item Uso dell'opera
\item Rimozione citazione su richiesta del licenziate\footnote{\`E necessario specificare: il licenziante \`e colui che licenzia l'opera, l'autore \`e colui che la scrive. }

\end{itemize}

Con l'attribution SA si aveva la garanzia del proprietario dei diritti di avere i diritti necessari e che l'opera rispetta i diritti d'autore, di marchio di fabbrica e altro. Questa clausola \`e stata rimossa dalla licenza 2.0 in poi.

\subsubsection{Strumenti riconoscimento licenze CC}
\begin{itemize}

\item CC plus: protocollo per esprimere licenze alternative, viene usato anche dai motori di ricerca per trovare documenti con specifiche licenze.
\item CC0: alternativa al public domain, che permette di scegliere in modo pi\`u vicino possibile in base alla giurisdizione del paese di rilasciare la vostra opera.
\item Founders' copyright: possibilit\`a di mantenere il copyright per 14 anni rinnovabili una volta. Questa funzionalit\`a \`e stata rimossa.

\end{itemize}

\subsection{RDF e RDFa}

Materiale di riferimento: \textit{Pratical RDF di Shelley Powers, RDF primer, RDFA primer, RDFA syntax al www.w3.org}

\paragraph*{Definizioni}RDF \`e un linguaggio per rappresentare informazioni semantiche sul web.
RDFa \`e un modo per esprimere le informazioni semantiche in RDF all'interno di una pagina web. \`E un'estensione xhtml.
Ccrel: vocabolario per RDFa per esprimere asserzioni sulla licenza di contenuti web.

\subsubsection{RDF}
L'obiettivo dell'RDF \`e di costruire un linguaggio per esprimere contenuti semantici in maniera semplice, estendibile attraverso moduli e interpretabile da un calcolatore.
La soluzione adottata \`e attraverso grafi: questo perch\`e pi\`u facile da elaborare da parte di un calcolatore, rimanendo comunque estendibili. Un grafo \`e formato da degli statement, che sono creati da soggetto, predicato, oggetto. \`E importante disambiguare in maniera semplice i vari significati delle parole, e questo viene attraverso degli URI o un nodo anonimo o letterale. La visualizzazione di queste opzioni avviene rispettivamente con un ovale contenente l'URI, cerchi vuoti, rettangoli contenenti la rispettiva stringa.

\paragraph*{Struttura dell'URI}Le URI non rappresentano sempre dei siti web, e in genere sono composti da:
\begin{itemize}

\item Schema (http, ftp, uuid)
\item Authority (opzionale)
\item Cammino assoluto o relativo
\item Query e identificatore di frammento

\end{itemize}

L'URI esprime solo nomi e non localizzazioni.

\paragraph*{Sintassi delle URI}La sintassi generica di un URI \`e la seguente:
\begin{verbatim}

schema:localpart [?query][#frammento]

\end{verbatim}

Dove lo schema pu\`o essere http, ftp, tel...
Per indentificare l'authority dev'essere presente la doppia barra ``//''.
Ad esempio:
\begin{verbatim}

http://www.debian.org/index.html

\end{verbatim}

Se si scrive http:/index.html allora si indica la pagina index dentro l'authority www.debian.org.
Per esempio il numero di telefono non hanno un authority, come anche per la voce mailto.

\paragraph*{Utilit\`a degli URI}Gli URI permettono di creare degli identificativi unici all'interno del web, creando uno schema forse difficile da leggere ma preciso.

\paragraph*{XML Qualified Name}Un qualified name \`e identificato da un URI. \`E da notare che gli escape forniscono qualified name diversi. I Qname in RDF dipendono dalla codifica adottata. Un modo abbastanza comune \`e specificare RDF con xmlns.
Esistono alcuni namespace standard:
\begin{itemize}

\item rdf
\item rdfs
\item dc
\item ex
\item foaf

\end{itemize}

Visitando prefix.cc \`e possibile accedere alle risorse che puntano alle varie risorse (rdf).

\paragraph*{Nodi anonimi}
Con i nodi anonimi \`e possibile creare nodi senza un URI associato, ed \`e inoltre possibile inserire letterali. I letterali di un grafo RDF possono avere un \textbf{tipo associato}, esso non \`e un tipo obbligatorio. I tipi in RDF sono definiti in XML Schema, ma non tutti i tipi di XML Schema sono associabili a un letterale, ma solo a un tipo primitivo. \newline

L'insieme di RDF/XML risulta essere molto verboso e scarsamente collegato al web.

\paragraph*{Notazione N-Triples}

Ogni riga pu\`o essere un commento (che inizia con ``\#'') o una tripla del tipo: subject predicate object.
I nodi anonimi si indicano con '\_:name'.
Questo approccio presenta gli stessi problemi di replicazione di RDF/XML, ma presenta una sintassi molto semplice.
