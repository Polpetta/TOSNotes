\section{Lezione del 27-10-15}
\graphicspath{ {res/data/27-10-15/} }

\paragraph*{Confronto con software libero} Le libert\`a tra il software libero e il software Open Source \`e simile, anche se hanno obiettivi diversi.

\subsection{Licenze software}

Materiale di riferimento: \textit{Understanding Open Source and Free Software Licensig, di Andrew Laurent}, \textit{Open Source Licensing: Sofrware Freedom and Intellectual property Law}

Capisaldi del diritto d'autore\footnote{Si riferiscono con i dati in america}:
\begin{itemize}
\item 70 anni dopo la morte per le persone
\item 95 anni dopo le pubblicazioni o 120 dopo la creazione per le corporazioni
\item Protezione per le sole opere espresse in forma percepibile
\item Nessuna registrazione richiesta
\item Work for hire: ogni volta che c'\`e un contratto di dipendenza i lavoratori sono visti come ``strumenti'', e il diritto \`e dell'azienda stessa. Questo non si applica per esempio per collaborazioni, se non diversamente specificato dal contratto.
\end{itemize}

\paragraph*{Garanzie} esistono diversi tipi di garanzie, che possono essere:
\begin{itemize}
\item Esplicite: in cui ci si assume la responsabilit\`a di una garanzia data
\item Implicite: garanzie che possono essere imposte per esempio per legge o dagli stati, come quelle di commercialit\`a, idoneit\`a e di non violazione di diritti terzi (garantisce che il software che si usa non violi in copyright software di altri)
\end{itemize}

In caso le garanzie non vengano rispettate possono esserci dei \textbf{danni}, che possono essere diretti o consequenziali.

\newpage

\subsubsection{MIT License}

\begin{verbatim}

The MIT License (MIT)

Copyright (c) [year] [fullname]

Permission is hereby granted, free of charge, to any person obtaining a copy
of this software and associated documentation files (the "Software"), to deal
in the Software without restriction, including without limitation the rights
to use, copy, modify, merge, publish, distribute, sublicense, and/or sell
copies of the Software, and to permit persons to whom the Software is
furnished to do so, subject to the following conditions:

The above copyright notice and this permission notice shall be included in all
copies or substantial portions of the Software.

THE SOFTWARE IS PROVIDED "AS IS", WITHOUT WARRANTY OF ANY KIND, EXPRESS OR
IMPLIED, INCLUDING BUT NOT LIMITED TO THE WARRANTIES OF MERCHANTABILITY,
FITNESS FOR A PARTICULAR PURPOSE AND NONINFRINGEMENT. IN NO EVENT SHALL THE
AUTHORS OR COPYRIGHT HOLDERS BE LIABLE FOR ANY CLAIM, DAMAGES OR OTHER
LIABILITY, WHETHER IN AN ACTION OF CONTRACT, TORT OR OTHERWISE, ARISING FROM,
OUT OF OR IN CONNECTION WITH THE SOFTWARE OR THE USE OR OTHER DEALINGS IN THE
SOFTWARE.

\end{verbatim}

Chiamata anche X License (creata per X.org) creata al MIT, \`e essenzialmente la licenza pi\`u libera che ci sia.
Punti importanti:
\begin{itemize}
\item Libera copia, modifica e ridistribuzione del software
\item Necessit\`a di mantenere la licenza originale
\item Possibilit\`a di sviluppi proprietari
\item Clausole di salvaguardia
\end{itemize}

\begin{figure}[h]
  \centering
  \includegraphics[scale=0.6]{27-10-15-01}
  \caption{In questo caso, se ho un pacchetto licenziato sotto MIT e aggiungo delle mie modifice (evidenziate in azzurro) i file originali rimangono sotto MIT, mentre le mie modifiche possono avere la licenza che vogliono. In ogni caso i file originali sotto MIT rimarranno tali.}
\end{figure}

\newpage

\subsubsection{BSD License}

\begin{verbatim}

Copyright (c) 1993 The Regents of the University of California. All
rights reserved.

This software was developed by the Computer Systems Engineering group
at Lawrence Berkeley Laboratory under DARPA contract BG 91-66 and
contributed to Berkeley.

All advertising materials mentioning features or use of this software
must display the following acknowledgement: This product includes
software developed by the University of California, Lawrence Berkeley
Laboratory.

Redistribution and use in source and binary forms, with or without
modification, are permitted provided that the following conditions are
met:

    1. Redistributions of source code must retain the above copyright
    notice, this list of conditions and the following disclaimer.

    2. Redistributions in binary form must reproduce the above copyright
    notice, this list of conditions and the following disclaimer in
    the documentation and/or other materials provided with the
    distribution.

    3. All advertising materials mentioning features or use of this
    software must display the following acknowledgement: This product
    includes software developed by the University of California,
    Berkeley and its contributors.

    4. Neither the name of the University nor the names of its
    contributors may be used to endorse or promote products derived
    from this software without specific prior written permission.

THIS SOFTWARE IS PROVIDED BY THE REGENTS AND CONTRIBUTORS ``AS IS''
AND ANY EXPRESS OR IMPLIED WARRANTIES, INCLUDING, BUT NOT LIMITED TO,
THE IMPLIED WARRANTIES OF MERCHANTABILITY AND FITNESS FOR A PARTICULAR
PURPOSE ARE DISCLAIMED. IN NO EVENT SHALL THE REGENTS OR CONTRIBUTORS
BE LIABLE FOR ANY DIRECT, INDIRECT, INCIDENTAL, SPECIAL, EXEMPLARY, OR
CONSEQUENTIAL DAMAGES (INCLUDING, BUT NOT LIMITED TO, PROCUREMENT OF
SUBSTITUTE GOODS OR SERVICES; LOSS OF USE, DATA, OR PROFITS; OR
BUSINESS INTERRUPTION) HOWEVER CAUSED AND ON ANY THEORY OF LIABILITY,
WHETHER IN CONTRACT, STRICT LIABILITY, OR TORT (INCLUDING NEGLIGENCE
OR OTHERWISE) ARISING IN ANY WAY OUT OF THE USE OF THIS SOFTWARE, EVEN
IF ADVISED OF THE POSSIBILITY OF SUCH DAMAGE.

\end{verbatim}

Licenze libere pi\`u diffuse, che permettono di diffondere un software e di licenziarlo sotto un'altra licenza proprietaria. Esistono 3 varianti principali:
\begin{itemize}
\item BSD a 4 clausole, licenza iniziale di BSD Unix. Era come BSD a 3 clausole, con l'aggiunta di una clausola che diceva che tutti i materiali (pubblicitari o simili) che menzionano l'utilizzo di questo software devono rendere visibile il seguente avviso: ``Questo prodotto include software sviluppato da -nome dell'organizzazione-''
\item BSD a 3 clausole. \`E identica alla BSD a 2 clausole, con l'aggiunta che dice che n\`e il nome del detentore del copyright n\`e il nome dei suoi contributori possono essere usati per promuovere i prodotti derivati da questo software senza un permesso scritto dal posessore iniziale
\item BSD a 2 clausole. Ammette la redistribuzione con o senza modifica rispettando due clausole: la ridistribuzione del codice sorgente deve mantere le note di copyright e la distribuzione in forma binaria deve riprodurre la nota di copyright, con la lista di condizioni e il disclamer nella documentazione o altri materiali dati insieme al software
\end{itemize}

La licenza BSD a 2 e a 3 clausole sono state approvate dall'OSI, mentre la 4 clausole non \`e stata approvata come Open Source, in quanto non compatibile dalla GPL.

\newpage

\subsubsection{Apache License}

Simile alla licenza BSD, con variazioni importanti. Ha avuto tre revisioni:
\begin{itemize}

\item Versione 1.0: BSD-4clausole con l'aggiunta di una clausola di rinomina
\item Versione 1.1: BSD-3clausole con l'aggiunta di una clausola di rinomina e di una clausola pubblicitaria sulla documentazione
\item Versione 2.0: Si ha un grosso cambiamento della licenza, affrontando una serie di problemi legati ai brevetti, ovvero facendo diventare fondamentale avere delle garanzie: quando viene donata una parte di codice devono essere donati anche i brevetti a esso associati. Ci\`o fa diventare di importanza fondamentale la distinzione tra gli sviluppatori originari e chi invece contribuisce.
\end{itemize}

Qui di sotto viene riportata la versione 2.0 della licenza.

\begin{verbatim}
                                 Apache License
                           Version 2.0, January 2004
                        http://www.apache.org/licenses/

   TERMS AND CONDITIONS FOR USE, REPRODUCTION, AND DISTRIBUTION

   1. Definitions.

      "License" shall mean the terms and conditions for use, reproduction,
      and distribution as defined by Sections 1 through 9 of this document.

      "Licensor" shall mean the copyright owner or entity authorized by
      the copyright owner that is granting the License.

      "Legal Entity" shall mean the union of the acting entity and all
      other entities that control, are controlled by, or are under common
      control with that entity. For the purposes of this definition,
      "control" means (i) the power, direct or indirect, to cause the
      direction or management of such entity, whether by contract or
      otherwise, or (ii) ownership of fifty percent (50%) or more of the
      outstanding shares, or (iii) beneficial ownership of such entity.

      "You" (or "Your") shall mean an individual or Legal Entity
      exercising permissions granted by this License.

      "Source" form shall mean the preferred form for making modifications,
      including but not limited to software source code, documentation
      source, and configuration files.

      "Object" form shall mean any form resulting from mechanical
      transformation or translation of a Source form, including but
      not limited to compiled object code, generated documentation,
      and conversions to other media types.

      "Work" shall mean the work of authorship, whether in Source or
      Object form, made available under the License, as indicated by a
      copyright notice that is included in or attached to the work
      (an example is provided in the Appendix below).

      "Derivative Works" shall mean any work, whether in Source or Object
      form, that is based on (or derived from) the Work and for which the
      editorial revisions, annotations, elaborations, or other modifications
      represent, as a whole, an original work of authorship. For the purposes
      of this License, Derivative Works shall not include works that remain
      separable from, or merely link (or bind by name) to the interfaces of,
      the Work and Derivative Works thereof.

      "Contribution" shall mean any work of authorship, including
      the original version of the Work and any modifications or additions
      to that Work or Derivative Works thereof, that is intentionally
      submitted to Licensor for inclusion in the Work by the copyright owner
      or by an individual or Legal Entity authorized to submit on behalf of
      the copyright owner. For the purposes of this definition, "submitted"
      means any form of electronic, verbal, or written communication sent
      to the Licensor or its representatives, including but not limited to
      communication on electronic mailing lists, source code control systems,
      and issue tracking systems that are managed by, or on behalf of, the
      Licensor for the purpose of discussing and improving the Work, but
      excluding communication that is conspicuously marked or otherwise
      designated in writing by the copyright owner as "Not a Contribution."

      "Contributor" shall mean Licensor and any individual or Legal Entity
      on behalf of whom a Contribution has been received by Licensor and
      subsequently incorporated within the Work.

   2. Grant of Copyright License. Subject to the terms and conditions of
      this License, each Contributor hereby grants to You a perpetual,
      worldwide, non-exclusive, no-charge, royalty-free, irrevocable
      copyright license to reproduce, prepare Derivative Works of,
      publicly display, publicly perform, sublicense, and distribute the
      Work and such Derivative Works in Source or Object form.

   3. Grant of Patent License. Subject to the terms and conditions of
      this License, each Contributor hereby grants to You a perpetual,
      worldwide, non-exclusive, no-charge, royalty-free, irrevocable
      (except as stated in this section) patent license to make, have made,
      use, offer to sell, sell, import, and otherwise transfer the Work,
      where such license applies only to those patent claims licensable
      by such Contributor that are necessarily infringed by their
      Contribution(s) alone or by combination of their Contribution(s)
      with the Work to which such Contribution(s) was submitted. If You
      institute patent litigation against any entity (including a
      cross-claim or counterclaim in a lawsuit) alleging that the Work
      or a Contribution incorporated within the Work constitutes direct
      or contributory patent infringement, then any patent licenses
      granted to You under this License for that Work shall terminate
      as of the date such litigation is filed.

   4. Redistribution. You may reproduce and distribute copies of the
      Work or Derivative Works thereof in any medium, with or without
      modifications, and in Source or Object form, provided that You
      meet the following conditions:

      (a) You must give any other recipients of the Work or
          Derivative Works a copy of this License; and

      (b) You must cause any modified files to carry prominent notices
          stating that You changed the files; and

      (c) You must retain, in the Source form of any Derivative Works
          that You distribute, all copyright, patent, trademark, and
          attribution notices from the Source form of the Work,
          excluding those notices that do not pertain to any part of
          the Derivative Works; and

      (d) If the Work includes a "NOTICE" text file as part of its
          distribution, then any Derivative Works that You distribute must
          include a readable copy of the attribution notices contained
          within such NOTICE file, excluding those notices that do not
          pertain to any part of the Derivative Works, in at least one
          of the following places: within a NOTICE text file distributed
          as part of the Derivative Works; within the Source form or
          documentation, if provided along with the Derivative Works; or,
          within a display generated by the Derivative Works, if and
          wherever such third-party notices normally appear. The contents
          of the NOTICE file are for informational purposes only and
          do not modify the License. You may add Your own attribution
          notices within Derivative Works that You distribute, alongside
          or as an addendum to the NOTICE text from the Work, provided
          that such additional attribution notices cannot be construed
          as modifying the License.

      You may add Your own copyright statement to Your modifications and
      may provide additional or different license terms and conditions
      for use, reproduction, or distribution of Your modifications, or
      for any such Derivative Works as a whole, provided Your use,
      reproduction, and distribution of the Work otherwise complies with
      the conditions stated in this License.

   5. Submission of Contributions. Unless You explicitly state otherwise,
      any Contribution intentionally submitted for inclusion in the Work
      by You to the Licensor shall be under the terms and conditions of
      this License, without any additional terms or conditions.
      Notwithstanding the above, nothing herein shall supersede or modify
      the terms of any separate license agreement you may have executed
      with Licensor regarding such Contributions.

   6. Trademarks. This License does not grant permission to use the trade
      names, trademarks, service marks, or product names of the Licensor,
      except as required for reasonable and customary use in describing the
      origin of the Work and reproducing the content of the NOTICE file.

   7. Disclaimer of Warranty. Unless required by applicable law or
      agreed to in writing, Licensor provides the Work (and each
      Contributor provides its Contributions) on an "AS IS" BASIS,
      WITHOUT WARRANTIES OR CONDITIONS OF ANY KIND, either express or
      implied, including, without limitation, any warranties or conditions
      of TITLE, NON-INFRINGEMENT, MERCHANTABILITY, or FITNESS FOR A
      PARTICULAR PURPOSE. You are solely responsible for determining the
      appropriateness of using or redistributing the Work and assume any
      risks associated with Your exercise of permissions under this License.

   8. Limitation of Liability. In no event and under no legal theory,
      whether in tort (including negligence), contract, or otherwise,
      unless required by applicable law (such as deliberate and grossly
      negligent acts) or agreed to in writing, shall any Contributor be
      liable to You for damages, including any direct, indirect, special,
      incidental, or consequential damages of any character arising as a
      result of this License or out of the use or inability to use the
      Work (including but not limited to damages for loss of goodwill,
      work stoppage, computer failure or malfunction, or any and all
      other commercial damages or losses), even if such Contributor
      has been advised of the possibility of such damages.

   9. Accepting Warranty or Additional Liability. While redistributing
      the Work or Derivative Works thereof, You may choose to offer,
      and charge a fee for, acceptance of support, warranty, indemnity,
      or other liability obligations and/or rights consistent with this
      License. However, in accepting such obligations, You may act only
      on Your own behalf and on Your sole responsibility, not on behalf
      of any other Contributor, and only if You agree to indemnify,
      defend, and hold each Contributor harmless for any liability
      incurred by, or claims asserted against, such Contributor by reason
      of your accepting any such warranty or additional liability.

   END OF TERMS AND CONDITIONS
\end{verbatim}

\newpage

\subsubsection{Academic Free License}

\begin{verbatim}

The Academic Free License
v. 2.1

This Academic Free License (the "License") applies to any original work of
authorship (the "Original Work") whose owner (the "Licensor") has placed the
following notice immediately following the copyright notice for the Original
Work:

Licensed under the Academic Free License version 2.1

1) Grant of Copyright License.  Licensor hereby grants You a world-wide,
   royalty-free, non-exclusive, perpetual, sublicenseable license to do the
   following:

   a) to reproduce the Original Work in copies;

   b) to prepare derivative works ("Derivative Works") based upon the Original
      Work;

   c) to distribute copies of the Original Work and Derivative Works to the 
      public;

   d) to perform the Original Work publicly; and

   e) to display the Original Work publicly.

2) Grant of Patent License.  Licensor hereby grants You a world-wide,
   royalty-free, non-exclusive, perpetual, sublicenseable license, under patent
   claims owned or controlled by the Licensor that are embodied in the Original
   Work as furnished by the Licensor, to make, use, sell and offer for sale the
   Original Work and Derivative Works.

3) Grant of Source Code License.  The term "Source Code" means the preferred
   form of the Original Work for making modifications to it and all available
   documentation describing how to modify the Original Work.  Licensor hereby
   agrees to provide a machine-readable copy of the Source Code of the Original
   Work along with each copy of the Original Work that Licensor distributes.
   Licensor reserves the right to satisfy this obligation by placing a
   machine-readable copy of the Source Code in an information repository 
   reasonably calculated to permit inexpensive and convenient access by You for
   as long as Licensor continues to distribute the Original Work, and by
   publishing the address of that information repository in a notice immediately
   following the copyright notice that applies to the Original Work.

4) Exclusions From License Grant.  Neither the names of Licensor, nor the names
   of any contributors to the Original Work, nor any of their trademarks or 
   service marks, may be used to endorse or promote products derived from this
   Original Work without express prior written permission of the Licensor. 
   Nothing in this License shall be deemed to grant any rights to trademarks,
   copyrights, patents, trade secrets or any other intellectual property of
   Licensor except as expressly stated herein.  No patent license is granted to
   make, use, sell or offer to sell embodiments of any patent claims other than
   the licensed claims defined in Section 2. No right is granted to the
   trademarks of Licensor even if such marks are included in the Original Work. 
   Nothing in this License shall be interpreted to prohibit Licensor from
   licensing under different terms from this License any Original Work that
   Licensor otherwise would have a right to license.

5) This section intentionally omitted.

6) Attribution Rights.  You must retain, in the Source Code of any Derivative
   Works that You create, all copyright, patent or trademark notices from the
   Source Code of the Original Work, as well as any notices of licensing and any
   descriptive text identified therein as an "Attribution Notice."  You must 
   cause the Source Code for any Derivative Works that You create to carry a
   prominent Attribution Notice reasonably calculated to inform recipients that
   You have modified the Original Work.

7) Warranty of Provenance and Disclaimer of Warranty.  Licensor warrants that
   the copyright in and to the Original Work and the patent rights granted 
   herein by Licensor are owned by the Licensor or are sublicensed to You under
   the terms of this License with the permission of the contributor(s) of those
   copyrights and patent rights.  Except as expressly stated in the immediately
   proceeding sentence, the Original Work is provided under this License on an
   "AS IS" BASIS and WITHOUT WARRANTY, either express or implied, including,
   without limitation, the warranties of NON-INFRINGEMENT, MERCHANTABILITY or
   FITNESS FOR A PARTICULAR PURPOSE.  THE ENTIRE RISK AS TO THE QUALITY OF THE
   ORIGINAL WORK IS WITH YOU. This DISCLAIMER OF WARRANTY constitutes an
   essential part of this License.  No license to Original Work is granted
   hereunder except under this disclaimer.

8) Limitation of Liability.  Under no circumstances and under no legal theory,
   whether in tort (including negligence), contract, or otherwise, shall the
   Licensor be liable to any person for any direct, indirect, special, 
   incidental, or consequential damages of any character arising as a result of
   this License or the use of the Original Work including, without limitation,
   damages for loss of goodwill, work stoppage, computer failure or malfunction,
   or any and all other commercial damages or losses.  This limitation of
   liability shall not apply to liability for death or personal injury resulting
   from Licensor's negligence to the extent applicable law prohibits such
   limitation.  Some jurisdictions do not allow the exclusion or limitation of
   incidental or consequential damages, so this exclusion and limitation may not
   apply to You.

9) Acceptance and Termination.  If You distribute copies of the Original Work or
   a Derivative Work, You must make a reasonable effort under the circumstances
   to obtain the express assent of recipients to the terms of this License. 
   Nothing else but this License (or another written agreement between Licensor
   and You) grants You permission to create Derivative Works based upon the
   Original Work or to exercise any of the rights granted in Section 1 herein,
   and any attempt to do so except under the terms of this License (or another
   written agreement between Licensor and You) is expressly prohibited by U.S. 
   copyright law, the equivalent laws of other countries, and by international
   treaty.  Therefore, by exercising any of the rights granted to You in Section
   1 herein, You indicate Your acceptance of this License and all of its terms
   and conditions.

10) Termination for Patent Action.  This License shall terminate automatically
    and You may no longer exercise any of the rights granted to You by this 
    License as of the date You commence an action, including a cross-claim or
    counterclaim, against Licensor or any licensee alleging that the Original
    Work infringes a patent.  This termination provision shall not apply for an
    action alleging patent infringement by combinations of the Original Work
    with other software or hardware.

11) Jurisdiction, Venue and Governing Law.  Any action or suit relating to this
    License may be brought only in the courts of a jurisdiction wherein the
    Licensor resides or in which Licensor conducts its primary business, and
    under the laws of that jurisdiction excluding its conflict-of-law
    provisions.  The application of the United Nations Convention on Contracts
    for the International Sale of Goods is expressly excluded.  Any use of the
    Original Work outside the scope of this License or after its termination
    shall be subject to the requirements and penalties of the U.S.  Copyright
    Act, 17 U.S.C. par. 101 et seq., the equivalent laws of other countries, and
    international treaty.  This section shall survive the termination of this
    License.

12) Attorneys Fees.  In any action to enforce the terms of this License or
    seeking damages relating thereto, the prevailing party shall be entitled to
    recover its costs and expenses, including, without limitation, reasonable
    attorneys' fees and costs incurred in connection with such action, including 
    any appeal of such action.  This section shall survive the termination of 
    this License.

13) Miscellaneous.  This License represents the complete agreement concerning
    the subject matter hereof.  If any provision of this License is held to be
    unenforceable, such provision shall be reformed only to the extent necessary 
    to make it enforceable.

14) Definition of "You" in This License.  "You" throughout this License, whether
    in upper or lower case, means an individual or a legal entity exercising 
    rights under, and complying with all of the terms of, this License.  For 
    legal entities, "You" includes any entity that controls, is controlled by,
    or is under common control with you.  For purposes of this definition,
    "control" means (i) the power, direct or indirect, to cause the direction or
    management of such entity, whether by contract or otherwise, or (ii)
    ownership of fifty percent (50%) or more of the outstanding shares, or (iii)
    beneficial ownership of such entity.

15) Right to Use.  You may use the Original Work in all ways not otherwise
    restricted or conditioned by this License or by law, and Licensor promises
    not to interfere with or be responsible for such uses by You. 

This license is Copyright (C) 2003-2004 Lawrence E. Rosen.  All rights reserved.
Permission is hereby granted to copy and distribute this license without
modification.  This license may not be modified without the express written
permission of its copyright owner.

\end{verbatim}

Licenza non molto utilizzata, si basa su BSD-3clausole con alcune aggiunte:
\begin{itemize}
\item Licenza brevettuale per i brevetti posseduti dal licenziante $\to$ non si ha alcuna garanzia sulla violazione di brevetti
\item Notule di attribuzione
\item Garanzia sulla propriet\`a del software $\to$ garantisce di non violare diritti di terzi
\item Terminazione per citazione su base brevettuale $\to$ simile alla apache license
\item Foro di competenza $\to$ chi cre\`o questa licenza voleva che in caso di una qualsiasi citazione il foro di comptenza fosse il foro in cui viveva l'autore in questo momento
\item Spese legali $\to$ sono previste una suddivisione delle spese legali
\end{itemize}

\subsubsection{Filosofia delle licenze BSD}

Le licenze BSD permettono sfruttamenti proprietari del software che viene spesso creato come prodotto di consorzi che lavorano insieme, che hanno tutto interesse di sviluppare un progetto comune (ad esempio un protocollo), in modo da permetterne la diffusione. Un software proprietario che include il software licenziato sotto BSD ne rafforza la sua posizione, in quanto viene diffuso. Lo scopo delle licenze copyleft \`e costruire una struttura di software libero per poter offrire una base comunitaria.

\newpage

\subsubsection{GPL v2}

\begin{verbatim}
                    GNU GENERAL PUBLIC LICENSE
                       Version 2, June 1991

 Copyright (C) 1989, 1991 Free Software Foundation, Inc.,
 51 Franklin Street, Fifth Floor, Boston, MA 02110-1301 USA
 Everyone is permitted to copy and distribute verbatim copies
 of this license document, but changing it is not allowed.

                            Preamble

  The licenses for most software are designed to take away your
freedom to share and change it.  By contrast, the GNU General Public
License is intended to guarantee your freedom to share and change free
software--to make sure the software is free for all its users.  This
General Public License applies to most of the Free Software
Foundation's software and to any other program whose authors commit to
using it.  (Some other Free Software Foundation software is covered by
the GNU Lesser General Public License instead.)  You can apply it to
your programs, too.

  When we speak of free software, we are referring to freedom, not
price.  Our General Public Licenses are designed to make sure that you
have the freedom to distribute copies of free software (and charge for
this service if you wish), that you receive source code or can get it
if you want it, that you can change the software or use pieces of it
in new free programs; and that you know you can do these things.

  To protect your rights, we need to make restrictions that forbid
anyone to deny you these rights or to ask you to surrender the rights.
These restrictions translate to certain responsibilities for you if you
distribute copies of the software, or if you modify it.

  For example, if you distribute copies of such a program, whether
gratis or for a fee, you must give the recipients all the rights that
you have.  You must make sure that they, too, receive or can get the
source code.  And you must show them these terms so they know their
rights.

  We protect your rights with two steps: (1) copyright the software, and
(2) offer you this license which gives you legal permission to copy,
distribute and/or modify the software.

  Also, for each author's protection and ours, we want to make certain
that everyone understands that there is no warranty for this free
software.  If the software is modified by someone else and passed on, we
want its recipients to know that what they have is not the original, so
that any problems introduced by others will not reflect on the original
authors' reputations.

  Finally, any free program is threatened constantly by software
patents.  We wish to avoid the danger that redistributors of a free
program will individually obtain patent licenses, in effect making the
program proprietary.  To prevent this, we have made it clear that any
patent must be licensed for everyone's free use or not licensed at all.

  The precise terms and conditions for copying, distribution and
modification follow.

                    GNU GENERAL PUBLIC LICENSE
   TERMS AND CONDITIONS FOR COPYING, DISTRIBUTION AND MODIFICATION

  0. This License applies to any program or other work which contains
a notice placed by the copyright holder saying it may be distributed
under the terms of this General Public License.  The "Program", below,
refers to any such program or work, and a "work based on the Program"
means either the Program or any derivative work under copyright law:
that is to say, a work containing the Program or a portion of it,
either verbatim or with modifications and/or translated into another
language.  (Hereinafter, translation is included without limitation in
the term "modification".)  Each licensee is addressed as "you".

Activities other than copying, distribution and modification are not
covered by this License; they are outside its scope.  The act of
running the Program is not restricted, and the output from the Program
is covered only if its contents constitute a work based on the
Program (independent of having been made by running the Program).
Whether that is true depends on what the Program does.

  1. You may copy and distribute verbatim copies of the Program's
source code as you receive it, in any medium, provided that you
conspicuously and appropriately publish on each copy an appropriate
copyright notice and disclaimer of warranty; keep intact all the
notices that refer to this License and to the absence of any warranty;
and give any other recipients of the Program a copy of this License
along with the Program.

You may charge a fee for the physical act of transferring a copy, and
you may at your option offer warranty protection in exchange for a fee.

  2. You may modify your copy or copies of the Program or any portion
of it, thus forming a work based on the Program, and copy and
distribute such modifications or work under the terms of Section 1
above, provided that you also meet all of these conditions:

    a) You must cause the modified files to carry prominent notices
    stating that you changed the files and the date of any change.

    b) You must cause any work that you distribute or publish, that in
    whole or in part contains or is derived from the Program or any
    part thereof, to be licensed as a whole at no charge to all third
    parties under the terms of this License.

    c) If the modified program normally reads commands interactively
    when run, you must cause it, when started running for such
    interactive use in the most ordinary way, to print or display an
    announcement including an appropriate copyright notice and a
    notice that there is no warranty (or else, saying that you provide
    a warranty) and that users may redistribute the program under
    these conditions, and telling the user how to view a copy of this
    License.  (Exception: if the Program itself is interactive but
    does not normally print such an announcement, your work based on
    the Program is not required to print an announcement.)

These requirements apply to the modified work as a whole.  If
identifiable sections of that work are not derived from the Program,
and can be reasonably considered independent and separate works in
themselves, then this License, and its terms, do not apply to those
sections when you distribute them as separate works.  But when you
distribute the same sections as part of a whole which is a work based
on the Program, the distribution of the whole must be on the terms of
this License, whose permissions for other licensees extend to the
entire whole, and thus to each and every part regardless of who wrote it.

Thus, it is not the intent of this section to claim rights or contest
your rights to work written entirely by you; rather, the intent is to
exercise the right to control the distribution of derivative or
collective works based on the Program.

In addition, mere aggregation of another work not based on the Program
with the Program (or with a work based on the Program) on a volume of
a storage or distribution medium does not bring the other work under
the scope of this License.

  3. You may copy and distribute the Program (or a work based on it,
under Section 2) in object code or executable form under the terms of
Sections 1 and 2 above provided that you also do one of the following:

    a) Accompany it with the complete corresponding machine-readable
    source code, which must be distributed under the terms of Sections
    1 and 2 above on a medium customarily used for software interchange; or,

    b) Accompany it with a written offer, valid for at least three
    years, to give any third party, for a charge no more than your
    cost of physically performing source distribution, a complete
    machine-readable copy of the corresponding source code, to be
    distributed under the terms of Sections 1 and 2 above on a medium
    customarily used for software interchange; or,

    c) Accompany it with the information you received as to the offer
    to distribute corresponding source code.  (This alternative is
    allowed only for noncommercial distribution and only if you
    received the program in object code or executable form with such
    an offer, in accord with Subsection b above.)

The source code for a work means the preferred form of the work for
making modifications to it.  For an executable work, complete source
code means all the source code for all modules it contains, plus any
associated interface definition files, plus the scripts used to
control compilation and installation of the executable.  However, as a
special exception, the source code distributed need not include
anything that is normally distributed (in either source or binary
form) with the major components (compiler, kernel, and so on) of the
operating system on which the executable runs, unless that component
itself accompanies the executable.

If distribution of executable or object code is made by offering
access to copy from a designated place, then offering equivalent
access to copy the source code from the same place counts as
distribution of the source code, even though third parties are not
compelled to copy the source along with the object code.

  4. You may not copy, modify, sublicense, or distribute the Program
except as expressly provided under this License.  Any attempt
otherwise to copy, modify, sublicense or distribute the Program is
void, and will automatically terminate your rights under this License.
However, parties who have received copies, or rights, from you under
this License will not have their licenses terminated so long as such
parties remain in full compliance.

  5. You are not required to accept this License, since you have not
signed it.  However, nothing else grants you permission to modify or
distribute the Program or its derivative works.  These actions are
prohibited by law if you do not accept this License.  Therefore, by
modifying or distributing the Program (or any work based on the
Program), you indicate your acceptance of this License to do so, and
all its terms and conditions for copying, distributing or modifying
the Program or works based on it.

  6. Each time you redistribute the Program (or any work based on the
Program), the recipient automatically receives a license from the
original licensor to copy, distribute or modify the Program subject to
these terms and conditions.  You may not impose any further
restrictions on the recipients' exercise of the rights granted herein.
You are not responsible for enforcing compliance by third parties to
this License.

  7. If, as a consequence of a court judgment or allegation of patent
infringement or for any other reason (not limited to patent issues),
conditions are imposed on you (whether by court order, agreement or
otherwise) that contradict the conditions of this License, they do not
excuse you from the conditions of this License.  If you cannot
distribute so as to satisfy simultaneously your obligations under this
License and any other pertinent obligations, then as a consequence you
may not distribute the Program at all.  For example, if a patent
license would not permit royalty-free redistribution of the Program by
all those who receive copies directly or indirectly through you, then
the only way you could satisfy both it and this License would be to
refrain entirely from distribution of the Program.

If any portion of this section is held invalid or unenforceable under
any particular circumstance, the balance of the section is intended to
apply and the section as a whole is intended to apply in other
circumstances.

It is not the purpose of this section to induce you to infringe any
patents or other property right claims or to contest validity of any
such claims; this section has the sole purpose of protecting the
integrity of the free software distribution system, which is
implemented by public license practices.  Many people have made
generous contributions to the wide range of software distributed
through that system in reliance on consistent application of that
system; it is up to the author/donor to decide if he or she is willing
to distribute software through any other system and a licensee cannot
impose that choice.

This section is intended to make thoroughly clear what is believed to
be a consequence of the rest of this License.

  8. If the distribution and/or use of the Program is restricted in
certain countries either by patents or by copyrighted interfaces, the
original copyright holder who places the Program under this License
may add an explicit geographical distribution limitation excluding
those countries, so that distribution is permitted only in or among
countries not thus excluded.  In such case, this License incorporates
the limitation as if written in the body of this License.

  9. The Free Software Foundation may publish revised and/or new versions
of the General Public License from time to time.  Such new versions will
be similar in spirit to the present version, but may differ in detail to
address new problems or concerns.

Each version is given a distinguishing version number.  If the Program
specifies a version number of this License which applies to it and "any
later version", you have the option of following the terms and conditions
either of that version or of any later version published by the Free
Software Foundation.  If the Program does not specify a version number of
this License, you may choose any version ever published by the Free Software
Foundation.

  10. If you wish to incorporate parts of the Program into other free
programs whose distribution conditions are different, write to the author
to ask for permission.  For software which is copyrighted by the Free
Software Foundation, write to the Free Software Foundation; we sometimes
make exceptions for this.  Our decision will be guided by the two goals
of preserving the free status of all derivatives of our free software and
of promoting the sharing and reuse of software generally.

                            NO WARRANTY

  11. BECAUSE THE PROGRAM IS LICENSED FREE OF CHARGE, THERE IS NO WARRANTY
FOR THE PROGRAM, TO THE EXTENT PERMITTED BY APPLICABLE LAW.  EXCEPT WHEN
OTHERWISE STATED IN WRITING THE COPYRIGHT HOLDERS AND/OR OTHER PARTIES
PROVIDE THE PROGRAM "AS IS" WITHOUT WARRANTY OF ANY KIND, EITHER EXPRESSED
OR IMPLIED, INCLUDING, BUT NOT LIMITED TO, THE IMPLIED WARRANTIES OF
MERCHANTABILITY AND FITNESS FOR A PARTICULAR PURPOSE.  THE ENTIRE RISK AS
TO THE QUALITY AND PERFORMANCE OF THE PROGRAM IS WITH YOU.  SHOULD THE
PROGRAM PROVE DEFECTIVE, YOU ASSUME THE COST OF ALL NECESSARY SERVICING,
REPAIR OR CORRECTION.

  12. IN NO EVENT UNLESS REQUIRED BY APPLICABLE LAW OR AGREED TO IN WRITING
WILL ANY COPYRIGHT HOLDER, OR ANY OTHER PARTY WHO MAY MODIFY AND/OR
REDISTRIBUTE THE PROGRAM AS PERMITTED ABOVE, BE LIABLE TO YOU FOR DAMAGES,
INCLUDING ANY GENERAL, SPECIAL, INCIDENTAL OR CONSEQUENTIAL DAMAGES ARISING
OUT OF THE USE OR INABILITY TO USE THE PROGRAM (INCLUDING BUT NOT LIMITED
TO LOSS OF DATA OR DATA BEING RENDERED INACCURATE OR LOSSES SUSTAINED BY
YOU OR THIRD PARTIES OR A FAILURE OF THE PROGRAM TO OPERATE WITH ANY OTHER
PROGRAMS), EVEN IF SUCH HOLDER OR OTHER PARTY HAS BEEN ADVISED OF THE
POSSIBILITY OF SUCH DAMAGES.

                     END OF TERMS AND CONDITIONS
\end{verbatim}

Obbliga i distributori a mantenere il software libero. Il funzionamento della GPL non prevede la firma di alcun contratto e lo si utilizza a patto che si accettino alcune condizioni, che sono:
\begin{itemize}

\item Libera copia non modificata a patto di mantenere le licenze e le clausole di salvaguardia intatte. Questo \textit{permette di poter vendere le copie} e di aggiungerne una garanzia in quanto la GPL normalmente non fornisce alcuna garanzia.
\item Libera distribuzione di versioni modificate con le seguenti clausole:
  \begin{itemize}
  \item Inserimento di avvisi di modifica
  \item Il risultato deve essere sotto GPL
  \item Gli avvisi interattivi sulla licenza devono rimanere
  \item Lavoro ``as a whole'' e non
  \end{itemize}

\end{itemize}
