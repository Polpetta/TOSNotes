\documentclass[11pt,a4paper,openany]{article}

%\usepackage[latin1]{inputenc}

\usepackage{hyperref}
\usepackage[utf8]{inputenc}
\usepackage[italian]{babel}
\usepackage{lmodern}
\hypersetup{
  colorlinks,
  citecolor=gray,
  filecolor=red,
  linkcolor=blue,
  urlcolor=blue
}

\usepackage{amsmath}
\usepackage{graphicx}
\usepackage{float}
\usepackage{amsfonts}
\usepackage{listings}
\lstset{
	commentstyle=\color{green},
	frame=single,
	keepspaces=true,
	keywordstyle=\color{blue},
	numbers=left,
	numberstyle=\tiny\color{black},
	rulecolor=\color{black},
        basicstyle=\ttfamily
}

\renewcommand\arraystretch{1.25}
\usepackage{array}
\usepackage{ragged2e}
\newcolumntype{C}[1]{>{\centering\let\newline\\\arraybackslash\hspace{0pt}}m{#1}}


\title{\textbf{Appunti di Tecnologie Open Source}}
\author{Polonio Davide}

%\date{1/10/2015}

\begin{document}

\maketitle

\tableofcontents
%da includere altre parti degli appunti (una pagina a giornata?)

\newpage

\section{Lezione 06-10-15}

\subsection{Introduzione al software libero}

Definizione: Il software libero\footnote{Libero \`e diverso da gratuito.} \`e software che garantisce le seguenti quattro libert\`a fondamentali:

\begin{enumerate}

\item Eseguire il programma per qualsiasi scopo: un programma libero non pu\`o mai imporre:

  \begin{itemize}

  \item Restrizioni in termini di tempo
  \item Restrizioni in termini di scopo
  \item Limitazioni di area geografica $\to$ questo pu\`o essere un caso speciale

  \end{itemize}

\item Studiare come funziona il programma e adattarlo alle proprie necessit\`a.
Nessuna restrizione sulla modifica e comprensione del tipo:

  \begin{itemize}

  \item Richiedere l'acquisto di licenze speciali
  \item Richiedere le firma di NDA
  \item Impedire l'accesso al codice sorgente

  \end{itemize}

\item Redistribuire copie in modo da aiutare il prossimo. Il software libero non proibisce di prestare la propria copia ad una persona o darne una copia, \underline{nemmeno dietro un pagamento di un compenso}

  \item Migliorare il programma e distribuire pubblicamente i miglioramenti, in modo tale che tutta la comunit\`a ne tragga beneficio. Infatti migliorare il programma e distribuire i miglioramenti permette a chi non ha il tempo o le capacita' per risolvere un problema di accedere indirettamente alla liber\`a di modifica $\to$ anche questo pu\`o essere dietro compenso.

\end{enumerate}

\subsubsection{Importanza del software libero}

\begin{itemize}

\item Riduzione dei costi

\item Trasparenze $\to$ soprattutto per gli enti statali

\item Nessun lock-in (si veda il caso di XFree86)

\item Sicurezza e affidabilit\`a\footnote{Nota: da un punto di vista pratico la qualit\`a del software libero tende a essere pari con il miglior software proprietario. Il problema \`e che \`e possibile che siano presenti bug di sicurezza non noti agli sviluppatori che possono essere sfruttati a scopo malevolo, senza che venga segnalato alla comunit\`a}

\end{itemize}

Il software libero, diversamente da quello proprietario, \`e un'infrastruttura e cambia il modo in cui si fa impresa (si veda Openerp, CUPS, Android). Quello che si sta capendo \`e che le molte menti creative senza lavorare per una certa azienda possono dare comunque una mano a quel determinato software, oltre che, dall'iterazione con gli utenti si hanno nuove idee, perch\`e viene gestito il progetto insieme.

\subsection{Albori del diritto d'autore}

\subsubsection{Venezia e i ``Privilegi''}

Nella Venezia del 1500 c.a. erano presenti i Privilegi. I privilegi si riferivano alla tecnologia utilizzata. I privilegi erano garantiti ai stampatori, ma non agli autori (rigurardo la pubblicazione e alla stampa di documenti-libri)

Nel 1517-1537 si va verso il privilegio d'autore. Si afferma infatti il ruolo dei privilegi sui libri comuni, con una limitazione della durata dei privilegi fino a 10 anni. Tutto ci\`o porta alla richieste di opere originali, con lo spostamente delle protezione delle opere: si cominciano a proteggere anche le modifiche\footnote{Inizialmente le opere modificate-rielaborate e ripubblicate erano considerate diverse dall'opera iniziale, e non una sua ``derivata''}

Le cooperazioni veneziane attive dal tredicesimo secolo, adibite alle professioni artigiane proteggono scrupolosamente le conoscenze artigiane:
\begin{itemize}

\item Restrizioni sui movimenti degli artigiani
\item Conoscenze trasmesse esclusivamente per via orale
\item Meccanismi per limitare la crescita delle singole attivit\`a

\end{itemize}

Con tutti i controlli nasce l'idea del diritto immateriale. Tutto il riserbo sulle conoscenze gli fanno acquisire importanza. Nel 1474 si hanno i privilegi d'autore (di cui accennato prima) a Venezia, nel tentativo di portare ordine:

\begin{itemize}

\item Non creava percorsi obbligati, si limitava a codificare le politiche relative ai privilegi

\item Mirata principalmente agli inventori e non alle corporazioni

\end{itemize}

Con l'avvento dell'umanismo si ha un aumento di persone che tentano di ``scappare'' dalle corporazioni e ottenere dei provilegi: si ha una maggiore attenzione alla necessit\`a di proteggere il proprio lavoro. Molti dotti cominciano a preoccuparsi a problemi pratici, con un assottigliamento dei confini tra conoscienza e capacit\`a meccaniche. Da qui vengono scritti i primi libri in cui si cerca di avere una maggiore importanza alla protezione delle conoscienze, e comincia a cambiare il modo in cui esse vengono trasmesse.


\subsubsection{Inghilterra e Copyright}

Con la prima stamperia nel 1476 (in ritardo ripetto a Venezia) i problemi di copia non autorizzate hanno una ridotta importanza. Con l'introduzione della censura da parte della Corona, e nel 1557 viene dato il controllo esclusivo sulla stampa dei libri alla Stationers Company, posta sotto il controllo della Camera Stellata. Agendo in tale maniera gli autori vengono estromessi dai propri lavori.

Nel 1640 si ha l'abolizione della Camera Stellata e delle limitazioni sulla stampa. Data l'esplosione dei giornali e delle libera stampa si ha che dal 1643-1694 viene instaurata di nuovo la censura da parte della Corona. Cos\`i facendo, dal 1695-1704 si hanno 13 tentativi di restaurare dei controlli censori. Nel 1740 con l'editto di Ann si ha lo scopo del Copyright visto come stimolazione della cultura, proteggendo le creazioni intangibili, e non pi\`u le copie come prima.
La licenza passa agli autori e non alle stamperie con una protezione di 14 anni e con la richiesta di estensione di 14 anni dopo la scadenza.

\subsubsection{Colonie Inglesi e USA}

In America nel 1638 reverendo Glover porta una macchina da stampa. Le autorit\`a del Massachusetts (dove in particolare si trovava la macchina) contribuiscono alla sua manutenzione. Nel 1672 viene dato il primo privilegio di stampa al signor Usher di stampare le leggi della colonia.

Il primo diritto d'autore viene concesso nel 1781: Andrew Law scrive una collezione di melodie, e per bloccare la competizione di altre collezioni simili chiede un monopolio.

Nel 1783 una petizione di John Cedgard cambia la situzione. Dopo aver scritto la sua stesura chiede il privilegio di stampa e dopo essergli stato concesso dalle autorit\`a, viene creata una \textit{Connecticut Copyright Statue}, stabilendo una legislazione generale (per quella regione). Sette anni dopo, nel 1790 si ha il Copyright Act, che da le prime regolamentazioni sul Copyright.

Con la diffusione del Copyright, si ha che le nazioni non riconoscono il Copyright di altri stati: nasce quindi l'esigenza di una legislazione uniforme. Nel 1883 si ha la Convenzione di Berna in cui le varie leggi vengono standardizzate. Inoltre viene resa automatica la tutela, senza nessuna registrazione. Il Copyright copriva solo gli oggetti tangibili e l'accordo includeva 165 paesi. Nel 1908 si deve affrontare White-Smith Music in una causa contro Apollo, primo caso legato a macchinari ``computazionali''. Si decide che la copia di musica su tubi per pianole non \`e una copia nel senso del Copyright Act. Si ha l'intervento del Congresso che afferma che i lavori derivati che possono essere percepiti solo attraverso l'uso i una macchina sono soggetti al Copyright. Si affronta il problema istituendo una legge ad hoc.

Nel 1950 con la nascita dei computer non si sente la necessit\`a del Copyright. Il software \`e piuttosto pensato principalmente come complemento alla macchina: poche esigenze di una protezione, e manca una legge sul Copyright software e i contratti che non si applicano a terzi\footnote{Data la mancanza di una legge sul copyright software se si volevano delle restrizioni nell'utilizzo del software da parte degli utenti si era soliti stipulare un contratto}. Inoltre il Copyright Act del 1909 imponeva la registrazione del materiale protetto ma non era chiaro cosa implicasse per il software.
C'\`e il desiderio delle case software di proteggere il software sviluppato, e nel 1976 con il nuovo Copyright Act si ha che il software diviene proteggibile da Copyright. Una commissione viene incaricata di indagare sul grado di protezione dato al software. La commissione racomand\`o una definizione di software, e di permettere la copia o l'adattamento del software a patto che servisse a scopo di backup e fossero necessari per l'utilizzo del software. Viene deciso che le copie esatte potevano essere vendute a costo che venissero trasferite completamente al nuovo proprietario. Le copie modificate potevano inoltre essere usate solo dal legittimo proprietario.

Nel 1980 il congresso approva i suggerimenti della commisione. Ma si ha che il termine ``legittimo possessore di una copia'' viene sostituito da ``proprietario di una copia''. Ci\`o porter\`a al caso giudiziario MAI vs Plack.

Nel 1990 viene limitato l'uso di piena vendita.

\subsubsection{Diritto in Italia oggi}

Diritto esclusivo dell'autore su:
\begin{itemize}

\item Ridistribuzione
\item Modifica
\item Adattamnto
\item Traduzione

\end{itemize}

Tale diritto\footnote{In Italia il diritto si suddivide in due tipi: Morale e Economico. Solamente il diritto Economico pu\`o essere trasferito/rinunciabile} \`e:
\begin{itemize}

\item Rinunciabile
\item Trasferibile

\end{itemize}

Licenze software: ad esempio le licenze di Windows. Le licenze GPL non vogliono fungere da contratto. La licenza GNU vuole differire dalle solite licenze software.

\section{Lezione 13-10-15}

\subsection{GNU}

Nota: argomenti trattati nel libro \textit{Heroes of the computer: Cap. 1,2 ed Epilogo}

Gli hack\footnote{In informatica: hack \`e un esercizio di codice che dimostra l'inventiva dell'autore, creato anche per il piacere dstesso dello scriverlo} del mit sono scherzi anonimi che dimostrano la creativit\`a della persona che li eseguono. Son presenti fino dagli anni 50.

\subsubsection{Gli albori}

Nel 1950-1960 la culla hacker nasce ai laboratori del MIT tra i ragazzi del Tech Model Railboard Club. Il primo gruppo lo si ha dal sottogruppo Signal \& Power, che si occupava della gestione della circuiteria e segnali dei treni. Questa complessit\`a rasentava quella del software. Il Corso di intelligenza arificiale di McCarthy del 1959 facilit\`o la formazione di questo gruppo, permettendo di cominciare a programmare le prime macchine. I primi hack si ebbero sull'IBM 704, tramite la programmazione non interattiva\footnote{Queste macchine sono i primi esempi di macchine commerciali e si programmavano con le schede perforate e il loro accesso per l'utilizzo era molto ristretto.}. Con l'IBM 407\footnote{Macchinetta di servizio per la perforazione delle schede} e determinate modifiche era possibile usare il macchinario per poter programmare. La vera programmazione si vede per\`o con il TX0\footnote{La programmazione in assembler ammetteva 4 istruzioni}, dove la politica di utilizzo era un po' pi\`u ``rilassata''.
Caratteristiche principali della cultura hacker:
\begin{itemize}

\item Giocosit\`a
\item Condivisione degli sforzi di sviluppo (programmare sulle macchine di quel tempo richiedeva una notevole mole di lavoro)
\item Politica di apertura agli esterni (ovviamente se si era capaci). Peter Deutsch, ad esempio, si avvicin\`o alla comunit\`a del MIT molto presto.
\item Lavoro notturno\footnote{Permetteva un utilizzo maggiore delle macchine perch\`e la notte si aveva una minore affluenza di personale}

\end{itemize}

Nel 1961 arriva il PDP-1, primo computer concentrato sull'utente piuttosto che sull'ottimizzazione delle risorse, offrendo un approccio interattivo e inserendo nuovi tool per lo sviluppo. La progettazione \`e assegnata a Gurley (ex membro del MIT), che si basa sul TX-0 e TX-2. Ci\`o porta a un rafforzamento della comunit\`a hacker, grazie anche alla nascita di ARPANET.

\subsubsection{Etica hacker}

\begin{itemize}
  
\item L'accesso al computer, e a tutto ci\`o da cui si pu\`o imparare dovrebbe essere illimitato e totale. L'apprendimento avveniva non per basi teoriche, ma per basi pratiche (si voleva che il computer venisse ``aperto'' e studiato). Il ``Midnight Requisitioning Committee'' era un comitato non ufficiale che si occupava di requisire componentistica elettronica principalmente inutilizzata per poterla riusare.

\item L'informazione dovrebbe essere libera: senza l'informazione \`e impossibile capire e quindi migliorare i sistemi, con l'idea che l'informazione dovrebbe essere come il flusso di bit in un computer. Era inconcepibile quindi che un software fosse proprietario.

\item ``Non fidarti dell'autorit\`a e promuovi la decentralizzazione''. Secondo la comunit\`a hacker, l'autorit\`a porta con s\`e la burocrazia, che promuove regole arbitrarie che mirano solo alla propria perpetuazione.

\item Gli hacker dovrrebbero essere giudicati per il loro valore, non sulla base di fattori come razza, religione, sesso o posizione sociale. L'unica cosa che conta \`e quanto l'hacker pu\`o contribuire all'avanzamento dello stato dell'hacking.

\item \`E possibile creare arte e bellezza su un computer. Infatti i computer di quel tempo disponevano di pochissime risorse, e l'eleganza nella programmazione \`e vista come bellezza.

\end{itemize}
 
\subsubsection{Incompatible Timesharing System}

Con le macchine non-timesharing si avevano code lunghe per l'utilizzo del computer, causando problemi. Il progetto MAC puntava alla costruzione di una rete come quella elettrica destinata a distribuire potenza di calcolo.

Il timesharing era mal visto dagli hacker, che gli ricordava il Multics o CTSS (Compatible Time Sharing System), in quanto non si poteva avere controllo totale della macchina. Alcuni programmi che gli hacker sviluppavano avevano veramente bisogno di tutte le risorse del computer disponibili. Era quindi necessario aver bisogno di un accesso totale, e si arriv\`o a un compromesso: timesharing durante il giorno e single mode durante la notte, e per questo era necessario uno sviluppo di un SO in timesharing ispirato all'etica hacker: ITS\footnote{Incompatible Timesharing System}.

L'incompatible timesharing system aveva le seguenti funzionalit\`a:
\begin{itemize}

\item Utenti multipli ma anche programmi multipli per ogni utente
\item No password e assenza di sistema di permessi
\item File personali per ogni utente, ma disponibili a tutti
\item Strumenti collaborativi (es: possibilit\`a di switchare terminale e passare al terminale di un altro utente)
\item Fede negli utenti

\end{itemize}

L'ITS era quindi un sistema vivo che cresceva  con i suoi utilizzatori.

Con l'avvento del nuovo sistema, il PDP-10, si ha una crisi: viene infatti proposto di utilizzare TENEX, un nuovo OS al posto di ITS. Questo porta nel 1968 alle manifestazioni contro il laboratorio, con conseguenti atriti.

\subsubsection{Stallman - L'ultimo vero hacker}

Richard Stallman nasce a New York nel 1953. Ha le prime esperienze con i computer all'IBM New York Scientific Center ed entra nel laboratorio del MIT attorno agli anni '70, quando il laboratorio si sta avvicinando alla fine. Non era un informatico, ma era interessato all'informatica in generale. Nel 1971 entra a Harward, ma si interessa di pi\`u al MIT, dove viene assunto da Russel Noftsker come programmatore di sistema. Stallman quindi si avvicina alla cultura hacker, e lavora insieme a Richard Greenblatt e Bill Gosper\footnote{Pi\`u incline nella risoluzione di problemi ambito ``lato matematico''.}.

\paragraph*{Emacs} (Editing Macros) Dalla telescrivente si hanno dei passaggi verso i primi monitor con una riga. I primi programmi per gli editor sono Expensive Typewriter e TECO\footnote{Type Editor and Corrector}, che fungevano da insieme di strumenti a cui si lavorava applicando un insieme di comandi al testo. Questi tool erano scomodi da usare, quindi Stallman decise di cercare un altro tipo di software altrove. Non trovandolo, fece un insieme di MACRO per TECO, che rendeva possibile:
\begin{itemize}
  
\item Editare il testo \textit{real time}
\item Permettere il \textit{random access editing}
\item Consentire l'aggiunta di ulteriori MACRO

\end{itemize}

La nascita di Emacs si ha dal caos dovuto alle troppe MACRO nate. Steele propone di generare un ordine nell'universo delle MACRO. Inizialmente impone una clausola che imponeva che ogni modifica a Emacs fosse inviata allo sviluppatore principale, in modo tale che, se fosse stata un'ottima idea, sarebbe stato possibile integrarla in Emacs e renderla disponibile a tutti. Questa clausola permise alle persone di lavorare insieme agli altri in una community ma impose una limitazione alla libert\`a di sviluppo.

\subsection{La crisi del movimento Hacker del MIT}

\paragraph*{Le prime incursioni} Con l'avvento delle password, Stallman cerca di convincere le altre persone a limitarne l'utilizzo, e a permettere agli altri utenti di utilizzare i file di tutti. Questo porta all'intervento del ministero della Difesa, che lo costringe all'uso di password. Ci\`o aliena la comunit\`a vicino a Stallman, anche a causa dell'introduzione dello \textit{sciopero del software}:
Stallman rifuita di fornire allo staff del laboratorio le ultime versioni di Emacs fino a quando non avessero eliminato il sistema di sicurezza nel laboratorio. La time bomb di Scribe fa prendere a Stallman la scelta di opporsi alle restrizioni sull'utilizzo del software.

\subsubsection{Cambiamento di mentalit\`a}

Nel 70-80 si ha una frammentazione della cultura hacker (nel 1976 si ha il copyright act), causata dal fatto che gli Hacker originari abbandonano il MIT per lavorare-aprire la propria azienda. Si ha un cambiamento dei visitatori al MIT, e con essi cominciano a essere presenti i primi programmi protetti da copyright nel laboratorio di intelligenza artificiale.

Questo a causa di un cambiamento degli equilibri di forze tra gli hacker e gli studenti-professori, che hanno un modo completamente diverso di vedere il software (ovvero come uno strumento).

\paragraph*{La lisp machine} La nascita della lisp\footnote{Linguaggio al tempo ad alto livello, molto potente ma macchinoso} machine causa la crisi finale. Dall'idea avuta da Greenblatt, viene costruita una macchina concepita per funzionare in sintonia con Lisp, che ebbe successo. Con i copiosi fondi del progetto, al MIT vennero prodotte 32 macchine, che si voleva far comunicare in rete per favorirne la condivisione. Ci\`o porta all'idea di creare un'azienda ``hacker friendly'' per la produzione di lisp machine. Da ci\`o si ha lo scontro con Russel Noftsker, che propone di creare un'azienda ``per azioni'' e renderla prettamente commerciale. Dato lo scontro, viene raggiunto un accordo: a Greenblatt viene dato un anno di tempo per creare un'azienda per la vendita di lisp machines, che riesce a far partire entro l'anno (Lisp Machine Inc., LMI). Nonostante ci\`o viene creata un'altra azienda, la Symbolics. Ci\`o caus\`o:
\begin{itemize}

\item Svuotamento del MIT
\item LMI e soprattutto Symbolics attingono pesantemente dal MIT
\item 1982: Symbolics rende le proprie modifiche al SO delle lisp machines proprietario, causando la vendetta di Stallman, che si mise a replicare tutte le funzionalit\`a e a donarle a LMI.
\item La comunit\`a hacker s'indebolisce, perch\`e minoritaria rispetto a studenti e professori, e diventa difficile sostenere dentro il laboratorio un sistema proprio interno.

\end{itemize}

\paragraph*{Cambiamento del PDP} Con il cambiamento del PDP tutto il software relativo diventa obsoleto. Ci\`o apre un fronte:
\begin{itemize}

\item Continuare a utilizzare ITS $\to$ diventato vecchio e poco sicuro
\item Usare Twenex, software proprietario derivato da Tenex (verr\`a scelto quest'ultimo)

\end{itemize}

\subsubsection{Nasce GNU}

GNU nasce grazie al caso della stampante Xerox, che veniva venduta con un software sotto la media e che tendeva a far inceppare spesso la stampante. Il software dato era proprietario, e i sorgenti sotto NDA\footnote{Atto di non divulgazione}. Stallman non poteva quindi migliorare il driver della stampante, e con ci\`o lasci\`o il MIT e and\`o a fondare il progetto \textit{GNU} nel 1983.

\paragraph*{L'appello} Nel 1983, su net.unix-wizard Stallman lancia dopo il giorno del ringraziamento un nuovo software Unix compatibile, che chiam\`o GNU\footnote{Gnu is Not Unix}. Unix venne scelto come base perch\`e:
\begin{itemize}

\item AT\&T, la proprietaria di Unix, non poteva venderlo causa della sua posizione dominante nella telefonia
\item Familiarit\`a con il codice sorgente
\item Portabilit\`a: Unix era stato sviluppato in un linguaggio creato appositamente per lui: il C.
\item Modularit\`a

\end{itemize}

Avendo bisogno di un compilatore, si mise a lavorare su Pastel, un compilatore libero. Purtroppo la struttura troppo pesante impediva a Pastel di lavorare su macchine poco potenti, e con ci\`o si mise alla stesura di GNU Emacs partendo dal codice di Gosling, reimplementando alcune funzionalit\`a sotto le minaccie legali di Unipress. Nel 1985 Stallman rilascia GNU Emacs, ponendo il problema di quale licenza usare per quel programma.

\paragraph*{GPL} Inizialmente, la licenza di GNU Emacs era ispirata dalle note di ``copyright'' sulle email. Inoltre, la diffusione del software in modo centralizzato, a differenza che con la commune. Per suggerimento di Gilmore si ha un cambio nome: si ha quindi la nascita della \textbf{GNU general public license}. GNU gpl v1 viene distribuita con il rilascio di gdb, con l'idea di unificare in un'unica licenza tutto il software GNU.

\paragraph*{L'incontro con BSD} Con la rottura del monopolio, AT\&T comincia a focalizzarsi sullo sviluppo di Unix a scopi commerciali. BSD era una distribuzione accademica derivata da Unix con vari contributi esterni ma richiedeva comunque il pagamento di una licenza AT\&T perch\`e non tutto il codice era di Berkeley. Nel 1984-1985 Stallman convince Bostic a sviluppare una distribuzione completamente libera.

Negli anni 80-90 Bruce Perens rilascia electric fence\footnote{Una libreria scritta in C} sotto GPL. Rich Morin fonda Prime Time Freeware, un'azienda che ricercava software open source nella rete e si occupava di rivendere i nastri.

La Cygnus era un'azienda che aveva cominciato a lavorare su gcc\footnote{L'idea era quella di contribuire a gcc e rivenderlo.}, facendone il porting al National Semiconductor's 32032. Data la modularit\`a di gcc, implement\`o il supporto a C++ e si occup\`o di portare il compilatore per sistemi embedded. Alla fine del 1990 aveva fatto $725.000$ \$ in supporto e contratti. L'azienda \`e stata ora venduta alla Red Hat.

\subsubsection{Espansione del progetto GNU}

Nel 1987 si ha la nascita di Libc:
\begin{itemize}

\item 1990 fork Linux
\item 1997 fork abbandonato

\end{itemize}

\paragraph*{GNU Hurd} Nel 1986 si ha il tentativo di basarsi su TRIX. Ma ci\`o causava il non funzionamento su macchine standard, e ci\`o portava a un numero troppo elevato di cambiamenti. Si prova quindi a basarsi sul codice BSD, ma si ha poca cooperazione da parte degli sviluppatori e si preferisce un approccio pi\`u ambizioso: basarsi su un microkernel usando \textit{Mach}. Nel 1990 iniziano i lavori sul kernel, ma si incontrano difficolt\`a di sviluppo, aggravata dalla poca attenzione dovuta dall'avvento di Linux. Attualmente Mach supporta i driver linux, supporta X, iceweasel e forse debian render\`a ufficiale una prossima release.

\subsection{BSD}

Testo di riferimento: \href{http://www.groklaw.net/staticpages/index.php?page=20051013231901859}{\textit{The Daemon, The GNU and the Penguin}}

\subsubsection{DARPA}

Nel 1957 si ha il lancio dello Sputnik1 da parte dell'URSS nello spazio. Gli americani, nel 1958 fondano l'ARPA (poi rinominata DARPA)\footnote{il cui scopo era lo sviluppo di nuove tecnologie per scopi militari.}, in cui veniva sviluppato MULTICS.

Nel 1963 nasce il progetto MAC\footnote{Multiple Access Computer, Machine Aided Cognitions}, sovvenzionato con due milioni di dollari dal DARPA. Gli obiettivi iniziali del progetto MAC erano quelli di rendere possibile l'affitto di potenza computazionale con la creazione di sistemi affidabili come quelli per la distribuzione di energia elettrica.

\paragraph*{Multics} Primo sistema operativo Hight Availability. \begin{itemize}

\item Sviluppato da MIT, General Eletric e Bell Labs (anche noto come AT\&T)
\item Struttura modulare: possibile aumentare le prestazioni del sistema semplicemente aggiungendo una ulteriore unit\`a (CPU, memoria, storage etc)
\item Riconfigurazione on-line
\item Linkaggio dinamico, filesystem gerarchico etc

\end{itemize}

Multics caus\`o un fallimento commerciale per via dell'estrema complessit\`a del sistema.

\paragraph*{Unix} Nel 1969 AT\&T si toglie dal progetto Multics. Dennis Richie e Ken Thompson desiderano continuare la ricerca sulle idee di Multics. Viene creata quindi una versione pi\`u leggera, compatibile con macchine pi\`u piccole, chiamata \textit{UNIX}. Nel 1972 Unix viene riscritto in C, per avere una maggiore portabilit\`a e maggiore facilit\`a di sviluppo. Nell'Ottobre del 1973 un articolo fa espandere la popolarit\`a di Unix.
Con il monopolio telefonico di AT\&T Unix viene distribuito liberamente, con la possibilit\`a di far liberamente delle modifiche (con codice sorgente). Nel 1977 John Lions pubblica il codice sorgente commentato di Unix, che causa un incremento dell'insegnamento di UNIX all'universit\`a, nonostante il tentativo di AT\&T di bloccare ci\`o. Due anni dopo con la distribuzione di Unix 7.0 AT\&T annuncia una restrizione sulla redistribuzione di Unix, che implicava sia una restrizione a livello commerciale sia una restrizione sulla possibilit\`a didattica nell'utilizzo di Unix.

\paragraph*{BSD Unix}
Nel 1973 John Fabris dell'universit\`a di Berkeley assiste al talk su Unix al SOSP\footnote{Symposiuom on Operating System}, e decide di fare un dual boot in una sua macchina. Nel 1975 vengono acquistate altre due nuove macchine Unix. Berkeley ha bisogno di un certo supporto ai suoi sistemi, e nel 1975 Chuck Haley e Bill Joy arrivano a Berkeley, cominciando lo viluppo di un compilatore e editor per Pascal. Quello che diede una spinta decisiva a BSD fu che DARPA decide di muovrere la loro DARPANET su UNIX. Nel 1980 i fondi di DARPA vengono usati per il miglioramento di BSD UNIX, che rilascia 4BSD. In tutto questo, AT\&T annuncia di voler commercializzare Unix ed entra in conflitto con BSD. Nel 1983 Unix viene completamente commercializzato, con licenze sui sorgenti molto costose, in particolare per il TCP. In tutto ci\`o BSD cerca di liberare la dipendenza da AT\&T, che viene coronata con NET2. BSD386 (poi rinominato BSDI), distribuzione completa basata su NET2 e a stampo commerciale nasce in quell'anno, facendo scaturire una causa tra Unix System Laboratories (USL), BSDI e Berkeley, finita con la vittoria di BSDI.

\section{Lezione del 20-10-15}

USL perde la causa, ma si ha un rallentamento della diffusione di BSD e nel 1994 si ha la definitiva chiusura del processo tra Berkeley e Novell. Nello stesso anno viene rilasciato 4.4BSD-Lite, una versione di BSD completamente libera\footnote{Ci furono 2 release, una con funzionalit\`a in pi\`u in cui erano presenti sorgenti AT\&T in cui bisognava pagare licenze aggiuntive}.

\subsection{Linux}

Materiale di riferimento: \textit{The Deamon, the GNU and the Penguin: a history of Free and Open Source}; Peter Salus.

\subsubsection{Minix}

\paragraph*{Premesse} John Lions pubblica il codice sorgente commentato di UNIX. Nel 1978-1979 vengono bloccati i commentari di John Lions, e si ha un aumento dei costi delle licenze e restrizioni sull'insegnamento in classe. Questo causa l'interruzione dell'insegnamento di UNIX in molte universit\`a.

\paragraph*{Minix} Andrew Tenenbaum, docente di Computer Science, decide di scrivere un proprio sistema operativo sulle traccie di Unix: Minix. Minix \`e V7 compatibile, completo di compilatore ed editor. Era pensato per scopi didattici, era rilasciato sotto licenza permissiva e non libera.

\paragraph*{Nascita di Linux} Nel 1990 l'universit\`a di Helsinki installa un MicroVax con Ultrix. Linus studia il libro di Tenenbaum, ma non possiede un computer con Unix per mettere in pratica ciò che ha imparato. Segue il corso di "C e Unix". Nel 1991 Linus viene portato alla conferenza di Stallman e nello stesso anno compra un PC e comincia a scriverci un emulatore di terminale a partire da Minix. Sempre nello stesso anno Linus rilascia Linux 0.0.1\footnote{All'inizio il nome non si chiamava Linux, ma Freax e veniva distribuito sotto licenza proprietaria, in quando non permetteva l'utilizzo commerciale}. Nell'anno successivo, con Linux 0.12\footnote{Si ha la prima richiesta di una nuova feature da parte di un ragazzo che aveva a disposizione un computer con poca RAM. Cosí Torvalds implementa il paging.} Linus cambia la licenza adottando la licenza GPL, dando la possibilit\`a a tutti a contribuire, comprese aziende come SUSE. Si ha anche la prima distribuzione Linux: Linux Pro con Yggdrasil\footnote{Un database inizilamente solo per UNIX}. Si ha anche il diverbio tra Tanembaum e Linus. Nel 1994 si arriva alla versione di Linux 1.0.

Nel 1993 Mark Ewing fonda RedHat, una distribuzione che viene rilevata da Bob Young e diventa RedHat Software Inc. Nel 1995 nasce RedHat Linux, che diventa la pi\`u diffusa distribuzione Linux, ora a pagamento. Vengono effettuati anche i primi porting di Linux a DEC Alpha e SUN SPARC, mentre nel 1996 si ha il supporto per multiprocessore. Nel 2011 si arriva a Linux 3.0 e nel 2015 Linux 4.0.

\subsubsection{Debian}

Distro guidata dalla nuova generazione hacker di Linus. Linux come kernel stava prendendo suo vigore, creando una comunit\`a a sé stante allontanandosi dal software libero. Un motivo di ci\`o fu la causa che si ebbe nel 1988 tra Apple e Microsoft, che impegn\`o lo GNU nella LPF\footnote{League of programming freedom, orgranizzazione che si oppene ai brevetti software e ai copyright delle interfacce utenti}, impedendo che passasse il principio di copyright nella interfaccia grafica. Murdock allora per riavvicinare la comunit\`a Linux a GNU annuncia la sua intenzione di fare una distribuzione completamente libera. Avendo possibilit\`a di focalizzarsi sul progetto, nel 1996 viene rilasciata la versione 1.1, dopo la versione erronea 1.0 del 1995. Con Bruce Perens come nuovo DPL vengono scritte le Debian Free Software Guidelines (DFSG), dando un insieme di caratteristiche che le licenze dovevano avere per essere considerate conformi agli standard del parco software Debian. Su queste guide furono date le future definizioni di Open Source. Venne anche creato l'Open Hardware Certification Program, programma con il quale le aziende potevano certificarsi per essere compatibili con Linux. Il Software in the Public Interest\footnote{Organizzazione non a scopo di lucro di supporto alla creazione e diffusione di software libero.} era un ombrello legale tramite cui la Debian diventa un'entit\`a in cui venne riconosciuta dai governi.

\paragraph*{Caratteristiche} Debian presenta il patto sociale, dove si ha un'attenzione maniacale alla qualit\`a, in cui si adotta solo software libero (secondo la DFSG) e dove gli sviluppi e decisioni sono presi in maniera comunitaria. Data l'impossibilit\`a di adottare solamente software libero, la Debian ha messo a disposizione (anche se non supporta ufficilamente) software non-free (ovvero costituito da software proprietario) e contrib (software libero che per funzionare ha bisogno di software proprietario). Diversi tipi di distribuzioni:
\begin{itemize}
  
\item Stable: versione rilasciata per il pubblico utilizzo.
\item Testing: entrano tutti i pacchetti che dopo 15 giorni di permanenza in unstable non hanno bug critici.
\item Unstable: versione per sviluppatori altamente instabile.

\end{itemize}

\subsection{Open Source}

Libro: \textit{Opensources: voices from the Open Source Revolution}

Nel 1997 Raymond scrive il libro \textit{The Cathedral and the Bazaar}, dove viene analizzato dal punto di vista architetturale Linux, che dall'esterno sembra destinato al collasso in quanto i collaboratori sono dei terzi che quindi hanno la possibilit\`a di sviluppare e dirottare il progetto (per questo che viene associato al Bazaar). Analizzado tutto ci\`o si riesce a dettare delle linee guida sullo sviluppo di una comunit\`a open source:
\begin{itemize}

\item Ogni progetto deve partire da un ``prurito'' del programmatore
\item I buoni programmatori sanno cosa scrivere. I grandi programmatori sanno cosa riscrivere
\item Tratta i tuoi utenti come co-sviluppatori
\item Rilascia presto e spesso e ascolta i tuoi utenti
\item Dato un sufficiente numero di beta tester ogni problema verr\`a identificato e risolto
\item Riconoscere le buone idee degli utenti \`e importante come averne di proprie

\end{itemize}

Grazie a questi principi, nel 1998 netscape (il primo browser grafico) annuncia la volont\`a di liberare il proprio codice sorgente, rendendo Raymond una celebrit\`a. Dopo il rilascio si cerca una strategia di lungo termine per mettere in vista l'Open Source, e si decide di lanciare una campagna di marketing, denominata ``Open Source'' sostituendo la nomenclatura ``Free Software''. Torvalds appoggi\`o l'iniziativa Open Source, con l'idea di usare DFSG come definizione di Open Source, e nel 1998 nasce la Open Source Initiative (1998) fondata da Raymond e Perens, causando il coinvolgimento di persone e di programmi. Stretegie del movimento open Source:
\begin{itemize}

\item Approccio top-down
\item Puntare su Linux come dimostrazione $\to$ in quanto \`e molto software libero, e Linux aveva il fatto che poteva affascinare molte persone, dimostrando la forza del software Open Souce.
\item Puntare sulle Fortune 500 $\to$ i volontari tentano di puntare sulle piccole aziendine
\item Puntare sui media che influenzano le grosse aziende
\item Educare gli hacker sulle tattiche di promozione da seguire $\to$ troppa campagna portava ad avere effetti negativi
\item Usare un programma di certificazione $\to$ vengono rilasciate delle certificazioni per poter dichiarare se un software \`e open source o meno.

\end{itemize}

Il 7 marzo 1998 al Free Software Summit una ventina di leader del movimento appoggiano l'iniziativa, e si cerca di spingere altre aziende a seguire l'esempio di Netscape (Corel annuncia computer basati su Linux seguita da Oracle e Informix che annunciano il porting su linux).

I documenti Halloween sono dei documenti rilasciati erroneamente dalla Microsoft dove vengono sottolineati tutti i pericoli strategici che potevano venir dal movimento Open Source, e il fatto che colossi come Micosoft fossero preoccupati sottoline\`o l'importanza che questo movimento aveva preso.

\paragraph*{OSI} Il movimento Open Source prese una strategia diversa dal software libero, ovvero partendo da una serie di principi cercare una strategia per aumentare la visibilit\`a e raggiungere persone e aziende prima non possibile. La filosofia consisteva in:
\begin{itemize}

\item Licenze libere e permissive $\to$ per poter contribuire allo sviluppo di software in maniera comunitaria
\item Costruzione di una comunit\`a attorno al software $\to$ ci\`o permette alla comunit\`a di comprendere determinate scelte, facendo sentire la comunit\`a parte del processo decisionale
\item Trasparenza del processo di sviluppo $\to$ senza la possibilit\`a di modificarlo non \`e possibile interagire con il software e con la comunit\`a di utenti
\item Codice sorgente liberamente disponibile
\item Codice sorgente liberamente modificabile
\item Libera redistribuzione $\to$ insieme di linee guida che dovevano incanalare queste idee di base, costruite in modo di essere uno strumento di visione del software. Il movimento Open Source pubblica la sua definizione quasi come fosse un manifesto, spiegando il perch\`e di ogni punto. Imponendo la libera redistribuzione, si elimina la tentazione di rinunciare a importanti guadagni a lungo termine in cambio di un guadagno materiale a breve termine, ottenuto con il controllo delle vendite
\item Vincoli su altro software: la licenza non deve porre restrizioni su altro software distribuito insieme al software licenziato
\item Neutralit\`a rispetto alle tecnologie: la licenza non deve contenere clausole che dipendano o si basino su particolari tecnologie o tipi di interfacce

\end{itemize}

\paragraph*{Perl artistic license} La Perl artistic license sancisce che non \`e redistribuibile il software per scopi commerciali, mentre pu\`o essere incluso in una distribuzione pi\`u grande per essere venduto commercialmente. I software open source bloccano la redistribuzione di software commercialmente, ma permettono la redistribuzione dello stesso software ``wrapperizzato'' all'interno di qualcosa di pi\`u grande.


\section{Lezione del 27-10-15}
\graphicspath{ {res/data/27-10-15/} }

\paragraph*{Confronto con software libero} Le libert\`a tra il software libero e il software Open Source \`e simile, anche se hanno obiettivi diversi.

\subsection{Licenze software}

Materiale di riferimento: \textit{Understanding Open Source and Free Software Licensig, di Andrew Laurent}, \textit{Open Source Licensing: Sofrware Freedom and Intellectual property Law}

Capisaldi del diritto d'autore\footnote{Si riferiscono con i dati in America}:
\begin{itemize}
\item 70 anni dopo la morte per le persone
\item 95 anni dopo le pubblicazioni o 120 dopo la creazione per le corporazioni
\item Protezione per le sole opere espresse in forma percepibile
\item Nessuna registrazione richiesta
\item Work for hire: ogni volta che c'\`e un contratto di dipendenza i lavoratori sono visti come ``strumenti'', e il diritto \`e dell'azienda stessa. Questo non si applica per esempio per collaborazioni, se non diversamente specificato dal contratto.
\end{itemize}

\paragraph*{Garanzie} esistono diversi tipi di garanzie, che possono essere:
\begin{itemize}
\item Esplicite: in cui ci si assume la responsabilit\`a di una garanzia data
\item Implicite: garanzie che possono essere imposte per esempio per legge o dagli stati, come quelle di commercialit\`a, idoneit\`a e di non violazione di diritti terzi (garantisce che il software che si usa non violi in copyright software di altri)
\end{itemize}

In caso le garanzie non vengano rispettate possono esserci dei \textbf{danni}, che possono essere diretti o consequenziali.

\newpage

\subsubsection{MIT License}

\begin{verbatim}

The MIT License (MIT)

Copyright (c) [year] [fullname]

Permission is hereby granted, free of charge, to any person obtaining a copy
of this software and associated documentation files (the "Software"), to deal
in the Software without restriction, including without limitation the rights
to use, copy, modify, merge, publish, distribute, sublicense, and/or sell
copies of the Software, and to permit persons to whom the Software is
furnished to do so, subject to the following conditions:

The above copyright notice and this permission notice shall be included in all
copies or substantial portions of the Software.

THE SOFTWARE IS PROVIDED "AS IS", WITHOUT WARRANTY OF ANY KIND, EXPRESS OR
IMPLIED, INCLUDING BUT NOT LIMITED TO THE WARRANTIES OF MERCHANTABILITY,
FITNESS FOR A PARTICULAR PURPOSE AND NONINFRINGEMENT. IN NO EVENT SHALL THE
AUTHORS OR COPYRIGHT HOLDERS BE LIABLE FOR ANY CLAIM, DAMAGES OR OTHER
LIABILITY, WHETHER IN AN ACTION OF CONTRACT, TORT OR OTHERWISE, ARISING FROM,
OUT OF OR IN CONNECTION WITH THE SOFTWARE OR THE USE OR OTHER DEALINGS IN THE
SOFTWARE.

\end{verbatim}

Chiamata anche X License (creata per X.org) creata al MIT, \`e essenzialmente la licenza pi\`u libera che ci sia.
Punti importanti:
\begin{itemize}
\item Libera copia, modifica e ridistribuzione del software
\item Necessit\`a di mantenere la licenza originale
\item Possibilit\`a di sviluppi proprietari
\item Clausole di salvaguardia
\end{itemize}

\begin{figure}[h]
  \centering
  \includegraphics[scale=0.6]{27-10-15-01}
  \caption{In questo caso, se ho un pacchetto licenziato sotto MIT e aggiungo delle mie modifice (evidenziate in azzurro) i file originali rimangono sotto MIT, mentre le mie modifiche possono avere la licenza che vogliono. In ogni caso i file originali sotto MIT rimarranno tali.}
\end{figure}

\newpage

\subsubsection{BSD License}

\begin{verbatim}

Copyright (c) 1993 The Regents of the University of California. All
rights reserved.

This software was developed by the Computer Systems Engineering group
at Lawrence Berkeley Laboratory under DARPA contract BG 91-66 and
contributed to Berkeley.

All advertising materials mentioning features or use of this software
must display the following acknowledgement: This product includes
software developed by the University of California, Lawrence Berkeley
Laboratory.

Redistribution and use in source and binary forms, with or without
modification, are permitted provided that the following conditions are
met:

    1. Redistributions of source code must retain the above copyright
    notice, this list of conditions and the following disclaimer.

    2. Redistributions in binary form must reproduce the above copyright
    notice, this list of conditions and the following disclaimer in
    the documentation and/or other materials provided with the
    distribution.

    3. All advertising materials mentioning features or use of this
    software must display the following acknowledgement: This product
    includes software developed by the University of California,
    Berkeley and its contributors.

    4. Neither the name of the University nor the names of its
    contributors may be used to endorse or promote products derived
    from this software without specific prior written permission.

THIS SOFTWARE IS PROVIDED BY THE REGENTS AND CONTRIBUTORS ``AS IS''
AND ANY EXPRESS OR IMPLIED WARRANTIES, INCLUDING, BUT NOT LIMITED TO,
THE IMPLIED WARRANTIES OF MERCHANTABILITY AND FITNESS FOR A PARTICULAR
PURPOSE ARE DISCLAIMED. IN NO EVENT SHALL THE REGENTS OR CONTRIBUTORS
BE LIABLE FOR ANY DIRECT, INDIRECT, INCIDENTAL, SPECIAL, EXEMPLARY, OR
CONSEQUENTIAL DAMAGES (INCLUDING, BUT NOT LIMITED TO, PROCUREMENT OF
SUBSTITUTE GOODS OR SERVICES; LOSS OF USE, DATA, OR PROFITS; OR
BUSINESS INTERRUPTION) HOWEVER CAUSED AND ON ANY THEORY OF LIABILITY,
WHETHER IN CONTRACT, STRICT LIABILITY, OR TORT (INCLUDING NEGLIGENCE
OR OTHERWISE) ARISING IN ANY WAY OUT OF THE USE OF THIS SOFTWARE, EVEN
IF ADVISED OF THE POSSIBILITY OF SUCH DAMAGE.

\end{verbatim}

Licenze libere pi\`u diffuse, che permettono di diffondere un software e di licenziarlo sotto un'altra licenza proprietaria. Esistono 3 varianti principali:
\begin{itemize}
\item BSD a 4 clausole, licenza iniziale di BSD Unix. Era come BSD a 3 clausole, con l'aggiunta di una clausola che diceva che tutti i materiali (pubblicitari o simili) che menzionano l'utilizzo di questo software devono rendere visibile il seguente avviso: ``Questo prodotto include software sviluppato da -nome dell'organizzazione-''
\item BSD a 3 clausole. \`E identica alla BSD a 2 clausole, con l'aggiunta che dice che n\`e il nome del detentore del copyright n\`e il nome dei suoi contributori possono essere usati per promuovere i prodotti derivati da questo software senza un permesso scritto dal possessore iniziale
\item BSD a 2 clausole. Ammette la redistribuzione con o senza modifica rispettando due clausole: la ridistribuzione del codice sorgente deve mantenere le note di copyright e la distribuzione in forma binaria deve riprodurre la nota di copyright, con la lista di condizioni e il disclamer nella documentazione o altri materiali dati insieme al software
\end{itemize}

La licenza BSD a 2 e a 3 clausole sono state approvate dall'OSI, mentre la 4 clausole non \`e stata approvata come Open Source, in quanto non compatibile dalla GPL.

\newpage

\subsubsection{Apache License}

Simile alla licenza BSD, con variazioni importanti. Ha avuto tre revisioni:
\begin{itemize}

\item Versione 1.0: BSD-4clausole con l'aggiunta di una clausola di rinomina
\item Versione 1.1: BSD-3clausole con l'aggiunta di una clausola di rinomina e di una clausola pubblicitaria sulla documentazione
\item Versione 2.0: Si ha un grosso cambiamento della licenza, affrontando una serie di problemi legati ai brevetti, ovvero facendo diventare fondamentale avere delle garanzie: quando viene donata una parte di codice devono essere donati anche i brevetti a esso associati. Ci\`o fa diventare di importanza fondamentale la distinzione tra gli sviluppatori originari e chi invece contribuisce.
\end{itemize}

Qui di sotto viene riportata la versione 2.0 della licenza.

\begin{verbatim}
                                 Apache License
                           Version 2.0, January 2004
                        http://www.apache.org/licenses/

   TERMS AND CONDITIONS FOR USE, REPRODUCTION, AND DISTRIBUTION

   1. Definitions.

      "License" shall mean the terms and conditions for use, reproduction,
      and distribution as defined by Sections 1 through 9 of this document.

      "Licensor" shall mean the copyright owner or entity authorized by
      the copyright owner that is granting the License.

      "Legal Entity" shall mean the union of the acting entity and all
      other entities that control, are controlled by, or are under common
      control with that entity. For the purposes of this definition,
      "control" means (i) the power, direct or indirect, to cause the
      direction or management of such entity, whether by contract or
      otherwise, or (ii) ownership of fifty percent (50%) or more of the
      outstanding shares, or (iii) beneficial ownership of such entity.

      "You" (or "Your") shall mean an individual or Legal Entity
      exercising permissions granted by this License.

      "Source" form shall mean the preferred form for making modifications,
      including but not limited to software source code, documentation
      source, and configuration files.

      "Object" form shall mean any form resulting from mechanical
      transformation or translation of a Source form, including but
      not limited to compiled object code, generated documentation,
      and conversions to other media types.

      "Work" shall mean the work of authorship, whether in Source or
      Object form, made available under the License, as indicated by a
      copyright notice that is included in or attached to the work
      (an example is provided in the Appendix below).

      "Derivative Works" shall mean any work, whether in Source or Object
      form, that is based on (or derived from) the Work and for which the
      editorial revisions, annotations, elaborations, or other modifications
      represent, as a whole, an original work of authorship. For the purposes
      of this License, Derivative Works shall not include works that remain
      separable from, or merely link (or bind by name) to the interfaces of,
      the Work and Derivative Works thereof.

      "Contribution" shall mean any work of authorship, including
      the original version of the Work and any modifications or additions
      to that Work or Derivative Works thereof, that is intentionally
      submitted to Licensor for inclusion in the Work by the copyright owner
      or by an individual or Legal Entity authorized to submit on behalf of
      the copyright owner. For the purposes of this definition, "submitted"
      means any form of electronic, verbal, or written communication sent
      to the Licensor or its representatives, including but not limited to
      communication on electronic mailing lists, source code control systems,
      and issue tracking systems that are managed by, or on behalf of, the
      Licensor for the purpose of discussing and improving the Work, but
      excluding communication that is conspicuously marked or otherwise
      designated in writing by the copyright owner as "Not a Contribution."

      "Contributor" shall mean Licensor and any individual or Legal Entity
      on behalf of whom a Contribution has been received by Licensor and
      subsequently incorporated within the Work.

   2. Grant of Copyright License. Subject to the terms and conditions of
      this License, each Contributor hereby grants to You a perpetual,
      worldwide, non-exclusive, no-charge, royalty-free, irrevocable
      copyright license to reproduce, prepare Derivative Works of,
      publicly display, publicly perform, sublicense, and distribute the
      Work and such Derivative Works in Source or Object form.

   3. Grant of Patent License. Subject to the terms and conditions of
      this License, each Contributor hereby grants to You a perpetual,
      worldwide, non-exclusive, no-charge, royalty-free, irrevocable
      (except as stated in this section) patent license to make, have made,
      use, offer to sell, sell, import, and otherwise transfer the Work,
      where such license applies only to those patent claims licensable
      by such Contributor that are necessarily infringed by their
      Contribution(s) alone or by combination of their Contribution(s)
      with the Work to which such Contribution(s) was submitted. If You
      institute patent litigation against any entity (including a
      cross-claim or counterclaim in a lawsuit) alleging that the Work
      or a Contribution incorporated within the Work constitutes direct
      or contributory patent infringement, then any patent licenses
      granted to You under this License for that Work shall terminate
      as of the date such litigation is filed.

   4. Redistribution. You may reproduce and distribute copies of the
      Work or Derivative Works thereof in any medium, with or without
      modifications, and in Source or Object form, provided that You
      meet the following conditions:

      (a) You must give any other recipients of the Work or
          Derivative Works a copy of this License; and

      (b) You must cause any modified files to carry prominent notices
          stating that You changed the files; and

      (c) You must retain, in the Source form of any Derivative Works
          that You distribute, all copyright, patent, trademark, and
          attribution notices from the Source form of the Work,
          excluding those notices that do not pertain to any part of
          the Derivative Works; and

      (d) If the Work includes a "NOTICE" text file as part of its
          distribution, then any Derivative Works that You distribute must
          include a readable copy of the attribution notices contained
          within such NOTICE file, excluding those notices that do not
          pertain to any part of the Derivative Works, in at least one
          of the following places: within a NOTICE text file distributed
          as part of the Derivative Works; within the Source form or
          documentation, if provided along with the Derivative Works; or,
          within a display generated by the Derivative Works, if and
          wherever such third-party notices normally appear. The contents
          of the NOTICE file are for informational purposes only and
          do not modify the License. You may add Your own attribution
          notices within Derivative Works that You distribute, alongside
          or as an addendum to the NOTICE text from the Work, provided
          that such additional attribution notices cannot be construed
          as modifying the License.

      You may add Your own copyright statement to Your modifications and
      may provide additional or different license terms and conditions
      for use, reproduction, or distribution of Your modifications, or
      for any such Derivative Works as a whole, provided Your use,
      reproduction, and distribution of the Work otherwise complies with
      the conditions stated in this License.

   5. Submission of Contributions. Unless You explicitly state otherwise,
      any Contribution intentionally submitted for inclusion in the Work
      by You to the Licensor shall be under the terms and conditions of
      this License, without any additional terms or conditions.
      Notwithstanding the above, nothing herein shall supersede or modify
      the terms of any separate license agreement you may have executed
      with Licensor regarding such Contributions.

   6. Trademarks. This License does not grant permission to use the trade
      names, trademarks, service marks, or product names of the Licensor,
      except as required for reasonable and customary use in describing the
      origin of the Work and reproducing the content of the NOTICE file.

   7. Disclaimer of Warranty. Unless required by applicable law or
      agreed to in writing, Licensor provides the Work (and each
      Contributor provides its Contributions) on an "AS IS" BASIS,
      WITHOUT WARRANTIES OR CONDITIONS OF ANY KIND, either express or
      implied, including, without limitation, any warranties or conditions
      of TITLE, NON-INFRINGEMENT, MERCHANTABILITY, or FITNESS FOR A
      PARTICULAR PURPOSE. You are solely responsible for determining the
      appropriateness of using or redistributing the Work and assume any
      risks associated with Your exercise of permissions under this License.

   8. Limitation of Liability. In no event and under no legal theory,
      whether in tort (including negligence), contract, or otherwise,
      unless required by applicable law (such as deliberate and grossly
      negligent acts) or agreed to in writing, shall any Contributor be
      liable to You for damages, including any direct, indirect, special,
      incidental, or consequential damages of any character arising as a
      result of this License or out of the use or inability to use the
      Work (including but not limited to damages for loss of goodwill,
      work stoppage, computer failure or malfunction, or any and all
      other commercial damages or losses), even if such Contributor
      has been advised of the possibility of such damages.

   9. Accepting Warranty or Additional Liability. While redistributing
      the Work or Derivative Works thereof, You may choose to offer,
      and charge a fee for, acceptance of support, warranty, indemnity,
      or other liability obligations and/or rights consistent with this
      License. However, in accepting such obligations, You may act only
      on Your own behalf and on Your sole responsibility, not on behalf
      of any other Contributor, and only if You agree to indemnify,
      defend, and hold each Contributor harmless for any liability
      incurred by, or claims asserted against, such Contributor by reason
      of your accepting any such warranty or additional liability.

   END OF TERMS AND CONDITIONS
\end{verbatim}

\newpage

\subsubsection{Academic Free License}

\begin{verbatim}

The Academic Free License
v. 2.1

This Academic Free License (the "License") applies to any original work of
authorship (the "Original Work") whose owner (the "Licensor") has placed the
following notice immediately following the copyright notice for the Original
Work:

Licensed under the Academic Free License version 2.1

1) Grant of Copyright License.  Licensor hereby grants You a world-wide,
   royalty-free, non-exclusive, perpetual, sublicenseable license to do the
   following:

   a) to reproduce the Original Work in copies;

   b) to prepare derivative works ("Derivative Works") based upon the Original
      Work;

   c) to distribute copies of the Original Work and Derivative Works to the 
      public;

   d) to perform the Original Work publicly; and

   e) to display the Original Work publicly.

2) Grant of Patent License.  Licensor hereby grants You a world-wide,
   royalty-free, non-exclusive, perpetual, sublicenseable license, under patent
   claims owned or controlled by the Licensor that are embodied in the Original
   Work as furnished by the Licensor, to make, use, sell and offer for sale the
   Original Work and Derivative Works.

3) Grant of Source Code License.  The term "Source Code" means the preferred
   form of the Original Work for making modifications to it and all available
   documentation describing how to modify the Original Work.  Licensor hereby
   agrees to provide a machine-readable copy of the Source Code of the Original
   Work along with each copy of the Original Work that Licensor distributes.
   Licensor reserves the right to satisfy this obligation by placing a
   machine-readable copy of the Source Code in an information repository 
   reasonably calculated to permit inexpensive and convenient access by You for
   as long as Licensor continues to distribute the Original Work, and by
   publishing the address of that information repository in a notice immediately
   following the copyright notice that applies to the Original Work.

4) Exclusions From License Grant.  Neither the names of Licensor, nor the names
   of any contributors to the Original Work, nor any of their trademarks or 
   service marks, may be used to endorse or promote products derived from this
   Original Work without express prior written permission of the Licensor. 
   Nothing in this License shall be deemed to grant any rights to trademarks,
   copyrights, patents, trade secrets or any other intellectual property of
   Licensor except as expressly stated herein.  No patent license is granted to
   make, use, sell or offer to sell embodiments of any patent claims other than
   the licensed claims defined in Section 2. No right is granted to the
   trademarks of Licensor even if such marks are included in the Original Work. 
   Nothing in this License shall be interpreted to prohibit Licensor from
   licensing under different terms from this License any Original Work that
   Licensor otherwise would have a right to license.

5) This section intentionally omitted.

6) Attribution Rights.  You must retain, in the Source Code of any Derivative
   Works that You create, all copyright, patent or trademark notices from the
   Source Code of the Original Work, as well as any notices of licensing and any
   descriptive text identified therein as an "Attribution Notice."  You must 
   cause the Source Code for any Derivative Works that You create to carry a
   prominent Attribution Notice reasonably calculated to inform recipients that
   You have modified the Original Work.

7) Warranty of Provenance and Disclaimer of Warranty.  Licensor warrants that
   the copyright in and to the Original Work and the patent rights granted 
   herein by Licensor are owned by the Licensor or are sublicensed to You under
   the terms of this License with the permission of the contributor(s) of those
   copyrights and patent rights.  Except as expressly stated in the immediately
   proceeding sentence, the Original Work is provided under this License on an
   "AS IS" BASIS and WITHOUT WARRANTY, either express or implied, including,
   without limitation, the warranties of NON-INFRINGEMENT, MERCHANTABILITY or
   FITNESS FOR A PARTICULAR PURPOSE.  THE ENTIRE RISK AS TO THE QUALITY OF THE
   ORIGINAL WORK IS WITH YOU. This DISCLAIMER OF WARRANTY constitutes an
   essential part of this License.  No license to Original Work is granted
   hereunder except under this disclaimer.

8) Limitation of Liability.  Under no circumstances and under no legal theory,
   whether in tort (including negligence), contract, or otherwise, shall the
   Licensor be liable to any person for any direct, indirect, special, 
   incidental, or consequential damages of any character arising as a result of
   this License or the use of the Original Work including, without limitation,
   damages for loss of goodwill, work stoppage, computer failure or malfunction,
   or any and all other commercial damages or losses.  This limitation of
   liability shall not apply to liability for death or personal injury resulting
   from Licensor's negligence to the extent applicable law prohibits such
   limitation.  Some jurisdictions do not allow the exclusion or limitation of
   incidental or consequential damages, so this exclusion and limitation may not
   apply to You.

9) Acceptance and Termination.  If You distribute copies of the Original Work or
   a Derivative Work, You must make a reasonable effort under the circumstances
   to obtain the express assent of recipients to the terms of this License. 
   Nothing else but this License (or another written agreement between Licensor
   and You) grants You permission to create Derivative Works based upon the
   Original Work or to exercise any of the rights granted in Section 1 herein,
   and any attempt to do so except under the terms of this License (or another
   written agreement between Licensor and You) is expressly prohibited by U.S. 
   copyright law, the equivalent laws of other countries, and by international
   treaty.  Therefore, by exercising any of the rights granted to You in Section
   1 herein, You indicate Your acceptance of this License and all of its terms
   and conditions.

10) Termination for Patent Action.  This License shall terminate automatically
    and You may no longer exercise any of the rights granted to You by this 
    License as of the date You commence an action, including a cross-claim or
    counterclaim, against Licensor or any licensee alleging that the Original
    Work infringes a patent.  This termination provision shall not apply for an
    action alleging patent infringement by combinations of the Original Work
    with other software or hardware.

11) Jurisdiction, Venue and Governing Law.  Any action or suit relating to this
    License may be brought only in the courts of a jurisdiction wherein the
    Licensor resides or in which Licensor conducts its primary business, and
    under the laws of that jurisdiction excluding its conflict-of-law
    provisions.  The application of the United Nations Convention on Contracts
    for the International Sale of Goods is expressly excluded.  Any use of the
    Original Work outside the scope of this License or after its termination
    shall be subject to the requirements and penalties of the U.S.  Copyright
    Act, 17 U.S.C. par. 101 et seq., the equivalent laws of other countries, and
    international treaty.  This section shall survive the termination of this
    License.

12) Attorneys Fees.  In any action to enforce the terms of this License or
    seeking damages relating thereto, the prevailing party shall be entitled to
    recover its costs and expenses, including, without limitation, reasonable
    attorneys' fees and costs incurred in connection with such action, including 
    any appeal of such action.  This section shall survive the termination of 
    this License.

13) Miscellaneous.  This License represents the complete agreement concerning
    the subject matter hereof.  If any provision of this License is held to be
    unenforceable, such provision shall be reformed only to the extent necessary 
    to make it enforceable.

14) Definition of "You" in This License.  "You" throughout this License, whether
    in upper or lower case, means an individual or a legal entity exercising 
    rights under, and complying with all of the terms of, this License.  For 
    legal entities, "You" includes any entity that controls, is controlled by,
    or is under common control with you.  For purposes of this definition,
    "control" means (i) the power, direct or indirect, to cause the direction or
    management of such entity, whether by contract or otherwise, or (ii)
    ownership of fifty percent (50%) or more of the outstanding shares, or (iii)
    beneficial ownership of such entity.

15) Right to Use.  You may use the Original Work in all ways not otherwise
    restricted or conditioned by this License or by law, and Licensor promises
    not to interfere with or be responsible for such uses by You. 

This license is Copyright (C) 2003-2004 Lawrence E. Rosen.  All rights reserved.
Permission is hereby granted to copy and distribute this license without
modification.  This license may not be modified without the express written
permission of its copyright owner.

\end{verbatim}

Licenza non molto utilizzata, si basa su BSD-3clausole con alcune aggiunte:
\begin{itemize}
\item Licenza brevettuale per i brevetti posseduti dal licenziante $\to$ non si ha alcuna garanzia sulla violazione di brevetti
\item Notule di attribuzione
\item Garanzia sulla propriet\`a del software $\to$ garantisce di non violare diritti di terzi
\item Terminazione per citazione su base brevettuale $\to$ simile alla apache license
\item Foro di competenza $\to$ chi cre\`o questa licenza voleva che in caso di una qualsiasi citazione il foro di comptenza fosse il foro in cui viveva l'autore in questo momento
\item Spese legali $\to$ sono previste una suddivisione delle spese legali
\end{itemize}

\subsubsection{Filosofia delle licenze BSD}

Le licenze BSD permettono sfruttamenti proprietari del software che viene spesso creato come prodotto di consorzi che lavorano insieme, che hanno tutto interesse di sviluppare un progetto comune (ad esempio un protocollo), in modo da permetterne la diffusione. Un software proprietario che include il software licenziato sotto BSD ne rafforza la sua posizione, in quanto viene diffuso. Lo scopo delle licenze copyleft \`e costruire una struttura di software libero per poter offrire una base comunitaria.

\newpage

\subsubsection{GPL v2}

\begin{verbatim}
                    GNU GENERAL PUBLIC LICENSE
                       Version 2, June 1991

 Copyright (C) 1989, 1991 Free Software Foundation, Inc.,
 51 Franklin Street, Fifth Floor, Boston, MA 02110-1301 USA
 Everyone is permitted to copy and distribute verbatim copies
 of this license document, but changing it is not allowed.

                            Preamble

  The licenses for most software are designed to take away your
freedom to share and change it.  By contrast, the GNU General Public
License is intended to guarantee your freedom to share and change free
software--to make sure the software is free for all its users.  This
General Public License applies to most of the Free Software
Foundation's software and to any other program whose authors commit to
using it.  (Some other Free Software Foundation software is covered by
the GNU Lesser General Public License instead.)  You can apply it to
your programs, too.

  When we speak of free software, we are referring to freedom, not
price.  Our General Public Licenses are designed to make sure that you
have the freedom to distribute copies of free software (and charge for
this service if you wish), that you receive source code or can get it
if you want it, that you can change the software or use pieces of it
in new free programs; and that you know you can do these things.

  To protect your rights, we need to make restrictions that forbid
anyone to deny you these rights or to ask you to surrender the rights.
These restrictions translate to certain responsibilities for you if you
distribute copies of the software, or if you modify it.

  For example, if you distribute copies of such a program, whether
gratis or for a fee, you must give the recipients all the rights that
you have.  You must make sure that they, too, receive or can get the
source code.  And you must show them these terms so they know their
rights.

  We protect your rights with two steps: (1) copyright the software, and
(2) offer you this license which gives you legal permission to copy,
distribute and/or modify the software.

  Also, for each author's protection and ours, we want to make certain
that everyone understands that there is no warranty for this free
software.  If the software is modified by someone else and passed on, we
want its recipients to know that what they have is not the original, so
that any problems introduced by others will not reflect on the original
authors' reputations.

  Finally, any free program is threatened constantly by software
patents.  We wish to avoid the danger that redistributors of a free
program will individually obtain patent licenses, in effect making the
program proprietary.  To prevent this, we have made it clear that any
patent must be licensed for everyone's free use or not licensed at all.

  The precise terms and conditions for copying, distribution and
modification follow.

                    GNU GENERAL PUBLIC LICENSE
   TERMS AND CONDITIONS FOR COPYING, DISTRIBUTION AND MODIFICATION

  0. This License applies to any program or other work which contains
a notice placed by the copyright holder saying it may be distributed
under the terms of this General Public License.  The "Program", below,
refers to any such program or work, and a "work based on the Program"
means either the Program or any derivative work under copyright law:
that is to say, a work containing the Program or a portion of it,
either verbatim or with modifications and/or translated into another
language.  (Hereinafter, translation is included without limitation in
the term "modification".)  Each licensee is addressed as "you".

Activities other than copying, distribution and modification are not
covered by this License; they are outside its scope.  The act of
running the Program is not restricted, and the output from the Program
is covered only if its contents constitute a work based on the
Program (independent of having been made by running the Program).
Whether that is true depends on what the Program does.

  1. You may copy and distribute verbatim copies of the Program's
source code as you receive it, in any medium, provided that you
conspicuously and appropriately publish on each copy an appropriate
copyright notice and disclaimer of warranty; keep intact all the
notices that refer to this License and to the absence of any warranty;
and give any other recipients of the Program a copy of this License
along with the Program.

You may charge a fee for the physical act of transferring a copy, and
you may at your option offer warranty protection in exchange for a fee.

  2. You may modify your copy or copies of the Program or any portion
of it, thus forming a work based on the Program, and copy and
distribute such modifications or work under the terms of Section 1
above, provided that you also meet all of these conditions:

    a) You must cause the modified files to carry prominent notices
    stating that you changed the files and the date of any change.

    b) You must cause any work that you distribute or publish, that in
    whole or in part contains or is derived from the Program or any
    part thereof, to be licensed as a whole at no charge to all third
    parties under the terms of this License.

    c) If the modified program normally reads commands interactively
    when run, you must cause it, when started running for such
    interactive use in the most ordinary way, to print or display an
    announcement including an appropriate copyright notice and a
    notice that there is no warranty (or else, saying that you provide
    a warranty) and that users may redistribute the program under
    these conditions, and telling the user how to view a copy of this
    License.  (Exception: if the Program itself is interactive but
    does not normally print such an announcement, your work based on
    the Program is not required to print an announcement.)

These requirements apply to the modified work as a whole.  If
identifiable sections of that work are not derived from the Program,
and can be reasonably considered independent and separate works in
themselves, then this License, and its terms, do not apply to those
sections when you distribute them as separate works.  But when you
distribute the same sections as part of a whole which is a work based
on the Program, the distribution of the whole must be on the terms of
this License, whose permissions for other licensees extend to the
entire whole, and thus to each and every part regardless of who wrote it.

Thus, it is not the intent of this section to claim rights or contest
your rights to work written entirely by you; rather, the intent is to
exercise the right to control the distribution of derivative or
collective works based on the Program.

In addition, mere aggregation of another work not based on the Program
with the Program (or with a work based on the Program) on a volume of
a storage or distribution medium does not bring the other work under
the scope of this License.

  3. You may copy and distribute the Program (or a work based on it,
under Section 2) in object code or executable form under the terms of
Sections 1 and 2 above provided that you also do one of the following:

    a) Accompany it with the complete corresponding machine-readable
    source code, which must be distributed under the terms of Sections
    1 and 2 above on a medium customarily used for software interchange; or,

    b) Accompany it with a written offer, valid for at least three
    years, to give any third party, for a charge no more than your
    cost of physically performing source distribution, a complete
    machine-readable copy of the corresponding source code, to be
    distributed under the terms of Sections 1 and 2 above on a medium
    customarily used for software interchange; or,

    c) Accompany it with the information you received as to the offer
    to distribute corresponding source code.  (This alternative is
    allowed only for noncommercial distribution and only if you
    received the program in object code or executable form with such
    an offer, in accord with Subsection b above.)

The source code for a work means the preferred form of the work for
making modifications to it.  For an executable work, complete source
code means all the source code for all modules it contains, plus any
associated interface definition files, plus the scripts used to
control compilation and installation of the executable.  However, as a
special exception, the source code distributed need not include
anything that is normally distributed (in either source or binary
form) with the major components (compiler, kernel, and so on) of the
operating system on which the executable runs, unless that component
itself accompanies the executable.

If distribution of executable or object code is made by offering
access to copy from a designated place, then offering equivalent
access to copy the source code from the same place counts as
distribution of the source code, even though third parties are not
compelled to copy the source along with the object code.

  4. You may not copy, modify, sublicense, or distribute the Program
except as expressly provided under this License.  Any attempt
otherwise to copy, modify, sublicense or distribute the Program is
void, and will automatically terminate your rights under this License.
However, parties who have received copies, or rights, from you under
this License will not have their licenses terminated so long as such
parties remain in full compliance.

  5. You are not required to accept this License, since you have not
signed it.  However, nothing else grants you permission to modify or
distribute the Program or its derivative works.  These actions are
prohibited by law if you do not accept this License.  Therefore, by
modifying or distributing the Program (or any work based on the
Program), you indicate your acceptance of this License to do so, and
all its terms and conditions for copying, distributing or modifying
the Program or works based on it.

  6. Each time you redistribute the Program (or any work based on the
Program), the recipient automatically receives a license from the
original licensor to copy, distribute or modify the Program subject to
these terms and conditions.  You may not impose any further
restrictions on the recipients' exercise of the rights granted herein.
You are not responsible for enforcing compliance by third parties to
this License.

  7. If, as a consequence of a court judgment or allegation of patent
infringement or for any other reason (not limited to patent issues),
conditions are imposed on you (whether by court order, agreement or
otherwise) that contradict the conditions of this License, they do not
excuse you from the conditions of this License.  If you cannot
distribute so as to satisfy simultaneously your obligations under this
License and any other pertinent obligations, then as a consequence you
may not distribute the Program at all.  For example, if a patent
license would not permit royalty-free redistribution of the Program by
all those who receive copies directly or indirectly through you, then
the only way you could satisfy both it and this License would be to
refrain entirely from distribution of the Program.

If any portion of this section is held invalid or unenforceable under
any particular circumstance, the balance of the section is intended to
apply and the section as a whole is intended to apply in other
circumstances.

It is not the purpose of this section to induce you to infringe any
patents or other property right claims or to contest validity of any
such claims; this section has the sole purpose of protecting the
integrity of the free software distribution system, which is
implemented by public license practices.  Many people have made
generous contributions to the wide range of software distributed
through that system in reliance on consistent application of that
system; it is up to the author/donor to decide if he or she is willing
to distribute software through any other system and a licensee cannot
impose that choice.

This section is intended to make thoroughly clear what is believed to
be a consequence of the rest of this License.

  8. If the distribution and/or use of the Program is restricted in
certain countries either by patents or by copyrighted interfaces, the
original copyright holder who places the Program under this License
may add an explicit geographical distribution limitation excluding
those countries, so that distribution is permitted only in or among
countries not thus excluded.  In such case, this License incorporates
the limitation as if written in the body of this License.

  9. The Free Software Foundation may publish revised and/or new versions
of the General Public License from time to time.  Such new versions will
be similar in spirit to the present version, but may differ in detail to
address new problems or concerns.

Each version is given a distinguishing version number.  If the Program
specifies a version number of this License which applies to it and "any
later version", you have the option of following the terms and conditions
either of that version or of any later version published by the Free
Software Foundation.  If the Program does not specify a version number of
this License, you may choose any version ever published by the Free Software
Foundation.

  10. If you wish to incorporate parts of the Program into other free
programs whose distribution conditions are different, write to the author
to ask for permission.  For software which is copyrighted by the Free
Software Foundation, write to the Free Software Foundation; we sometimes
make exceptions for this.  Our decision will be guided by the two goals
of preserving the free status of all derivatives of our free software and
of promoting the sharing and reuse of software generally.

                            NO WARRANTY

  11. BECAUSE THE PROGRAM IS LICENSED FREE OF CHARGE, THERE IS NO WARRANTY
FOR THE PROGRAM, TO THE EXTENT PERMITTED BY APPLICABLE LAW.  EXCEPT WHEN
OTHERWISE STATED IN WRITING THE COPYRIGHT HOLDERS AND/OR OTHER PARTIES
PROVIDE THE PROGRAM "AS IS" WITHOUT WARRANTY OF ANY KIND, EITHER EXPRESSED
OR IMPLIED, INCLUDING, BUT NOT LIMITED TO, THE IMPLIED WARRANTIES OF
MERCHANTABILITY AND FITNESS FOR A PARTICULAR PURPOSE.  THE ENTIRE RISK AS
TO THE QUALITY AND PERFORMANCE OF THE PROGRAM IS WITH YOU.  SHOULD THE
PROGRAM PROVE DEFECTIVE, YOU ASSUME THE COST OF ALL NECESSARY SERVICING,
REPAIR OR CORRECTION.

  12. IN NO EVENT UNLESS REQUIRED BY APPLICABLE LAW OR AGREED TO IN WRITING
WILL ANY COPYRIGHT HOLDER, OR ANY OTHER PARTY WHO MAY MODIFY AND/OR
REDISTRIBUTE THE PROGRAM AS PERMITTED ABOVE, BE LIABLE TO YOU FOR DAMAGES,
INCLUDING ANY GENERAL, SPECIAL, INCIDENTAL OR CONSEQUENTIAL DAMAGES ARISING
OUT OF THE USE OR INABILITY TO USE THE PROGRAM (INCLUDING BUT NOT LIMITED
TO LOSS OF DATA OR DATA BEING RENDERED INACCURATE OR LOSSES SUSTAINED BY
YOU OR THIRD PARTIES OR A FAILURE OF THE PROGRAM TO OPERATE WITH ANY OTHER
PROGRAMS), EVEN IF SUCH HOLDER OR OTHER PARTY HAS BEEN ADVISED OF THE
POSSIBILITY OF SUCH DAMAGES.

                     END OF TERMS AND CONDITIONS
\end{verbatim}

Obbliga i distributori a mantenere il software libero. Il funzionamento della GPL non prevede la firma di alcun contratto e lo si utilizza a patto che si accettino alcune condizioni, che sono:
\begin{itemize}

\item Libera copia non modificata a patto di mantenere le licenze e le clausole di salvaguardia intatte. Questo \textit{permette di poter vendere le copie} e di aggiungerne una garanzia in quanto la GPL normalmente non fornisce alcuna garanzia.
\item Libera distribuzione di versioni modificate con le seguenti clausole:
  \begin{itemize}
  \item Inserimento di avvisi di modifica
  \item Il risultato deve essere sotto GPL
  \item Gli avvisi interattivi sulla licenza devono rimanere
  \item Lavoro ``as a whole'' e non
  \end{itemize}

\end{itemize}

\section{Lezione del 03-11-15}

Si permette una esenzione una componente del sistema alla licenza GPL ma solo se sono generiche.

%\paragraph*{Clausola su programmi interattivi}In caso lanciando un programma interattivo mi venga mostrata la licenza, devo mantenere nelle versioni derivate anche il suo avviso originale di licenza.

La GPL impone delle restrizioni sulla distribuzione di binari, che devono essere accompagnati insieme ai relativi sorgenti. I sorgenti sono il mezzo preferito dal programmatore per produrre quei binari (ovvero il vero sorgente originale, non per esempio quelli passati dai precompilatori). Esistono tre modi di consegnare i sorgenti:
\begin{itemize}

\item Insieme ai binari ci sono anche i sorgenti
\item Vengono consegnati solo i binari ma lo sviluppatore si impegna per tre anni a consegnare i sorgenti
\item La dichiarazione pu\`o essere delegata al sito del produttore originale se e solo se la distribuzione non \`e a scopi commerciali e se in possesso di una dichiarazione scritta

\end{itemize}

Dato che questi tre metodi sarebbero onerosi, se si \`e in internet \`e possibile inserire nello \textit{stesso sito} dove vengono distribuiti i binari anche i sorgenti, rendendoli disponibili.\footnote{Questa nota \`e esplicitata nella licenza per evitare ambiguit\`a}

Ogni volta che viene distribuito un software coperto da GPL oltre ottenere una licenza uso dalla persone/organizzazione che ha consegnato il software si ottiene anche la concessione originale da chi sviluppa quel software, e questo \`e molto importante in quanto in questa maniera non \`e possibile violare i diritti dello sviluppatore originale, garantendogli un certo controllo e mantenendolo comunque il licenzatario, dandogli la possibilit\`a per esempio di intentare direttamente causa contro chi violasse la licenza.

\paragraph*{Clausola di libert\`a o morte (7)}In caso la GPL venga violata per qualsiasi motivo ad esempio per problemi esterni (legali) allora non \`e possibile distribuire interamente i sorgenti, comportando la ``morte'' della licenza. Questo per evitare il diffondersi di software con licenze GPL con determinati casi particolari.

\subsubsection{LGPL}

%inserire termini della licenza

La LGPL \`e progettata per le librerie. Una libreria sotto LGPL che viene modificata deve essere distribuita sotto LGPL a sua volta rendendo disponibile il codice sorgente, ma quando viene creato un lavoro che usa i file della libreria allora bisogna rilasciare il codice oggetto o se il sistema operativo supporta il link dinamico allora \`e possibile rilasciare solamente i binari, perch\`e \`e possibile cambiare la libreria senza restringere la libert\`a degli utenti.
Feature della LGPL:
\begin{itemize}

\item Non \`e possibile rilasciare software che non sia libreria sotto LGPL
\item Una modifica della libreria LGPL \`e possibile ma ogni funzione che fa riferimento a delle tabelle esterne deve avere la possibilit\`a di funzionare anche senza questi elementi
\item Un lavoro licenziato sotto LGPL \`e possibile ``convertirlo'' in GPL
\item \`E possibile usare la libreria in software coperto da altre licenze se e solo se viene distribuito il codice sorgente (della libreria) e vien permesso il reverse engineering, inoltre dev'essere distribuito il codice sorgente originale della libreria insieme ai file oggetto del resto del programma oppure si deve utilizzare un linking dinamico

\end{itemize}


\subsubsection{Filosofia delle licenze copyleft}

Nelle licenze copyleft si ha la massima libert\`a per gli utenti. In questa maniera vien favorito lo \textit{sviluppo comunitario}. La GPL in questo senso fornisce una sorta di ``carta dei diritti'', avendo le garanzia che certe aziende non possano guadagnare in maniera proprietaria da lavoro creato da altre aziende.

\subsubsection{GPL v3}

\begin{verbatim}
                    GNU GENERAL PUBLIC LICENSE
                       Version 3, 29 June 2007

 Copyright (C) 2007 Free Software Foundation, Inc. <http://fsf.org/>
 Everyone is permitted to copy and distribute verbatim copies
 of this license document, but changing it is not allowed.

                            Preamble

  The GNU General Public License is a free, copyleft license for
software and other kinds of works.

  The licenses for most software and other practical works are designed
to take away your freedom to share and change the works.  By contrast,
the GNU General Public License is intended to guarantee your freedom to
share and change all versions of a program--to make sure it remains free
software for all its users.  We, the Free Software Foundation, use the
GNU General Public License for most of our software; it applies also to
any other work released this way by its authors.  You can apply it to
your programs, too.

  When we speak of free software, we are referring to freedom, not
price.  Our General Public Licenses are designed to make sure that you
have the freedom to distribute copies of free software (and charge for
them if you wish), that you receive source code or can get it if you
want it, that you can change the software or use pieces of it in new
free programs, and that you know you can do these things.

  To protect your rights, we need to prevent others from denying you
these rights or asking you to surrender the rights.  Therefore, you have
certain responsibilities if you distribute copies of the software, or if
you modify it: responsibilities to respect the freedom of others.

  For example, if you distribute copies of such a program, whether
gratis or for a fee, you must pass on to the recipients the same
freedoms that you received.  You must make sure that they, too, receive
or can get the source code.  And you must show them these terms so they
know their rights.

  Developers that use the GNU GPL protect your rights with two steps:
(1) assert copyright on the software, and (2) offer you this License
giving you legal permission to copy, distribute and/or modify it.

  For the developers' and authors' protection, the GPL clearly explains
that there is no warranty for this free software.  For both users' and
authors' sake, the GPL requires that modified versions be marked as
changed, so that their problems will not be attributed erroneously to
authors of previous versions.

  Some devices are designed to deny users access to install or run
modified versions of the software inside them, although the manufacturer
can do so.  This is fundamentally incompatible with the aim of
protecting users' freedom to change the software.  The systematic
pattern of such abuse occurs in the area of products for individuals to
use, which is precisely where it is most unacceptable.  Therefore, we
have designed this version of the GPL to prohibit the practice for those
products.  If such problems arise substantially in other domains, we
stand ready to extend this provision to those domains in future versions
of the GPL, as needed to protect the freedom of users.

  Finally, every program is threatened constantly by software patents.
States should not allow patents to restrict development and use of
software on general-purpose computers, but in those that do, we wish to
avoid the special danger that patents applied to a free program could
make it effectively proprietary.  To prevent this, the GPL assures that
patents cannot be used to render the program non-free.

  The precise terms and conditions for copying, distribution and
modification follow.

                       TERMS AND CONDITIONS

  0. Definitions.

  "This License" refers to version 3 of the GNU General Public License.

  "Copyright" also means copyright-like laws that apply to other kinds of
works, such as semiconductor masks.

  "The Program" refers to any copyrightable work licensed under this
License.  Each licensee is addressed as "you".  "Licensees" and
"recipients" may be individuals or organizations.

  To "modify" a work means to copy from or adapt all or part of the work
in a fashion requiring copyright permission, other than the making of an
exact copy.  The resulting work is called a "modified version" of the
earlier work or a work "based on" the earlier work.

  A "covered work" means either the unmodified Program or a work based
on the Program.

  To "propagate" a work means to do anything with it that, without
permission, would make you directly or secondarily liable for
infringement under applicable copyright law, except executing it on a
computer or modifying a private copy.  Propagation includes copying,
distribution (with or without modification), making available to the
public, and in some countries other activities as well.

  To "convey" a work means any kind of propagation that enables other
parties to make or receive copies.  Mere interaction with a user through
a computer network, with no transfer of a copy, is not conveying.

  An interactive user interface displays "Appropriate Legal Notices"
to the extent that it includes a convenient and prominently visible
feature that (1) displays an appropriate copyright notice, and (2)
tells the user that there is no warranty for the work (except to the
extent that warranties are provided), that licensees may convey the
work under this License, and how to view a copy of this License.  If
the interface presents a list of user commands or options, such as a
menu, a prominent item in the list meets this criterion.

  1. Source Code.

  The "source code" for a work means the preferred form of the work
for making modifications to it.  "Object code" means any non-source
form of a work.

  A "Standard Interface" means an interface that either is an official
standard defined by a recognized standards body, or, in the case of
interfaces specified for a particular programming language, one that
is widely used among developers working in that language.

  The "System Libraries" of an executable work include anything, other
than the work as a whole, that (a) is included in the normal form of
packaging a Major Component, but which is not part of that Major
Component, and (b) serves only to enable use of the work with that
Major Component, or to implement a Standard Interface for which an
implementation is available to the public in source code form.  A
"Major Component", in this context, means a major essential component
(kernel, window system, and so on) of the specific operating system
(if any) on which the executable work runs, or a compiler used to
produce the work, or an object code interpreter used to run it.

  The "Corresponding Source" for a work in object code form means all
the source code needed to generate, install, and (for an executable
work) run the object code and to modify the work, including scripts to
control those activities.  However, it does not include the work's
System Libraries, or general-purpose tools or generally available free
programs which are used unmodified in performing those activities but
which are not part of the work.  For example, Corresponding Source
includes interface definition files associated with source files for
the work, and the source code for shared libraries and dynamically
linked subprograms that the work is specifically designed to require,
such as by intimate data communication or control flow between those
subprograms and other parts of the work.

  The Corresponding Source need not include anything that users
can regenerate automatically from other parts of the Corresponding
Source.

  The Corresponding Source for a work in source code form is that
same work.

  2. Basic Permissions.

  All rights granted under this License are granted for the term of
copyright on the Program, and are irrevocable provided the stated
conditions are met.  This License explicitly affirms your unlimited
permission to run the unmodified Program.  The output from running a
covered work is covered by this License only if the output, given its
content, constitutes a covered work.  This License acknowledges your
rights of fair use or other equivalent, as provided by copyright law.

  You may make, run and propagate covered works that you do not
convey, without conditions so long as your license otherwise remains
in force.  You may convey covered works to others for the sole purpose
of having them make modifications exclusively for you, or provide you
with facilities for running those works, provided that you comply with
the terms of this License in conveying all material for which you do
not control copyright.  Those thus making or running the covered works
for you must do so exclusively on your behalf, under your direction
and control, on terms that prohibit them from making any copies of
your copyrighted material outside their relationship with you.

  Conveying under any other circumstances is permitted solely under
the conditions stated below.  Sublicensing is not allowed; section 10
makes it unnecessary.

  3. Protecting Users' Legal Rights From Anti-Circumvention Law.

  No covered work shall be deemed part of an effective technological
measure under any applicable law fulfilling obligations under article
11 of the WIPO copyright treaty adopted on 20 December 1996, or
similar laws prohibiting or restricting circumvention of such
measures.

  When you convey a covered work, you waive any legal power to forbid
circumvention of technological measures to the extent such circumvention
is effected by exercising rights under this License with respect to
the covered work, and you disclaim any intention to limit operation or
modification of the work as a means of enforcing, against the work's
users, your or third parties' legal rights to forbid circumvention of
technological measures.

  4. Conveying Verbatim Copies.

  You may convey verbatim copies of the Program's source code as you
receive it, in any medium, provided that you conspicuously and
appropriately publish on each copy an appropriate copyright notice;
keep intact all notices stating that this License and any
non-permissive terms added in accord with section 7 apply to the code;
keep intact all notices of the absence of any warranty; and give all
recipients a copy of this License along with the Program.

  You may charge any price or no price for each copy that you convey,
and you may offer support or warranty protection for a fee.

  5. Conveying Modified Source Versions.

  You may convey a work based on the Program, or the modifications to
produce it from the Program, in the form of source code under the
terms of section 4, provided that you also meet all of these conditions:

    a) The work must carry prominent notices stating that you modified
    it, and giving a relevant date.

    b) The work must carry prominent notices stating that it is
    released under this License and any conditions added under section
    7.  This requirement modifies the requirement in section 4 to
    "keep intact all notices".

    c) You must license the entire work, as a whole, under this
    License to anyone who comes into possession of a copy.  This
    License will therefore apply, along with any applicable section 7
    additional terms, to the whole of the work, and all its parts,
    regardless of how they are packaged.  This License gives no
    permission to license the work in any other way, but it does not
    invalidate such permission if you have separately received it.

    d) If the work has interactive user interfaces, each must display
    Appropriate Legal Notices; however, if the Program has interactive
    interfaces that do not display Appropriate Legal Notices, your
    work need not make them do so.

  A compilation of a covered work with other separate and independent
works, which are not by their nature extensions of the covered work,
and which are not combined with it such as to form a larger program,
in or on a volume of a storage or distribution medium, is called an
"aggregate" if the compilation and its resulting copyright are not
used to limit the access or legal rights of the compilation's users
beyond what the individual works permit.  Inclusion of a covered work
in an aggregate does not cause this License to apply to the other
parts of the aggregate.

  6. Conveying Non-Source Forms.

  You may convey a covered work in object code form under the terms
of sections 4 and 5, provided that you also convey the
machine-readable Corresponding Source under the terms of this License,
in one of these ways:

    a) Convey the object code in, or embodied in, a physical product
    (including a physical distribution medium), accompanied by the
    Corresponding Source fixed on a durable physical medium
    customarily used for software interchange.

    b) Convey the object code in, or embodied in, a physical product
    (including a physical distribution medium), accompanied by a
    written offer, valid for at least three years and valid for as
    long as you offer spare parts or customer support for that product
    model, to give anyone who possesses the object code either (1) a
    copy of the Corresponding Source for all the software in the
    product that is covered by this License, on a durable physical
    medium customarily used for software interchange, for a price no
    more than your reasonable cost of physically performing this
    conveying of source, or (2) access to copy the
    Corresponding Source from a network server at no charge.

    c) Convey individual copies of the object code with a copy of the
    written offer to provide the Corresponding Source.  This
    alternative is allowed only occasionally and noncommercially, and
    only if you received the object code with such an offer, in accord
    with subsection 6b.

    d) Convey the object code by offering access from a designated
    place (gratis or for a charge), and offer equivalent access to the
    Corresponding Source in the same way through the same place at no
    further charge.  You need not require recipients to copy the
    Corresponding Source along with the object code.  If the place to
    copy the object code is a network server, the Corresponding Source
    may be on a different server (operated by you or a third party)
    that supports equivalent copying facilities, provided you maintain
    clear directions next to the object code saying where to find the
    Corresponding Source.  Regardless of what server hosts the
    Corresponding Source, you remain obligated to ensure that it is
    available for as long as needed to satisfy these requirements.

    e) Convey the object code using peer-to-peer transmission, provided
    you inform other peers where the object code and Corresponding
    Source of the work are being offered to the general public at no
    charge under subsection 6d.

  A separable portion of the object code, whose source code is excluded
from the Corresponding Source as a System Library, need not be
included in conveying the object code work.

  A "User Product" is either (1) a "consumer product", which means any
tangible personal property which is normally used for personal, family,
or household purposes, or (2) anything designed or sold for incorporation
into a dwelling.  In determining whether a product is a consumer product,
doubtful cases shall be resolved in favor of coverage.  For a particular
product received by a particular user, "normally used" refers to a
typical or common use of that class of product, regardless of the status
of the particular user or of the way in which the particular user
actually uses, or expects or is expected to use, the product.  A product
is a consumer product regardless of whether the product has substantial
commercial, industrial or non-consumer uses, unless such uses represent
the only significant mode of use of the product.

  "Installation Information" for a User Product means any methods,
procedures, authorization keys, or other information required to install
and execute modified versions of a covered work in that User Product from
a modified version of its Corresponding Source.  The information must
suffice to ensure that the continued functioning of the modified object
code is in no case prevented or interfered with solely because
modification has been made.

  If you convey an object code work under this section in, or with, or
specifically for use in, a User Product, and the conveying occurs as
part of a transaction in which the right of possession and use of the
User Product is transferred to the recipient in perpetuity or for a
fixed term (regardless of how the transaction is characterized), the
Corresponding Source conveyed under this section must be accompanied
by the Installation Information.  But this requirement does not apply
if neither you nor any third party retains the ability to install
modified object code on the User Product (for example, the work has
been installed in ROM).

  The requirement to provide Installation Information does not include a
requirement to continue to provide support service, warranty, or updates
for a work that has been modified or installed by the recipient, or for
the User Product in which it has been modified or installed.  Access to a
network may be denied when the modification itself materially and
adversely affects the operation of the network or violates the rules and
protocols for communication across the network.

  Corresponding Source conveyed, and Installation Information provided,
in accord with this section must be in a format that is publicly
documented (and with an implementation available to the public in
source code form), and must require no special password or key for
unpacking, reading or copying.

  7. Additional Terms.

  "Additional permissions" are terms that supplement the terms of this
License by making exceptions from one or more of its conditions.
Additional permissions that are applicable to the entire Program shall
be treated as though they were included in this License, to the extent
that they are valid under applicable law.  If additional permissions
apply only to part of the Program, that part may be used separately
under those permissions, but the entire Program remains governed by
this License without regard to the additional permissions.

  When you convey a copy of a covered work, you may at your option
remove any additional permissions from that copy, or from any part of
it.  (Additional permissions may be written to require their own
removal in certain cases when you modify the work.)  You may place
additional permissions on material, added by you to a covered work,
for which you have or can give appropriate copyright permission.

  Notwithstanding any other provision of this License, for material you
add to a covered work, you may (if authorized by the copyright holders of
that material) supplement the terms of this License with terms:

    a) Disclaiming warranty or limiting liability differently from the
    terms of sections 15 and 16 of this License; or

    b) Requiring preservation of specified reasonable legal notices or
    author attributions in that material or in the Appropriate Legal
    Notices displayed by works containing it; or

    c) Prohibiting misrepresentation of the origin of that material, or
    requiring that modified versions of such material be marked in
    reasonable ways as different from the original version; or

    d) Limiting the use for publicity purposes of names of licensors or
    authors of the material; or

    e) Declining to grant rights under trademark law for use of some
    trade names, trademarks, or service marks; or

    f) Requiring indemnification of licensors and authors of that
    material by anyone who conveys the material (or modified versions of
    it) with contractual assumptions of liability to the recipient, for
    any liability that these contractual assumptions directly impose on
    those licensors and authors.

  All other non-permissive additional terms are considered "further
restrictions" within the meaning of section 10.  If the Program as you
received it, or any part of it, contains a notice stating that it is
governed by this License along with a term that is a further
restriction, you may remove that term.  If a license document contains
a further restriction but permits relicensing or conveying under this
License, you may add to a covered work material governed by the terms
of that license document, provided that the further restriction does
not survive such relicensing or conveying.

  If you add terms to a covered work in accord with this section, you
must place, in the relevant source files, a statement of the
additional terms that apply to those files, or a notice indicating
where to find the applicable terms.

  Additional terms, permissive or non-permissive, may be stated in the
form of a separately written license, or stated as exceptions;
the above requirements apply either way.

  8. Termination.

  You may not propagate or modify a covered work except as expressly
provided under this License.  Any attempt otherwise to propagate or
modify it is void, and will automatically terminate your rights under
this License (including any patent licenses granted under the third
paragraph of section 11).

  However, if you cease all violation of this License, then your
license from a particular copyright holder is reinstated (a)
provisionally, unless and until the copyright holder explicitly and
finally terminates your license, and (b) permanently, if the copyright
holder fails to notify you of the violation by some reasonable means
prior to 60 days after the cessation.

  Moreover, your license from a particular copyright holder is
reinstated permanently if the copyright holder notifies you of the
violation by some reasonable means, this is the first time you have
received notice of violation of this License (for any work) from that
copyright holder, and you cure the violation prior to 30 days after
your receipt of the notice.

  Termination of your rights under this section does not terminate the
licenses of parties who have received copies or rights from you under
this License.  If your rights have been terminated and not permanently
reinstated, you do not qualify to receive new licenses for the same
material under section 10.

  9. Acceptance Not Required for Having Copies.

  You are not required to accept this License in order to receive or
run a copy of the Program.  Ancillary propagation of a covered work
occurring solely as a consequence of using peer-to-peer transmission
to receive a copy likewise does not require acceptance.  However,
nothing other than this License grants you permission to propagate or
modify any covered work.  These actions infringe copyright if you do
not accept this License.  Therefore, by modifying or propagating a
covered work, you indicate your acceptance of this License to do so.

  10. Automatic Licensing of Downstream Recipients.

  Each time you convey a covered work, the recipient automatically
receives a license from the original licensors, to run, modify and
propagate that work, subject to this License.  You are not responsible
for enforcing compliance by third parties with this License.

  An "entity transaction" is a transaction transferring control of an
organization, or substantially all assets of one, or subdividing an
organization, or merging organizations.  If propagation of a covered
work results from an entity transaction, each party to that
transaction who receives a copy of the work also receives whatever
licenses to the work the party's predecessor in interest had or could
give under the previous paragraph, plus a right to possession of the
Corresponding Source of the work from the predecessor in interest, if
the predecessor has it or can get it with reasonable efforts.

  You may not impose any further restrictions on the exercise of the
rights granted or affirmed under this License.  For example, you may
not impose a license fee, royalty, or other charge for exercise of
rights granted under this License, and you may not initiate litigation
(including a cross-claim or counterclaim in a lawsuit) alleging that
any patent claim is infringed by making, using, selling, offering for
sale, or importing the Program or any portion of it.

  11. Patents.

  A "contributor" is a copyright holder who authorizes use under this
License of the Program or a work on which the Program is based.  The
work thus licensed is called the contributor's "contributor version".

  A contributor's "essential patent claims" are all patent claims
owned or controlled by the contributor, whether already acquired or
hereafter acquired, that would be infringed by some manner, permitted
by this License, of making, using, or selling its contributor version,
but do not include claims that would be infringed only as a
consequence of further modification of the contributor version.  For
purposes of this definition, "control" includes the right to grant
patent sublicenses in a manner consistent with the requirements of
this License.

  Each contributor grants you a non-exclusive, worldwide, royalty-free
patent license under the contributor's essential patent claims, to
make, use, sell, offer for sale, import and otherwise run, modify and
propagate the contents of its contributor version.

  In the following three paragraphs, a "patent license" is any express
agreement or commitment, however denominated, not to enforce a patent
(such as an express permission to practice a patent or covenant not to
sue for patent infringement).  To "grant" such a patent license to a
party means to make such an agreement or commitment not to enforce a
patent against the party.

  If you convey a covered work, knowingly relying on a patent license,
and the Corresponding Source of the work is not available for anyone
to copy, free of charge and under the terms of this License, through a
publicly available network server or other readily accessible means,
then you must either (1) cause the Corresponding Source to be so
available, or (2) arrange to deprive yourself of the benefit of the
patent license for this particular work, or (3) arrange, in a manner
consistent with the requirements of this License, to extend the patent
license to downstream recipients.  "Knowingly relying" means you have
actual knowledge that, but for the patent license, your conveying the
covered work in a country, or your recipient's use of the covered work
in a country, would infringe one or more identifiable patents in that
country that you have reason to believe are valid.

  If, pursuant to or in connection with a single transaction or
arrangement, you convey, or propagate by procuring conveyance of, a
covered work, and grant a patent license to some of the parties
receiving the covered work authorizing them to use, propagate, modify
or convey a specific copy of the covered work, then the patent license
you grant is automatically extended to all recipients of the covered
work and works based on it.

  A patent license is "discriminatory" if it does not include within
the scope of its coverage, prohibits the exercise of, or is
conditioned on the non-exercise of one or more of the rights that are
specifically granted under this License.  You may not convey a covered
work if you are a party to an arrangement with a third party that is
in the business of distributing software, under which you make payment
to the third party based on the extent of your activity of conveying
the work, and under which the third party grants, to any of the
parties who would receive the covered work from you, a discriminatory
patent license (a) in connection with copies of the covered work
conveyed by you (or copies made from those copies), or (b) primarily
for and in connection with specific products or compilations that
contain the covered work, unless you entered into that arrangement,
or that patent license was granted, prior to 28 March 2007.

  Nothing in this License shall be construed as excluding or limiting
any implied license or other defenses to infringement that may
otherwise be available to you under applicable patent law.

  12. No Surrender of Others' Freedom.

  If conditions are imposed on you (whether by court order, agreement or
otherwise) that contradict the conditions of this License, they do not
excuse you from the conditions of this License.  If you cannot convey a
covered work so as to satisfy simultaneously your obligations under this
License and any other pertinent obligations, then as a consequence you may
not convey it at all.  For example, if you agree to terms that obligate you
to collect a royalty for further conveying from those to whom you convey
the Program, the only way you could satisfy both those terms and this
License would be to refrain entirely from conveying the Program.

  13. Use with the GNU Affero General Public License.

  Notwithstanding any other provision of this License, you have
permission to link or combine any covered work with a work licensed
under version 3 of the GNU Affero General Public License into a single
combined work, and to convey the resulting work.  The terms of this
License will continue to apply to the part which is the covered work,
but the special requirements of the GNU Affero General Public License,
section 13, concerning interaction through a network will apply to the
combination as such.

  14. Revised Versions of this License.

  The Free Software Foundation may publish revised and/or new versions of
the GNU General Public License from time to time.  Such new versions will
be similar in spirit to the present version, but may differ in detail to
address new problems or concerns.

  Each version is given a distinguishing version number.  If the
Program specifies that a certain numbered version of the GNU General
Public License "or any later version" applies to it, you have the
option of following the terms and conditions either of that numbered
version or of any later version published by the Free Software
Foundation.  If the Program does not specify a version number of the
GNU General Public License, you may choose any version ever published
by the Free Software Foundation.

  If the Program specifies that a proxy can decide which future
versions of the GNU General Public License can be used, that proxy's
public statement of acceptance of a version permanently authorizes you
to choose that version for the Program.

  Later license versions may give you additional or different
permissions.  However, no additional obligations are imposed on any
author or copyright holder as a result of your choosing to follow a
later version.

  15. Disclaimer of Warranty.

  THERE IS NO WARRANTY FOR THE PROGRAM, TO THE EXTENT PERMITTED BY
APPLICABLE LAW.  EXCEPT WHEN OTHERWISE STATED IN WRITING THE COPYRIGHT
HOLDERS AND/OR OTHER PARTIES PROVIDE THE PROGRAM "AS IS" WITHOUT WARRANTY
OF ANY KIND, EITHER EXPRESSED OR IMPLIED, INCLUDING, BUT NOT LIMITED TO,
THE IMPLIED WARRANTIES OF MERCHANTABILITY AND FITNESS FOR A PARTICULAR
PURPOSE.  THE ENTIRE RISK AS TO THE QUALITY AND PERFORMANCE OF THE PROGRAM
IS WITH YOU.  SHOULD THE PROGRAM PROVE DEFECTIVE, YOU ASSUME THE COST OF
ALL NECESSARY SERVICING, REPAIR OR CORRECTION.

  16. Limitation of Liability.

  IN NO EVENT UNLESS REQUIRED BY APPLICABLE LAW OR AGREED TO IN WRITING
WILL ANY COPYRIGHT HOLDER, OR ANY OTHER PARTY WHO MODIFIES AND/OR CONVEYS
THE PROGRAM AS PERMITTED ABOVE, BE LIABLE TO YOU FOR DAMAGES, INCLUDING ANY
GENERAL, SPECIAL, INCIDENTAL OR CONSEQUENTIAL DAMAGES ARISING OUT OF THE
USE OR INABILITY TO USE THE PROGRAM (INCLUDING BUT NOT LIMITED TO LOSS OF
DATA OR DATA BEING RENDERED INACCURATE OR LOSSES SUSTAINED BY YOU OR THIRD
PARTIES OR A FAILURE OF THE PROGRAM TO OPERATE WITH ANY OTHER PROGRAMS),
EVEN IF SUCH HOLDER OR OTHER PARTY HAS BEEN ADVISED OF THE POSSIBILITY OF
SUCH DAMAGES.

  17. Interpretation of Sections 15 and 16.

  If the disclaimer of warranty and limitation of liability provided
above cannot be given local legal effect according to their terms,
reviewing courts shall apply local law that most closely approximates
an absolute waiver of all civil liability in connection with the
Program, unless a warranty or assumption of liability accompanies a
copy of the Program in return for a fee.

                     END OF TERMS AND CONDITIONS

\end{verbatim}

La GPL v3 presenta profonde differenze con la GPL v2, affrontando tutta una serie di problemi non previsti dalla GPL v2, causando anche scontento dalla parte della comunit\`a. Al momento della creazione della GPL v2 non c'erano tutta una serie di problemi, che son presenti oggigiorno:
\begin{itemize}

\item Tivoizzazione: codice GPL viene usato in macchine o dispositivi che effettuano controllo sul firmware, limitando le libert\`a di modifica
\item DRM e Digital Millenium Copyright Act/European Union Copyright Directive che affermano che rimuovere dispositivi di anti-copia \`e illegale, creando un paradosso per cui \`e possibile usare sistemi di anti-copia su software con licenza GPL, limitando le libert\`a di fatto
\item Brevetti software
\item Patto Microsoft/Novell: la Microsoft ha dato una licenza alla Novell per cui aveva il diritto di distribuire sistemi GNU/Linux senza violare i brevetti della Microsoft.

\end{itemize}

La GPL v3\footnote{Non \`e compatibile con GPL v2, non \`e possibile creare un prodotto con licenza mista tra GPL v2 e v3} affronta anche nuovi punti che la GPL v2 non affronta, rinnovandosi anche nel lato legale.

\paragraph*{Brevetti software}In Europa viene negato l'utilizzo di brevetti sul software, tra cui le teorie matematiche e i programmi informatici.
Successivamente \`e stato previsto che un software possa essere brevettato all'interno di un prodotto pi\`u completo (ad esempio il sistema software di raffreddamento all'interno di una trivella), causando lo sfruttamento di questa norma e aggirando il vincolo sui brevetti software.
In questa maniera inoltre \`e pi\`u difficile identificare quando un software viola un brevetto.

Nel caso del software libero \`e possibile rilasciare il codice libero ma protetto da brevetti, quindi come soluzione a questo approccio si ha che rilasciando il codice sotto GPL v3 si rilasciano i brevetti alla versione attuale che si possiede.

\paragraph*{Tivoizzazione}Essendo un fenomeno crescente, la GPL v3 sancisce che chi distribuisce apparecchi dotati di software licenziato sotto la GNU GPL v3 deve includere nella distribuzione del codice sorgente tutte le informazioni necessarie - ivi incluse, per esempio, le chiavi digitali necessarie - per compilare una versione del codice modificato pienamente funzionante sull'hardware distribuito. Queste condizioni vengono applicate solamente nei prodotti di tipo consumer.

\paragraph*{DMCA/EUCD}La GPL v3 afferma che rilasciando il software sotto questa licenza si prende atto di non considerare le proprie modifiche protezioni digitali. Questo per impedire usi non adeguati della GPL. \newline


La GPL v3 presenta una migliore internazionalizzazione, svincolandosi dalla legislazione americana e adottando un linguaggio neutrale e pi\`u compatibile rispetto ai diversi sistemi giuridici. Inoltre regola i lavori su commissione e rende possibile la distribuzione via internet del codice sorgente.
La GPL v3 diventa compatibile con la licenza AGPL v3, inizialmente non creata dalla FSF. Inoltre \`e disponibile il download tramite peer2peer separando il concetto di propagazione\footnote{Posso dare il software in una maniera tale che sia possibile effettuare una violazione del copyright} con quello di distribuzione\footnote{Tutti i modi con il quale si pu\`o distribuire i sorgenti con i binari}, con la possibilit\`a di aggiungere permessi aggiuntivi al proprio codice, e di aggiungere un insieme limitato di restrizioni aggiuntive. Ci\`o rende altre licenze compatibili con la GPL v3, come l'Apache 2 license e la Xfree86 license.


\section{Lezione del 10-11-15}

\subsubsection{Affero GPL v3}

Creata dalla Affero inc. per regolare i diritti sui servizi con codice coperto da GPL v3. La Affero GPL v3 \`e stato adottata dalla GNU.

\subsubsection{Mozilla Public License}

Originariamente nata dalla Mozilla per il rilascio del codice di Netscape, essa voleva rendere possibile l'aggiunta di software proprietario ``di contorno'' (come per esempio dei plugin), e quindi fu creata una licenza (inizialmente chiamata NPL) MPL che fu una via di mezzo tra la GPL iniziale e una BSD.

Con la Mozilla Public License \`e possibile avere una integrazione con le licenze proprietarie. Questo in quanto inizialmente voleva essere protetto il codice originario di Netscape. Con la Mozilla Public License viene consegnata una licenza brevettuale per i brevetti che sono necessari per l'utilizzo di quella versione del software.

\subsubsection{Perl Artistic License}

La Perl Artistic License non \`e pensata per diffondere, come nel caso della GPL, il software libero, ma \`e pensata per permettere all'autore del proprio software un controllo ``artistico'' sulla sua opera.

In questa licenza viene privilegiato il copyright holder. Il software in s\`e non necessita alcun pagamento, e il destinatario ha la possibilit\`a di distribuire il software agli stessi termini.

\paragraph*{Redistribuzione dei sorgenti}Il codice \`e redistribuibile mantenendo la nota di licenza originale, ma \`e possibile incorporare codice che sia o sotto dominio pubblico o sia scritto dallo stesso autore che detiene il copyright. Per la redistribuzione dei sorgenti modificabili \`e possibile rendere nel dominio pubblico le modifiche, o renderle disponibili gratuitamente. \`E possibile distribuire localmente il codice (per esempio per una propria azienda) con la licenza che si desidera. Se invece si vuole distribuire al pubblico, bisogna allora rinominare tutti i file modificati, in modo che sia chiaro che quella versione \`e stata modificata e non sia possibile confondersi con la versione originale.

\paragraph*{Redistribuzione dei binari}\`E possibile redistribuire direttamente i binari, a patto che siano disponibili anche i sorgenti originari con delle istruzioni su come prelevare la versione standard. Altres\`i \`e possibile distribuire insieme ai binari modificati anche i sorgenti modificati, oppure \`e necessario fare la rinominazione degli eseguibili e documentare le differenze.

\paragraph*{Uso commerciale}Non \`e possibile usare questa licenza per software con scopi commerciali, in quanto la vendita del software \`e vietata. \`E permesso ricevere un pagamento per il supporto e \`e possibile l'integrazione in superpacchetti commerciali. L'uso del software deve essere isolato dall'utente.

\subsection{Creative Commons}

Testi di riferimenti: \textit{Viral Spiral, How the Commoners Built a Digital Republic of Their Own - David Bollier} e \textit{Creative Commons: a user guide - Simone Aliprandi}

\subsubsection{GNU Free Documentation License}

Nata come una licenza sulla documentazione del software, in quanto problema molto sentito e non trattato dalla GPL (alcune tematiche riguardo alla trattazione dei documenti non vengono considerate).

Obiettivi della GNU Free Documentation License:
\begin{itemize}

\item Libert\`a di modifica
\item Tutela dei diritti morali dell'autore $\to$ per gli autori \`e possibile che ci siano parti importanti da non modificare, quindi \`e stata introdotta la possibilit\`a di inserire alcune sezioni non varianti, che possono essere tutte quelle sezioni che non sono l'argomento principale dell'opera.
\item Problema gestione delle copie non trasparenti $\to$ viene regolato il modo in cui viene gestita la redistribuzione per grandi quantit\`a 
\item Copia in grande quantit\`a e non

\end{itemize} 

La licenza suddivide il documento in varie parti:
\begin{itemize}
  
\item Documento
\item Titolo
\item Sezioni secondarie e invarianti
\item Testi di copertina
\item Storia del documento $\to$ changelog all'interno dell'opera
\item Licenza
  
\end{itemize}

La GNU FDL impone delle restrizioni sulle redistribuzioni senza modifica:
\begin{itemize}

\item Mantenimento della licenza
\item No misure tecnologiche di restrizione
\item Esibizioni pubbliche permesse
\item Redistribuzioni voluminose:
  \begin{itemize}

  \item Obbligo di identificarsi come editore
  \item Obbligo di mantenere titolo e testi di copertina
  \item Obbligo di distribuire il sorgente

  \end{itemize}

\end{itemize}

Se invece vengono applicate delle modifiche \`e necessario:
\begin{itemize}

\item Modifica del titolo
\item Indicazione degli autori delle modifiche e del documento
\item Rimozione dei ``Riconoscimenti''
\item Preservazione della sezione delle dediche
\item Aggiornamento della sezione cronologia
\item Preservazione degli invarianti\footnote{Anche le traduzioni vengono considerate delle modifiche}
\item Preservazione della versione trasparente
\item Preservazione e/o aggiunta dei testi di copertina

\end{itemize}

\`E presente una clausola speciale, creata per Wikipedia, che sotto specifiche condizioni permettevano di convertire la licenza sotto CC-BY-SA (la licenza Creative Commons pi\`u vicina alla FDL). Questa clausola, temporanea, \`e ora scaduta.

Nei documenti ha senso trattare i vari tipi di unioni (a differenza della GPL), ed \`e possibile mantenere solo una licenza nel documento.
Nella traduzione dei documenti \`e possibile tradurre anche la licenza, ma \`e necessario mantenerne una versione in inglese.

La GNU FDL \`e un ``ponte'' tra le licenze software e le licenze sui documenti.

\subsubsection{Transizione alle Creative Commons}

Creata da un insieme di persone e dopo molti anni.

\paragraph*{Background}Negli anni '60 si ha un aumento pervasivo del copyright, fino al culmine degli anni '80 quando il giornale Nation prese un estratto dal libro del presidente Ford, causando una causa legale vinta da Ford. Negli anni successivi ci furono i cambiamenti dei mezzi d'informazione, coinvolgendo nella copia anche le persone, causando una maggiore sensibilizzazione del copyright al grande pubblico.

\paragraph*{Inizio della crisi}Nel 1990 si ha la nascita di EFF, associazione molto attiva che si contrapponeva alle stringenti regole del copyright dettate dalle lobby. Si ha anche la nascita della National Information Infrastructure creata da parte di Clinton che cercava di portare ordine su Internet, in cui vennero proposte tutte una serie di modifiche restrittive nei confronti del pubblico. Questo caus\`o una forte reazione, con la pubblicazione di un articolo nel 1994 dal The Economy of Ideas di Barlow, accumunando molte persone. Tutto ci\`o sfoci\`o nel 1995 nella creazione della Digital Future Coalition, che si contrapponeva al DMCA.
Nel 1996 si ha la Dichiarazione d'indipendenza del cyberspazio, in cui viene dichiarata l'internazionalit\`a dei contenuti del web e dove viene affermato che gli stati non dovrebbero averne influenza.
Nel 1999 si ha la conferenza ``Private censorship/perfect choice''.
Alla fine degli anni novanta si valorizza il dominio pubblico da parte di queste comunit\`a createsi, che considerano i lavori di dominio pubblico una base per la nascita di futuri progetti.

\section{Lezione del 24-11-15}

Nel 1998 viene approvato il Sonny Bono Copyright Extension Act\footnote{Detto anche Mickey Mouse Protection Act}, che come sostenitore aveva la Disney, dove l'estensione del copyright veniva estesa di ulteriori 20 anni. Questa estensione colp\`i gli interessi di tantissime opere, come il caso di Eldricht Press di Eric Eldred che ospitava un sito web di libri con il copyiright scaduto.

%\subsection{Lessing}

\paragraph*{Lessig}Politico in Erba derivante da una famiglia repubblicana, dopo il college a Cambridge divent\`o liberale e si interess\`o di Giurisprudenza, incominciandosi a interessare dell'effetto Eldred, dove una stessa legge era stata interpretata in maniera diversa in base a pressioni sociali. Si interess\`o dell'impatto dell'architettura di internet sulla legge, dove siti grandi visitati da molti utenti finivano per ottenere un controllo sugli utenti.
Negli ultimi anni si assiste a una fusione tra il mondo reale e il mondo cybernetico.
Con il patto Lessig-Eldred si ha la denuncia dell'incostituzionalit\`a del SBCEA. Si ebbe l'apertura del sito \textit{openlaw}, dove persone avevano la possibilit\`a di confrontarsi e trasmettere dei valori sul modo di vedere su ci\`o che doveva essere il modo di vedere la diffusione dell'informazione. La causa fu persa 7-2 alla corte suprema nel 2003.

\subsection{Creative Commons}

Creata dall'idea di Eldred: creare una copyright conservancy. Dopo la fine della causa con Microsoft Eldred comincia a farlo fin dall'inizio dei principi, in quanto bisogna vedere l'apertura del copyright per tutte le opere, in quanto le case produttrici dopo pochi anni smettevano di vendere le opere in quanto non c'era pi\`u un vero guadagno. Era inoltre necessario abbandonare i principi sul fair use\footnote{Utilizzo dell'opere senza vincoli. Sono presenti dei paletti (\`e possibile per una piccola citazione che non sia contro l'autore stesso etc..).}. Era necessario creare un movimento tecnico-legale, senza esserne a conoscenza si veniva a creare l'inizio della \textit{Creative Commons}. Gi\`a nel 1999 era nato il \textit{Copyright Commons}, una newsletter mensile sul caso Eldred vs Reno. Da qui si ha l'idea per la creazione della compagna Counter Copyright (CC), per dare la possibilit\`a di creare opere e distribuirle anche sotto Public Domain, anche se non era possibile in quanto ogni opera \`e sempre sotto un dominio.

Nel 1999 si ha anche la nascita di Napster, un sito in cui scaricare musica illegale che fece moltissimo successo. Lessig capisce che c'\`e moltissima gente che ha bisogno di accedere ai contenuti liberamente. Da qui nasce nel 2000 dall'idea di Habelson: una fondazione che accetti donazioni di opere, anche se scartata in quanto troppa problematica. Da ci\`o si ha nel 2001 la fondazione della Creative Commons\footnote{Il nome deriva dalla tragedia di Garret Hardin ``tragedia dei commons''.}, dove era possibile adottare una licenza comune per tutte le opere che volevano essere diffuse liberamente. Gli obiettivi del progetto sono:
\begin{itemize}

\item Promuovere la diffusione di un modello ``some rights reserved''
\item Tutelare il marchio ``Creative Commons''
\item Creazione di  appropriati strumenti legali e tecnologici

\end{itemize}
Le creative commons si pongono a met\`a tra il pubblico dominio e tra il copyright assoluto.

Creative Commons non \`e un ente pubblico, n\`e un organismo per la raccolta dei diritti d'autore e non offre servizi di consulenza legale.

\paragraph*{Licenze CC}Le licenze CC sono pensate per opere creative non software. Sono disponibili per 52 sistemi giuridici, e hanno come caratteristiche comuni sulla preservazione delle note di licenza, sul permesso di copia e distribuzione dell'opera e sull'esibizione dell'opera e l'uso di restrizioni tecnologiche vietato.
\`E possibile inserire un digital code che segna la licenza che permette di facilitare la ricerca nei motori come Google, rendendo la loro visibili\`a pi\`u rapida. Ogni licenza \`e identificata dal suo commons deed:
\begin{itemize}
  
\item Attribuzione
\item No opere derivate
\item Non commerciale
\item Condividi allo stesso modo

\end{itemize}

Tutte le combinazioni formano le licenze Creative Commons. \`E da notare che ``Attribuzione'' \`e presente in tutte le licenze e che non esiste ``No-Derivate'' e ``Share-alike'' nella stessa licenza in quanto non avrebbe molto senso.
Le licenze possibili sono:
\begin{itemize}

\item Attribution
\item Attribution-SA $\to$ per certi versi la pi\`u simile alla GNU GPL.
\item Attribution-ND $\to$ non \`e permesso creare opere derivate, ma \`e sempre possibile la distribuzione (sempre sotto la stessa licenza)
\item Attribution-NC $\to$ non sar\`a mai possibile la ridistribuzione a scopo commerciale. \`E anche possibile cambiare licenza basta che rimanga la clausola NC.
\item Attribution-NC-SA $\to$ \`e necessario riconoscere l'autore, non \`e possibile utilizzarla per scopi commerciali e \`e necessario condividerla allo stesso modo
\item Attribution-NC-ND $\to$ non \`e possibile usarla per scopi commerciali senza alcuna modifica. Questo si usa per opere che vogliono avere un'ampia diffusione e veniva utilizzato recentemente per i temi Wordpress.

\end{itemize}

\subsubsection{Obblighi delle licenze CC}

Una raccolta di opere \`e da considerarsi come un lavoro collettivo. La distribuzione della licenza \`e possibile attraverso un URL. Per eseguire una citazione su un'opera protetta da CC \`e necessario specificare:
\begin{itemize}

\item Autore originario
\item Titolo dell'opera originaria
\item Se possibile URL associato all'opera
\item Uso dell'opera
\item Rimozione citazione su richiesta del licenziate\footnote{\`E necessario specificare: il licenziante \`e colui che licenzia l'opera, l'autore \`e colui che la scrive. }

\end{itemize}

Con l'attribution SA si aveva la garanzia del proprietario dei diritti di avere i diritti necessari e che l'opera rispetta i diritti d'autore, di marchio di fabbrica e altro. Questa clausola \`e stata rimossa dalla licenza 2.0 in poi.

\subsubsection{Strumenti riconoscimento licenze CC}
\begin{itemize}

\item CC plus: protocollo per esprimere licenze alternative, viene usato anche dai motori di ricerca per trovare documenti con specifiche licenze.
\item CC0: alternativa al public domain, che permette di scegliere in modo pi\`u vicino possibile in base alla giurisdizione del paese di rilasciare la vostra opera.
\item Founders' copyright: possibilit\`a di mantenere il copyright per 14 anni rinnovabili una volta. Questa funzionalit\`a \`e stata rimossa.

\end{itemize}

\subsection{RDF e RDFa}

Materiale di riferimento: \textit{Pratical RDF di Shelley Powers, RDF primer, RDFA primer, RDFA syntax al www.w3.org}

\paragraph*{Definizioni}RDF \`e un linguaggio per rappresentare informazioni semantiche sul web.
RDFa \`e un modo per esprimere le informazioni semantiche in RDF all'interno di una pagina web. \`E un'estensione xhtml.
Ccrel: vocabolario per RDFa per esprimere asserzioni sulla licenza di contenuti web.

\subsubsection{RDF}
L'obiettivo dell'RDF \`e di costruire un linguaggio per esprimere contenuti semantici in maniera semplice, estendibile attraverso moduli e interpretabile da un calcolatore.
La soluzione adottata \`e attraverso grafi: questo perch\`e pi\`u facile da elaborare da parte di un calcolatore, rimanendo comunque estendibi. Un grafo \`e formato da degli statement, che sono creati da soggetto, predicato, oggetto. \`E importante disambiguare in maniera semplice i vari significati delle parole, e questo viene attraverso degli URI o un nodo anonimo o letterale. La visualizzazione di queste opzioni avviene rispettivamente con un ovale contentente l'URI, cerchi vuoti, rettangoli contententi la rispettiva stringa.

\paragraph*{Struttura dell'URI}Le URI non rappresentano sempre dei siti web, e in genere sono composti da:
\begin{itemize}

\item Schema (http, ftp, uuid)
\item Authority (opzionale)
\item Cammino assoluto o relativo
\item Query e identificatore di frammento

\end{itemize}

L'URI esprime solo nomi e non localizzazioni.

\paragraph*{Sintassi delle URI}La sintassi generica di un URI \`e la seguente:
\begin{verbatim}

schema:localpart [?query][#frammento]

\end{verbatim}

Dove lo schema pu\`o essere http, ftp, tel...
Per indentificare l'authority dev'essere presente la doppia barra ``//''.
Ad esempio:
\begin{verbatim}

http://www.debian.org/index.html

\end{verbatim}

Se si scrive http:/index.html allora si indica la pagina index dentro l'authority www.debian.org.
Per esempio il numero di telefono non hanno un authority, come anche per la voce mailto.

\paragraph*{Utilit\`a degli URI}Gli URI permettono di creare degli identificativi unici all'interno del web, creando uno schema forse difficile da leggere ma preciso.

\paragraph*{XML Qualified Name}Un qualified name \`e identificato da un URI. \`E da notare che gli escape forniscono qualified name diversi. I Qname in RDF dipendono dalla codifica adottata. Un modo abbastanza comune \`e specificare RDF con xmlns.
Esistono alcuni namespace standard:
\begin{itemize}

\item rdf
\item rdfs
\item dc
\item ex
\item foaf

\end{itemize}

Visitando prefix.cc \`e possibile accedere alle risorse che puntano alle varie risorse (rdf).

\paragraph*{Nodi anonimi}
Con i nodi anonimi \`e possibile creare nodi senza un URI associato, ed \`e inoltre possibile inserire letterali. I letterali di un grafo RDF possono avere un \textbf{tipo associato}, esso non \`e un tipo obbligatorio. I tipi in RDF sono definiti in XML Schema, ma non tutti i tipi di XML Schema sono associabili a un letterale, ma solo a un tipo primitivo. \newline

L'insieme di RDF/XML risulta essere molto verboso e scarsamente collegato al web.

\paragraph*{Notazione N-Triples}

Ogni riga pu\`o essere un commento (che inizia con ``\#'') o una tripla del tipo: subject predicate object.
I nodi anonimi si indicano con '\_:name'.
Questo approccio presenta gli stessi problemi di replicazione di RDF/XML, ma presenta una sintassi molto semplice.

\section{Lezione del 01-12-15}
\graphicspath{ {res/data/01-12-15/} }

\paragraph*{Collezioni RDF}

I container RDF non sono ``bloccabili''. Quindi, se necessario c'\`e il bisogno dell'uso delle collezioni. Le collezioni vengono viste come liste singolarmente linkate, e si accede tramite le seguenti voci: \textit{rdf:first, rdf:rest, rdf:nil}

\paragraph*{Reification}

I reification permettono di esprimere predicati che parlano ad altri predicati. Per esprimere ci\`o esistono dei predicati standard da applicare ai nodi anonimi:
\begin{itemize}

\item rdf:Statement
\item rdf:subject
\item rdf:predicate
\item rdf:object

\end{itemize}

\subsubsection{RDFa}

Il linguaggio RDFa \`e un linguaggio di annotazione rdf per XHTML, compatibile con i namespace.

\paragraph*{Flusso RDFa}Inizialmente il soggetto corrente \`e il documento corrente, avendo esso stesso un url. Esegue una scansione depth-first (lavorando in maniera ricorsiva), cercando un tag che specifica il predicato. Se \`e presente un oggetto e un soggetto, ottiene una tripla completa, aggiungendolo al grafo. Il soggetto iniziale \`e sempre il tag ``base''. \newline

RDFa supporta una \href{http.//www.w3.org/2011/rdfa-context/rdfa-1}{lista di prefissi iniziali predefiniti}.

\newpage

\paragraph*{Property}Il tag property setta l'oggetto di una propriet\`a. Ad esempio il codice:
\begin{verbatim}

<p>Il contenuto di questo sito e' licenziato sotto
<a property="cc:license" 
href="http://creativecommons.org/licenses/by/3.0/">CC-BY</a>
</p>
\end{verbatim}

Viene schematizzata nella seguente maniera:

\begin{figure}[h]
  \centering
  \includegraphics[scale=0.6]{01-12-15-01}
\end{figure}

\paragraph*{Content}Specifica il contenuto da usarsi a fini RDFa quando il contenuto del tag corrente \textbf{non \`e desiderabile} o \textbf{non \` disponibile}. Ci\`o avviene quando si vuole sostituire il contenuto non appropriato con un alternativo per qualche ragione (es. data americana con data europea).

\paragraph*{Vocab}Setta un vocabolare di default per tutti i discendenti. Pu\`o essere mescolato con full uris, e pu\`o essere settato su diversi livelli dell'albero xhtml. Utile soprattutto quando si hanno molte propriet\`a dello stesso vocabolario.

\paragraph*{Prefix}Permette di utilizzare velocemente pi\`u di un prefisso. Questo sistema permette di essere mescolato con vocab.
Ad esempio:
\begin{verbatim}
<body prefix="p1: http://www.p1.org/ p2: http://www.p2.org">
<span property="p1:property">properieta' 1</span>
[...]
</body>
\end{verbatim}

\paragraph*{About}L'attributo about permette di settare il soggetto, che diventer\`a il soggetto corrente all'interno del tag
Es:
\begin{verbatim}
<div about="/barbecue">
[...]
</div>
\end{verbatim}

Mescolando abount e property si ha il setting contemporaneo di soggetto e predicato. \`E importante notare che l'ordine con cui gli attributi vengono inseriti \`e importante. Ad esempio:
\begin{verbatim}
<p>frammento di codice rilasciato sotto
<span about="#code" property="license">
cc-by
</span>
</p>
\end{verbatim}

L'about permette non solo di settare il soggetto, ma di settare il soggetto in modo anonimo. Ad esempio:
\begin{verbatim}

<link about="[_:n1]" rel="foaf:mbox" href="mailto:john@ex.org" />
\end{verbatim}

\paragraph*{Typeof}
Typeof pu\`o essere usato in due situazioni:
\begin{itemize}
  
\item Settare un tipo per un soggetto $\to$ funziona solamente inserendo anche l'attributo about
\item Creare un nodo anonimo usato come base per i predicati $\to$ esempio:
\begin{verbatim}
  <div typeof="foaf:Person">
    <p propery="foaf:name">Alice</p>
    <p>
      Email
      <a rel="foaf:mbox" href="mailto:alice@example.com">
        alice@example.com</a>
    </p>
    <p>Phone: <a rel="foaf:phone"
      href="tel:0444123456">0444123456</a></p>
  </div>

\end{verbatim}

\end{itemize}

\paragraph*{Rel}L'attributo rel collega l'oggetto all'attributo resource, href o src o ai nodi contenuti. Rel e Property presentano delle differenze tra loro:
\begin{itemize}

\item Il Rel non si collega al contenuto interno del tag e nemmeno al valore di content
\item Rel supporta il \textit{chaining}
  
\end{itemize}

Il chaining \`e uno strumento molto potente, che pu\`o concaterane oggetti esterni e soggetti interni tra loro. Ad esempio:
\begin{verbatim}
<a about="http://www.debian.org" rel="dc:creator"
href="http://www.ian.org">
<span property="foaf:name">Ian Murdock</span>
</a>
\end{verbatim}

Come si pu\`o vedere, lo span di foaf:name \`e si lega a www.ian.org e a Ian Murdock: ovvero l'oggetto della proposizione esterna diventa soggetto di quella interna.

\`E anche possibile che il soggetto della proposizione interna diventi oggetto di quella esterna. Ad esempio:
\begin{verbatim}
<div about="#me" rel="foaf:knows">
  <div about="http://www.w3.org/People/Ivan/#me"
   property="foaf:name" content="Ivan Herman" />
</div>
\end{verbatim}

\begin{figure}[h]
  \centering
  \includegraphics[scale=0.5]{01-12-15-02}
  \caption{Schema dell'esempio sopra riportato}
\end{figure}

\`E anche possibile il \textit{chaining con intermediario anonimo}. Se non viene specificato n\'e un oggetto per la proposizione esterna n\'e un soggetto per quella interna, il collegament \`e fornito da un nodo anonimo creato automaticamente.

\paragraph*{Datatype}Datatype pu\`e venire usato per esprimere il tipo di un letterale.
Per esempio \`e possibile dichiarare l'et\`a di un sito.

\subsubsection{RDF Schema}
Descrive il vocabolario usabili in RDF. Si dividono in vari elementi.

\paragraph*{Classi}Collezione di elementi, dette istanza collegate agli elementi attraverso rdf:type. \`E possibile creare gerarchie gerarchiche. Si ddice che una classe A sia sottoclasse di B se ogni istanza di A \`e anche istanza di B.

\paragraph*{Propriet\`a}Connessioni tra i vari elementi. \`E possibile definire delle \textbf{sottoproriet\`a}, in modo che se \`e presente una propriet\`a allora vale anche un'altra. Inoltre \`e possibile definire \textbf{range} e \textbf{domain}.

\section{Lezione del 15-12-15}
\graphicspath{ {res/data/15-12-15/} }


\paragraph*{Dublin Core}Questo viene importato con il prefisso ``dc'', e presenta le principali voci:
\begin{itemize}
\item[Contenuto]
  \begin{itemize}
    
  \item Title: Titolo dell'opera
  \item Subject: Topic dell'opera
  \item Description: Abstract
  \item Source: Opera originaria
    
  \end{itemize}

\item[Propriet\`a Intellettuale]
  \begin{itemize}

  \item Creator: autore
  \item Contributor: ulteriore contributore
  \item Publisher: per chi \`e responsabile per la pubblicazione della risorsa $\to$ per una collana pu\`o essere il rispettivo pubblicatore
    \item Rights: un link alla licenza
    
  \end{itemize}
  
\end{itemize}

I predicati double core possono essere usati per specificare la licenza, ma pu\`o essere possibile usare un altra vocabolario pi\`u specifico.

\paragraph*{CCrel} \`E un vocabolario creato per parlare delle licenze creato dai membri della creative commons. Permette di specificare le propriet\`a del lavoro e della licenza.
Voci:
\begin{itemize}

\item cc:attributionName
\item cc:attributionUrl $\to$ permette di citare direttamente la licenza di tutto il documento, semplificando il suo ritrovamento
\item dc:source
\item cc:morePermission $\to$ permette dare un link per permettere di acquistare una licenza pi\`u permissiva
  
\end{itemize}

Le altre voci vengono riprese da altri vocabolari, come ad esempio Double Core o xhtml.

\textbf{Dc:type} permette di specificare il tipo di oggetto. Alcuni possibili valori possono essere:
\begin{itemize}

\item dcmitype:Text
\item dcmitype:Sound
\item dcmitype:StillImage
\item dcmitype:MovingImage
  
\end{itemize}

Un esempio \`e lo schema creato dal ``choose license'' di Creative Commons

\begin{figure}[h]
  \centering
  \includegraphics[scale=0.4]{15-12-15-02}
\end{figure}

\subsection{Subversion}

Altre risorse esterne: \textit{Version Control with Subversion} - Rilasciato sotto licenza CC all'indirizzo \url{http://svnbook.red-bean.com}

Subversion presenta un approccio completamente diverso rispetto ad altri sistemi di versionamento come Git o Mercurial: \`e presente un server centrale, e i client possono accedere, leggere o pushare delle modifiche. Ogni modifica del repository viene creata una nuova revisione. Subversion permette di avere non solamente uno snapshot del sistema in quel momento ma anche una shadow copy del lavoro: ovvero \`e possibile avere una visione ``ibrida'' tra due revisioni. \`E possibile avere pi\`u rami, ed \`e possibile eseguire il merge dei rami.
Subversion non si occupa di eseguire alcun tipo di analisi semantica, ma lavora dal punto di vista puramente testuale.
Termini importanti:
\begin{itemize}

\item Checkout e working copy
\item Export $\to$ \`e possibile eseguire una copia del file senza i file di subversion
\item Commit $\to$ la modifica viene inviata sul server e si ottiene una nuova revisione
\item Update $\to$ vengono presi gli ultimi aggiornamenti dal repository
\item Revisioni $\to$ memorizza l'ultima modifica del file nel repository, e permette di mantenere le versioni. Funzionamento delle revisioni:
  \begin{itemize}
    
  \item Per repository
  \item Collegate al singolo file
  \item Tag e branches $\to$ i tag permettono di dare un nome alla ultima revisione. I rami (branch) sono delle revisioni multiple che hanno una vita propria, e possono essere di sviluppo, di prova o anche stabili. Nei rami \`e possibile continuare a lavorare, mentre nei tag no.
    
  \end{itemize}

\end{itemize}

In Subversion nulla vieta di avere pi\`u progetti in un unico repository. All'interno di ogni cartella si \`e liberi avere la propria gerarchia di cartelle. \`E importante notare che un repository non contiene una copia diretta dei progetti ospitati. Il repository \`e possibile accederci via molteplici interfacce: http, https, svn, svn+ssh, file.

\subsubsection{SVN vs Git}

\begin{table}[H]
\centering
\begin{tabular}{|C{3.5cm}|C{4cm}|C{4cm}|}
\hline
& \textbf{SVN} & \textbf{Git} \\
\hline
Controllo di versione & Centrale & Distribuito \\
\hline
Repository & Repository centrale in cui vengono create le copie di lavoro & Copie locali del repository su cui poter lavorare \\
\hline
Permesso di accesso & Basato sul percorso & Per tutta la directory \\
\hline
Visualizzazione delle modifiche & Registra i file & Registra i contenuti \\
\hline
Cronologia delle modifiche & Completa solo nel repository, le copie di lavoro contengono solo la versione più recente & Repository e copie di lavoro contengono la cronologia completa \\
\hline
Connessione di rete & Ad ogni accesso & Necessaria solo per la sincronizzazione \\
\hline
\end{tabular}
\end{table}
Dovreste preferire Git se…
\begin{itemize}
\item …non volete essere sempre connessi per poter lavorare ovunque al vostro progetto. 
\item …volete sentirvi sicuri nel caso di un guasto o di una perdita di dati del repository centrale.
\item …non avete bisogno di un permesso di scrittura e di lettura per directory speciali (tuttavia queste possono essere impostate su un percorso più complesso anche con Git).
\item …date maggiore importanza a una trasmissione veloce delle modifiche.
\end{itemize}
Subversion è la scelta migliore se…
\begin{itemize}
\item …avete bisogno di autorizzazioni d’accesso basate sul percorso per ambiti diversi del vostro progetto.
\item …volete legare tutto il vostro lavoro a un luogo centrale.
\item …lavorate con molti file binari.
\item …volete registrare completamente le strutture delle directory vuote (Git le elimina, poiché non hanno alcun contenuto).
\end{itemize}

\subsubsection{Problema dell'Locking}Il problema del Locking si verifica quando due utenti eseguono le modifiche al server contemporaneamente. Si vuole evitare che il lavoro di un utente non venga sovrascritto dal lavoro di un altro. Esistono due approcci principali:
\begin{itemize}

\item Alla apertura di un file il repository blocca l'accesso a quel file, impedendo gli altri utenti di editare quello specifico file. Questo sistema porta ad una serie di problemi: il lock pu\`o essere lungo in modo arbitrario, viene creata una serializzazione non necessaria e si ha una falsa sicurezza.
\item Locking ottimistico: permette a pi\`u utenti di effettuare le modifiche allo stesso file contemporaneamente. Quando accadono i conflitti devono essere risolti dai programmatori, e poi le modifiche devono essere caricate sul server, mantenendo una versione consistente (merge manuale).
  
\end{itemize}

\subsubsection{Revisioni e file}
Subversion memorizza per ogni file: la revisioni del repository su cui \`e basato, un timestamp di quando \`e stato aggiornato da repository l'ultima volta, e i file originali prelevati dal repository.

Durante il lavoro un file pu\`o essere in uno stato di:
\begin{itemize}

\item Inalterato localmente e aggiornato
\item Alterato localmente e aggiornato
\item Inalterato localmente e non aggiornato
\item Alterato localmente e non aggiornato $\to$ si ottiene l'errore ``out of date'' ed \`e necessario eseguire un merge manuale
  
\end{itemize}

\paragraph*{Esempio pratico}Per creare un repository:
\begin{verbatim}
svnadmin create repo #repo e' il nome del repository
\end{verbatim}
Dopo la creazione di un repo, per inserire i dati basta semplicemente creare i file all'esterno della cartella di svn. Per inserirli nel repository \`e necessario creare una certa struttura standard. Per importare i dati si usa la keyword \underline{import}, in cui bisogna specificare la cartella e il repository in cui bisogna aggiungere questi file. \`E necessario specificare lo schema (protocollo), l'indirizzo e il path.
\begin{verbatim}
svn import proj/ file:///home/cesco/tmp/repo
\end{verbatim}
A questo punto si otterr\`a un errore, in quanto \`e necessario specificare un editor per committare i messaggi. Con l'opzione ``-m'' \`e possibile inserire direttamente un messaggio
\begin{verbatim}
svn import proj/ file:///home/cesco/tmp/repo -m "Import Iniziale"
\end{verbatim}
Per visualizzare i file all'interno del repo \`e possibile dare:
\begin{verbatim}
svn ls file:///home/cesco/tmp/repo
\end{verbatim}
\`E importante notare che non comparr\`a la cartella, in quanto vengono aggiunti solamente i file all'interno della cartella ``proj'' specificata prima.

Per importare un repository si esegue un \textit{checkout}, ovvero:
\begin{verbatim}
svn co file:///home/cesco/tmp/repo proj
#proj e' la cartella dove verranno inseriti i file
\end{verbatim}
A questo punto viene prelevato dal repository i file e vengono inseriti in una cartella con lo stesso nome.

Per controllare eventuali file modificati \`e necessario scrivere:
\begin{verbatim}
svn status #oppure svn st
\end{verbatim}

\subsubsection{Strutture e configurazioni}

\`E presente un file di configurazione (svnserve.conf) che permette di gestire le varie configurazioni, come per le password (passwd) e il login (authz). db \`e la cartella del repository, e hooks \`e la cartella dove possono essere inseriti degli script agganciabili agli eventi. \`E presente un README e un file format.



\end{document}
