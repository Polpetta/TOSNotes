\documentclass[11pt,a4paper,openany]{article}

%\usepackage[latin1]{inputenc}

\usepackage{hyperref}
\usepackage[utf8]{inputenc}
\usepackage[italian]{babel}
\usepackage{lmodern}
\hypersetup{
  colorlinks,
  citecolor=gray,
  filecolor=red,
  linkcolor=blue,
  urlcolor=blue
}

\usepackage{amsmath}
\usepackage{graphicx}
\usepackage{float}
\usepackage{amsfonts}
\usepackage{listings}
\lstset{
	commentstyle=\color{green},
	frame=single,
	keepspaces=true,
	keywordstyle=\color{blue},
	numbers=left,
	numberstyle=\tiny\color{black},
	rulecolor=\color{black},
        basicstyle=\ttfamily
}

\renewcommand\arraystretch{1.25}
\usepackage{array}
\usepackage{ragged2e}
\newcolumntype{C}[1]{>{\centering\let\newline\\\arraybackslash\hspace{0pt}}m{#1}}


\title{\textbf{Appunti di Tecnologie Open Source}}
\author{Polonio Davide}

%\date{1/10/2015}

\begin{document}

\maketitle

\tableofcontents
%da includere altre parti degli appunti (una pagina a giornata?)

\newpage

\section{Lezione 06-10-15}

\subsection{Introduzione al software libero}

Definizione: Il software libero\footnote{Libero \`e diverso da gratuito.} \`e software che garantisce le seguenti quattro libert\`a fondamentali:

\begin{enumerate}

\item Eseguire il programma per qualsiasi scopo: un programma libero non pu\`o mai imporre:

  \begin{itemize}

  \item Restrizioni in termini di tempo
  \item Restrizioni in termini di scopo
  \item Limitazioni di area geografica $\to$ questo pu\`o essere un caso speciale
    
  \end{itemize}

\item Studiare come funziona il programma e adattarlo alle proprie necessit\`a. Nessua modifica sulla modifica e comprensione del tipo:

  \begin{itemize}

  \item Richiedere l'acquisto di licenze speciali
  \item Richiedere le firma di NDA
  \item Impedire l'accesso al codice sorgente
    
  \end{itemize}

\item Redistribuire copie in modo da aiutare il prossimo. Il software libero non proibisce di prestare la propria copia ad una persona o darle una copia, \underline{nemmeno dietro un pagamento di un compenso}

  \item Migliorare il programma e distribuire pubblicamente i miglioramenti, in modo tale che tutta la comunit\`a ne tragga beneficio. Infatti migliorare il programma e distribuire i miglioramenti permette a chi non ha il tempo o le capacita' per risolvere un problema di accedere indirettamente alla liber\`a di modifica $\to$ anche questo pu\`o essere dietro compenso.
  
\end{enumerate}

\subsubsection{Importanza del software libero}

\begin{itemize}

\item Riduzione dei costi

\item Trasparenze $\to$ soprattutto per gli enti statali

\item Nessun lock-in (si veda il caso di XFree86)

\item Sicurezza e affidabilit\`a\footnote{Nota: da un punto di vista pratico la qualit\`a del software libero tende a essere pari con il miglior software proprietario. Il problema \`e che \`e possibile che siano presenti bug di sicurezza non noti agli sviluppatori che possono essere sfruttati a scopo malevolo, senza che venga segnalato alla comunit\`a}

\end{itemize}

Il software libero, diversamente da quello proprietario, \`e un'infrastruttura e cambia il modo in cui si fa impresa (si veda Openerp, CUPS, Android). Quello che si sta capendo \`e che le molte menti creative senza lavorare per una certa azienda pu\`o dare comunque una mano a quel determinato software, oltre che, dall'iterazione con gli utenti si hanno nuove idee, perch\`e viene gestito il progetto insieme.

\subsection{Albori del diritto d'autore}

\subsubsection{Venezia e i ``Privilegi''}

Nella Venezia del 1500 c.a. erano presenti i Privilegi. I privilegi si riferivano alla tecnologia utilizzata. I privilegi erano garantiti ai stampatori, ma non agli autori (rigurardo la pubblicazione e alla stampa di documenti-libri)

Nel 1517-1537 si va verso il privilegio d'autore. Si afferma infatti il ruolo dei privilegi sui libri comuni, con una limitazione della durata dei privilegi fino a 10 anni. Tutto ci\`o porta alla richieste di opere originali, con lo spostamente delle protezione delle opere: si cominciano a progettere anche le modifiche\footnote{Inizialmente le opere modificate-rielaborate e ripubblicate erano considerate diverse dall'opera iniziale, e non una sua ``derivata''}

Le cooperazioni veneziane attive dal tredicesimo secolo, adibite alle professioni artigiane proteggono scrupolosamente le conoscenze artigiane:
\begin{itemize}

\item Restrizioni sui movimenti degli artigiani
\item Conoscenze trasmesse esclusivamente per via orale
\item Meccanismi per limitare la crescita delle singole attivit\`a

\end{itemize}

Con tutti i controlli nasce l'idea del diritto immateriale. Tutto il riserbo sulle conoscenze gli fanno acquisire importanza. Nel 1474 si hanno i privilegi d'autore (di cui accennato prima) a Venezia, nel tentativo di portare ordine:

\begin{itemize}

\item Non creava percorsi obbligati, si limitava a codificare le politiche relative ai privilegi

\item Mirata principalmente agli inventori e non alle corporazioni

\end{itemize}

Con l'avvento dell'umanismo si ha un aumento di persone che tenta di ``scappare'' dalle corporazioni e ottenere dei provilegi: si ha una maggiore attenzione alla necessit\`a di proteggere il proprio lavoro. Molti dotti cominciano a preoccuparsi a problemi pratici, con un assottigliamento dei confini tra conoscienza e capacit\`a meccaniche. Da qui vengono scritti i primi libri in cui si cerca di avere una maggiore importanza alla protezione delle conoscienze, e comincia a cambiare il modo in cui esse vengono trasmesse.


\subsubsection{Inghilterra e Copyright}

Con la prima stamperia nel 1476 (cio\`e in ritardo) i problemi di copia non autorizzate hanno una ridotta importanza. Con l'introduzione della censura da parte della Corona, e nel 1557 viene dato il controllo esclusivo sulla stampa dei libri all Stationers Company, posto sotto il controllo della Camera Stellata. Agendo in tale maniera gli autovi vengono estromessi dai propri lavori.

Nel 1640 si ha l'abolizione della Camera Stellata e delle limitazioni sulla stampa. Data l'esplosione dei giornali e delle libera stampa si ha che dal 1643-1694 viene instaurata di nuovo la censura da parte della Corona. Cosi\`i facendo, dal 1695-1704 si hanno 13 tentativi di restaurare dei controlli censori. nel 1740 con l'editto di Ann si ha lo scopo del Copyright visto come stimolazione della cultura, proteggendo le creazioni intangili, e non pi\`u le copie come prima.
la licenza passa agli autori e non alle stamperie con una protezione di 14 anni, con la richiesta di estensione di 14 anni dopo la scadenza.

\subsubsection{Colonie Inglesi e USA}

In America nel 1638 reverendo Glover porta una macchina da stampa. Le autorit\`a del Massachusetts (dove in particolare si trovava la macchina) contribuiscono alla sua manutenzione. Nel 1672 viene dato il primo privilegio di stampa al signor Usher di stampare le leggi della colonia.

Il primo diritto d'autore viene concesso nel 1781: Andrew Law scrive una collezione di melodie, e pe bloccare la competizione di altre collezioni simili chiede un monopolio.

Nel 1783 una petizione di John Cedgard cambia la situzione. Dopo aver scrutti la sua stesura chiede il privilegio di stampa e dopo averglielo concesso, viene creaa una Connecticut copyright statue, dando una legislazione generale (per quella regione). Sette anni dopo, nel 1790 si ha il Copyright Act, che da le prime regolamentazioni sul Copyright.

Con la diffusione del Copyright, si ha che le nazioni non riconoscono il Copyright di altri stati: nasce quindi l'esigenza di una legislazione uniforme. Nel 1883 si ha la Convenzione di Berna in cui le varie leggi vengono standardizzate. Inoltre viene resa automatica la tutela, senza nessuna registrazione. Il Copyright copriva solo gli oggetti tangibili e l'accordo includeva 165 paesi. Nel 1908 si deve affrontare White-Smith Music in una causa contro Apollo, primo caso legato a macchinari ``computazionali''. Si decide che la copia di musica su tubi per pianole non \`e una copia nel senso del Copyright Act. Si ha l'intervento del Congresso che afferma che i lavori derivati che possono essere percepiti solo attraverso l'uso i una macchina sono soggetti al copyright. Si affronta il problema con una legge ad hoc.

Nel 1950 con la nascita dei computer non si sente la necessit\`a del copyright. Il software \`e piuttosto pensato principalmente come complemento alla macchina: poche esigenze di una protezione, e manca una legge sul copyright software e i contratti che non si applicano a terzi\footnote{Data la mancanza di una legge sul copyright software se si volevano delle restrizioni nell'utilizzo del software da parte degli utenti si era soliti stipulare un contratto}. Inoltre il Copyright Act del 1909 imponeva la registrazione del materiale protetto ma non era chiaro cosa implicasse per il software.
C'\`e il desiderio delle case Software di proteggere il software sviluppato, e nel 1976 con il nuovo Copyright Act si ha che il software diviene proteggibile da Copyright. Una commissione viene incaricata di indagare sul grado di protezione dato al software. La commissione racomand\`o una definizione di software, e di permettere la copia o l'adattamento del software a patto che servisse p=a scopo di backup e siano necessari per l;utilizzo del software. Viene deciso che le copie esatte potevano essere vendute a posto che venissero trasferite completamente al nuovo proprietario. Le copie modificate potevano inoltre essere usate solo dal legittimo proprietario.

Nel 1980 il congresso approva i suggerimenti della commisione. Ma si ha che il termine ``legittimo possessore di una copia'' viene sostituito da ``proprietario di una copia''. Ci\`o porter\`a al caso giudiziario MAI vs Plack.

Nel 1990 viene limitato l'uso di piena vendita.

\subsubsection{Diritto in Italia}

Diritto esclusivo dell'autore su:
\begin{itemize}

\item Ridistribuzione
\item Modifica
\item Adattamnto
\item Traduzione

\end{itemize}

Tale diritto\footnote{In Italia il diritto si suddivide in due tipi: Morale e Economico. Solamente il diritto Economico pu\`o essere trasferito-rinunciabile} \`e:
\begin{itemize}

\item Rinunciabile
\item Trasferibile

\end{itemize}

Licenze software: ad esempio le licenze di windows. Le licenze GPL non vogliono fungere da contratto. La licenza GNU vuole differire dalle solite licenze software.

\section{Lezione 13-10-15}

\subsection{GNU}

Nota: argomenti trattati nel libro \textit{Heroes of the computer: Cap. 1,2 ed Epilogo}

Gli hack\footnote{In informatica: hack \`e un esercizio di codice che dimostra l'inventiva dell'autore, creato anche per il piacere dstesso dello scriverlo} del mit sono scherzi anonimi che dimostrano la creativit\`a della persona che li eseguono. Son presenti fino dagli anni 50.

\subsubsection{Gli albori}

Nel 1950-1960 la culla hacker nasce ai laboratori del MIT tra i ragazzini del Tech Model Railboard club. Il primo gruppo lo si ha dal sottogruppo Signal \& Power, che si occupava della gestione della circuiteria e segnali dei treni. Questa complessit\`a rasentava quella del software. Il Corso di intelligenza arificiale di McCarthy del 1959 facilit\`o la formazione di questo gruppo, permettendo di cominciare a programmare le prime macchine. I primi hack si ebbero sull'IBM 704, tramite la programmazione non interattiva\footnote{Queste macchine sono i primi esempi di macchine commerciali e si programmavano con le schede perforate e il loro accesso per l'utilizzo era molto ristretto.}. Con l'IBM 407\footnote{Macchinetta di servizio per la perforazione delle schede} e determinate modifiche era possibile usare il macchinario per poter programmare. La vera programmazione si vede per\`o con il TX0\footnote{La programmazione in assembler ammetteva 4 istruzioni}, dove la politica di utilizzo era un po' pi\`u ``rilassata''.
Caratteristiche principali della cultura hacker:
\begin{itemize}

\item Giocosit\`a
\item Condivisione degli sforzi di sviluppo (programmare sulle macchine di quel tempo richiedeva una notevole mole di lavoro)
\item Politica di apertura agli esterni (ovviamente se si era capaci). Peter Deutsch, ad esempio, si avvicin\`o alla comunit\`a del MIT molto presto.
\item Lavoro notturno\footnote{Permetteva un utilizzo maggiore delle macchine perch\`e la notte si aveva una minore affluenza di personale}

\end{itemize}

Nel 1961 arriva il PDP-1, primo computer concentrato sull'utente piuttosto che sull'ottimizzazione delle risorse, offrendo un approccio interattivo e inserendo nuovi tool per lo sviluppo. La progettazione \`e assegnata a Gurley (ex membro del MIT), che si basa sul TX-0 e TX-2. Ci\`o porta a un rafforzamento della comunit\`a hacker, grazie anche alla nascita di ARPANET.

\subsubsection{Etica hacker}

\begin{itemize}
  
\item L'accesso al computer, e a tutto ci\`o da cui si pu\`o imparare dovrebbe essere illimitato e totale. L'apprendimento avveniva non per basi teoriche, ma per basi pratiche (si voleva che il computer venisse ``aperto'' e studiato). Il ``Midnight Requisitioning Committee'' era un comitato non ufficiale che si occupava di requisire componentistica elettronica principalmente inutilizzata per poterla riusare.

\item L'informazione dovrebbe essere libera: senza l'informazione \`e impossibile capire e quindi migliorare i sistemi, con l'idea che l'informazione dovrebbe essere come il flusso di bit in un computer. Era inconcepibile quindi che un software fosse proprietario.

\item ``Non fidarti dell'autorit\`a e promuovi la decentralizzazione''. Secondo la comunit\`a hacker, l'autorit\`a porta con s\`e la burocrazia, che promuove regole arbitrarie che mirano solo alla propria perpetuazione.

\item Gli hacker dovrrebbero essere giudicati per il loro valore, non sulla base di fattori come razza, religione, sesso o posizione sociale. L'unica cosa che conta \`e quanto l'hacker pu\`o contribuire all'avanzamento dello stato dell'hacking.

\item \`E possibile creare arte e bellezza su un computer. Infatti i computer di quel tempo disponevano di pochissime risorse, e l'eleganza nella programmazione \`e vista come bellezza.

\end{itemize}
 
\subsubsection{Incompatible Timesharing System}

Con le macchine non-timesharing si avevano code lunghe per l'utilizzo del computer, causando problemi. Il progetto MAC puntava alla costruzione di una rete come quella elettrica destinata a distribuire potenza di calcolo.

Il timesharing era mal visto dagli hacker, che gli ricordava il Multics o CTSS (Compatible Time Sharing System), in quanto non si poteva avere controllo totale della macchina. Alcuni programmi che gli hacker sviluppavano avevano veramente bisogno di tutte le risorse del computer disponibili. Era quindi necessario aver bisogno di un accesso totale, e si arriv\`o a un compromesso: timesharing durante il giorno e single mode durante la notte, e per questo era necessario uno sviluppo di un SO in timesharing ispirato all'etica hacker: ITS\footnote{Incompatible Timesharing System}.

L'incompatible timesharing system aveva le seguenti funzionalit\`a:
\begin{itemize}

\item Utenti multipli ma anche programmi multipli per ogni utente
\item No password e assenza di sistema di permessi
\item File personali per ogni utente, ma disponibili a tutti
\item Strumenti collaborativi (es: possibilit\`a di switchare terminale e passare al terminale di un altro utente)
\item Fede negli utenti

\end{itemize}

L'ITS era quindi un sistema vivo che cresceva  con i suoi utilizzatori.

Con l'avvento del nuovo sistema, il PDP-10, si ha una crisi: viene infatti proposto di utilizzare TENEX, un nuovo OS al posto di ITS. Questo porta nel 1968 alle manifestazioni contro il laboratorio, con conseguenti atriti.

\subsubsection{Stallman - L'ultimo vero hacker}

Stallman ha le prime esperienze con i computer allo scientific center di New York. Stallman entra nel laboratorio attorno agli anni '70, quando il laboratorio si sta avvicinando alla fine. Non era un informatico, ma era interessato all'informatica in generale. Nel 1971 entra a Harward, ma si interessa di pi\`u al MIT, dove viene assunto da Russel Noftsker come programmatore di sistema. Stallman quindi si avvicina alla cultura hacker, e lavora insieme a Richard Greenblatt e Bill Gosper\footnote{Pi\`u incline nella risoluzione di problemi ambito ``lato matematico''}.

\paragraph*{Emacs} Dalla telescrivente si hanno dei passaggi verso i primi monitor con una riga. I primi programmi per gli editor sono Expensive Typewriter e TECO\footnote{Type Editor and Corrector}, che fungevano da insieme di strumenti a cui si lavorava applicando un insieme di comandi al testo. Questi tool erano comunque scomodi da usare, quindi Stallman decise di cercare un altro tipo di software altrove. Non trovandolo, fece un insieme di MACRO per TECO, dove era possibile:
\begin{itemize}
  
\item Editare il testo \textit{real time}
\item Permettere il \textit{random access editing}
\item Consentire l'aggiunta di ulteriori MACRO

\end{itemize}

La nascita di Emacs si ha dal caos dovuto alle troppe MACRO nate. Steele propone di generare un ordine nell'universo delle MACRO. Inizialmente impone una clausola che imponeva che ogni modifica a Emacs fosse inviata allo sviluppatore principale, in modo tale che, se fosse stata un'ottima idea, sarebbe stato possibile integrarla in Emacs e renderla disponibile a tutti. Questa clausola permett\`e alle persone di lavorare insieme agli altri in una community ma impose una limitazione alla libert\`a di sviluppo.

\subsection{La crisi del movimento Hacker del MIT}

\paragraph*{Le prime incursioni} Con l'avvento delle password, Stallman cerca di convincere le altre persone a limitarne l'utilizzo, e a permettere agli altri utenti di utilizzare i file di tutti. Ci\`o porta l'intervento del ministero della Difesa, che lo costringe all'uso di password. Ci\`o aliena la comunit\`a vicino a Stallman, anche a causa dell'introduzione dello \textit{sciopero del software}\footnote{Stallman rifuita di fornire allo staff del laboratorio le ultime versioni di Emacs fino a quando non avessero eliminato il sistema di sicurezza nel laboratorio}. La time bomb di Scribe fa prendere a Stallman la scelta di opporsi alle restrizioni sull'utilizzo del software.

\subsubsection{Cambiamento di mentalit\`a}

Nel 70-80 si ha una frammentazione della cultura hacker (nel 1976 si ha il copyright act), causata dal fatto che gli Hacker originari abbandonano il MIT per lavorare-aprire la propria azienda. Si ha un cambiamento dei visitatori al MIT, e con essi cominciano a essere presenti i primi programmi protetti da copyright nel laboratorio di intelligenza artificiale.

Questo a causa di un cambiamento degli equilibri di forze tra gli hacker e gli studenti-professori, che hanno un modo completamente diverso di vedere il software (ovvero come uno strumento).

\paragraph*{La lisp machine} La nascita della lisp\footnote{Linguaggio al tempo ad alto livello, molto potente ma macchinoso} machine causa la crisi finale. Dall'idea avuta da Greenblat, viene costruita una macchina concepita per funzionare in sintonia con Lisp, che ebbe successo. Con i copiosi fondi del progetto, al MIT vennero prodotte 32 macchine, che si voleva far comunicare in rete per favorirne la condivisione. Ci\`o porta all'idea di creare un'azienda ``hacker friendly'' per la produzione di lisp machine. Da ci\`o si ha lo scontro con Russel Noftsker, che propone di creare un'azienda ``per azioni'' e renderla prettamente commerciale. Dato lo scontro, viene raggiunto un accordo: a Greenblatt viene dato un anno di tempo per creare un'azienda per la vendita di lisp machines, che riesce a far partire entro l'anno (LMI). Nonostante ci\`o viene creata un'altra azienda, la Symbolics. Ci\`o caus\`o:
\begin{itemize}

\item Svuotamento del MIT
\item LMI e soprattutto Symbolics attingono pesantemente dal MIT
\item 1982: Symbolics rende le proprie modifiche al SO delle lisp machines proprietario, causando la vendetta di Stallman, che si mise a replicare tutte le funzionalit\`a e a donarle a LMI.
\item La comunit\`a hacker s'indebolisce, perch\`e minoritaria rispetto a studenti e professori, e diventa difficile sostenere dentro il laboratorio un sistema proprio interno.

\end{itemize}

\paragraph*{Cambiamento del PDP} Con il cambiamento del PDP tutto il software relativo diventa obsoleto. Ci\`o apre un fronte:
\begin{itemize}

\item Continuare a utilizzare ITS $\to$ diventato vecchio e poco sicuro
\item Usare Twenex, software proprietario derivato da Tenex (verr\`a scelto quest'ultimo)

\end{itemize}

\subsubsection{Nasce GNU}

GNU nasce grazie al caso della stampante Xerox, che veniva venduta con un software sotto la media e che tendeva a far inceppare spesso la stampante. Il software dato era proprietario, e i sorgenti sotto NDA\footnote{Atto di non divulgazione}. Stallman non poteva quindi migliorare il driver della stampante, e con ci\`o lasci\`o il MIT e and\`o a fondare il progetto \textit{GNU} nel 1983.

\paragraph*{L'appello} Nel 1983, su net.unix-wizard Stallman lancia dopo il giorno del ringraziamento un nuovo software Unix compatibile, che chiam\`o GNU\footnote{Gnu is Not Unix}. Unix venne scelto come base perch\`e:
\begin{itemize}

\item AT\&T, la proprietaria di Unix, non poteva venderlo causa della sua posizione dominante nella telefonia
\item Familiarit\`a con il codice sorgente
\item Portabilit\`a: Unix era stato sviluppato in un linguaggio creato appositamente per lui: il C.
\item Modularit\`a

\end{itemize}

Avendo bisogno di un compilatore, si mise a lavorare su Pastel, un compilatore libero. Purtroppo la struttura troppo pesante impediva a Pastel di lavorare su macchine poco potenti, e con ci\`o si mise alla stesura di GNU Emacs partendo dal codice di Gosling, reimplementando alcune funzionalit\`a sotto le minaccie legali di Unipress. Nel 1985 Stallman rilascia GNU Emacs, ponendo il problema di quale licenza usare per quel programma.

\paragraph*{GPL} Inizialmente, la licenza di GNU Emacs era ispirata dalle note di ``copyright'' sulle email. Inoltre, la diffusione del software in modo centralizzato, a differenza che con la commune. Per suggerimento di Gilmore si ha un cambio nome: si ha quindi la nascita della \textbf{GNU general public license}. GNU gpl v1 viene distribuita con il rilascio di gdb, con l'idea di unificare in un'unica licenza tutto il software GNU.

\paragraph*{L'incontro con BSD} Con la rottura del monopolio, AT\&T comincia a focalizzarsi sullo sviluppo di Unix a scopi commerciali. BSD era una distribuzione accademica derivata da Unix con vari contributi esterni ma richiedeva comunque il pagamento di una licenza AT\&T perch\`e non tutto il codice era di Berkeley. Nel 1984-1985 Stallman convince Bostic a sviluppare una distribuzione completamente libera.

Negli anni 80-90 Bruce Perens rilascia electric fence\footnote{Una libreria scritta in C} sotto GPL. Rich Morin fonda Prime Time Freeware, un'azienda che ricercava software open source nella rete e si occupava di rivendere i nastri.

La Cygnus era un'azienda che aveva cominciato a lavorare su gcc, facendone il porting al National Semiconductor's 32032. Data la modularit\`a di gcc, implement\`o il supporto a C++ e si occup\`o di portare il compilatore per sistemi embedded. Alla fine del 1990 aveva fatto $725.000$ \$ in supporto e contratti. L'azienda \`e stata ora venduta alla Red Hat.

\subsubsection{Espansione del progetto GNU}

Nel 1987 si ha la nascita di Libc:
\begin{itemize}

\item 1990 fork Linux
\item 1997 fork abbandonato

\end{itemize}

\paragraph*{GNU Hurd} Nel 1986 si ha il tentativo di basarsi su TRIX. Ma ci\`o causava il non funzionamento su macchine standard, e ci\`o portava a un numero troppo elevato di cambiamenti. Si prova quindi a basarsi sul codice BSD, ma si ha poca cooperazione da parte degli sviluppatori e si preferisce un approccio pi\`u ambizioso: basarsi su un microkernel usando \textit{Mach}. Nel 1990 iniziano i lavori sul kernel, ma si incontrano difficolt\`a di sviluppo, aggravata dalla poca attenzione dovuta dall'avvento di Linux. Attualmente Mach supporta i driver linux, supporta X, iceweasel e forse debian render\`a ufficiale una prossima release.

\subsection{BSD}

Testo di riferimento: \href{http://www.groklaw.net/staticpages/index.php?page=20051013231901859}{\textit{The Daemon, The GNU and the Penguin}}

\subsubsection{DARPA}

Nel 1957 si ha il lancio dello Sputnik1 da parte dell'URSS nello spazio. Gli americani, nel 1958 fondano l'ARPA (poi rinominata DARPA), \footnote{il cui scopo era lo sviluppo di nuove tecnologie per scopi militari} in cui veniva sviluppato MULTICS.

Nel 1963 nasce il progetto MAC\footnote{Multiple Access Computer, Machine Aided Cognitions}, sovvenzionato con due milioni di dollari dal DARPA. Gli obiettivi iniziali del progetto MAC erano quelli di rendere possibile l'affitto di potenza computazionale con la creazione di sistemi affidabili come quelli per la distribuzione di energia elettrica.

\paragraph*{Multics} Primo sistema operativo Hight Availability. \begin{itemize}

\item Sviluppato da MIT, General Eletric e Bell Labs
\item Struttura modulare: possibile aumentare le prestazioni del sistema semplicemente aggiungendo una ulteriore unit\`a (CPU, memoria, storage etc)
\item Riconfigurazione on-line
\item Linkaggio dinamico, filesystem gerarchico etc

\end{itemize}

Multics caus\`o un fallimento commerciale per via dell'estrema complessit\`a del sistema.

\paragraph*{Unix} Nel 1969 AT\&T si toglie dal progetto Multics. Dennis Richie e Ken Thompson desiderano continuare la ricerca sulle idee di Multics. Viene creata quindi una versione pi\`u leggera, compatibile con macchine pi\`u piccole, chiamata \textit{UNIX}. Nel 1972 Unix viene riscritto in C, per avere una maggiore portabilit\`a e maggiore facilit\`a di sviluppo. Nell'Ottobre del 1973 un articolo fa espandere la popolarit\`a di Unix.
Con il monopolio telefonico di AT\&T Unix viene distribuito liberamente, con la possibilit\`a di far liberamente delle modifiche (con codice sorgente). Nel 1977 John Lion pubblica il codice sorgente commentato di Unix, che causa un incremento dell'insegnamento di UNIX all'universit\`a, nonostante il tentativo di AT\&T di bloccare ci\`o. Conseguentemente, due anni dopo, AT\&T annuncia una restrizione sulla redistribuzione di Unix, che implicava sia una restrizione a livello commerciale sia una restrizione sulla possibilit\`a didattica nell'utilizzo di Unix.

Nel 1973 John Fabris dell'universit\`a di Berkeley assiste al talk su Unix al SOSP\footnote{Symposiuom on Operating System}, e decide di fare un dual boot in una sua macchina. Nel 1975 vengono acquistate altre due nuove macchine Unix. Berkeley ha bisogno di un certo supporto ai suoi sistemi, e nel 1975 Chuck Haley e Bill Joy arrivano a Berkeley, cominciando lo viluppo di un compilatore e editor per Pascal. Quello che diede una spinta decisiva a BSD fu che DARPA decide di muovrere la loro DARPANET su UNIX. Nel 1980 i fondi di DARPA vengono usati per il miglioramento di BSD UNIX, che rilascia 4BSD. In tutto questo, AT\&T annuncia di voler commercializzare Unix ed entra in conflitto con BSD. Nel 1983 Unix viene completamente commercializzato, con licenze sui sorgenti molto costose, in particolare per il TCP. In tutto ci\`o BSD cerca di liberare la dipendenza da AT\&T, che viene coronata con NET2. BSD386 (poi rinominato BSDI), distribuzione completa basata su NET2 e a stampo commerciale nasce in quell'anno, facendo scaturire una causa tra Unix System Laboratories (USL), BSDI e Berkeley, finita con la vittoria di BSDI.

\section{Lezione del 20-10-15}

USL perde la causa, ma si ha un rallentamento della diffusione di BSD e nel 1994 si ha la definitiva chiusura del processo tra Berkeley e Novell. Nello stesso anno viene rilasciato 4.4BSD-Lite, una versione di BSD completamente libera\footnote{Ci furono 2 release, una con funzionalit\`a in pi\`u in cui erano presenti sorgenti AT\&T in cui bisognava pagare licenze aggiuntive}.

\subsection{Linux}

Materiale di riferimento: \textit{The Deamon, the GNU and the Penguin: a history of Free and Open Source}; Peter Salus.

\subsubsection{Minix}

\paragraph*{Premesse} John Lions pubblica il codice sorgente commentato di UNIX. Nel 1978-1979 vengono bloccati i commentari di John Lions, e si ha un aumento dei costi delle licenze e restrizioni sull'insegnamento in classe. Questo causa l'interruzione dell'insegnamento di UNIX in molte universit\`a.

\paragraph*{Minix} Andrew Tanenbaum, docente di Computer Science, decide di scrivere un proprio sistema operativo sulle traccie di Unix: Minix. Minix \`e V7 compatibile, completo di compilatore ed editor. Era pensato per scopi didattici, era rilasciato sotto licenza permissiva e non libera.

\paragraph*{Nascita di Linux} Nel 1990 l'universit\`a di Helsinki installa un MicroVax con Ultrix. Linus studia il libro di Tenenbaum, ma non possiede un computer con Unix per mettere in pratica ciò che ha imparato. Segue il corso di "C e Unix". Nel 1991 Linus viene portato alla conferenza di Stallman e nello stesso anno compra un PC e comincia a scriverci un emulatore di terminale a partire da Minix. Sempre nello stesso anno Linus rilascia Linux 0.0.1\footnote{All'inizio il nome non si chiamava Linux, ma Freax e veniva distribuito sotto licenza proprietaria, in quando non permetteva l'utilizzo commerciale}. Nell'anno successivo, con Linux 0.12\footnote{Si ha la prima richiesta di una nuova feature da parte di un ragazzo che aveva a disposizione un computer con poca RAM. Cosí Torvalds implementa il paging.} Linus cambia la licenza adottando la licenza GPL, dando la possibilit\`a a tutti a contribuire, comprese aziende come SUSE. Si ha anche la prima distribuzione Linux: Linux Pro con Yggdrasil\footnote{Un database inizialmente solo per UNIX}. Si ha anche il diverbio tra Tanembaum e Linus. Nel 1994 si arriva alla versione di Linux 1.0.

Nel 1993 Mark Ewing fonda RedHat, una distribuzione che viene rilevata da Bob Young e diventa RedHat Software Inc. Nel 1995 nasce RedHat Linux, che diventa la pi\`u diffusa distribuzione Linux, ora a pagamento. Vengono effettuati anche i primi porting di Linux a DEC Alpha e SUN SPARC, mentre nel 1996 si ha il supporto per multiprocessore. Nel 2011 si arriva a Linux 3.0 e nel 2015 Linux 4.0.

\subsubsection{Debian}

Distro guidata dalla nuova generazione hacker di Linus. Linux come kernel stava prendendo suo vigore, creando una comunit\`a a sé stante allontanandosi dal software libero. Un motivo di ci\`o fu la causa che si ebbe nel 1988 tra Apple e Microsoft, che impegn\`o lo GNU nella LPF\footnote{League of programming freedom, orgranizzazione che si oppone ai brevetti software e ai copyright delle interfacce utenti}, impedendo che passasse il principio di copyright nella interfaccia grafica. Murdock allora per riavvicinare la comunit\`a Linux a GNU annuncia la sua intenzione di fare una distribuzione completamente libera. Avendo possibilit\`a di focalizzarsi sul progetto, nel 1996 viene rilasciata la versione 1.1, dopo la versione erronea 1.0 del 1995. Con Bruce Perens come nuovo DPL vengono scritte le Debian Free Software Guidelines (DFSG), dando un insieme di caratteristiche che le licenze dovevano avere per essere considerate conformi agli standard del parco software Debian. Su queste guide furono date le future definizioni di Open Source. Venne anche creato l'Open Hardware Certification Program, programma con il quale le aziende potevano certificarsi per essere compatibili con Linux. Il Software in the Public Interest\footnote{Organizzazione non a scopo di lucro di supporto alla creazione e diffusione di software libero.} era un ombrello legale tramite cui la Debian diventa un'entit\`a riconosciuta dai governi.

\paragraph*{Caratteristiche} Debian presenta il patto sociale, dove si ha un'attenzione maniacale alla qualit\`a, in cui si adotta solo software libero (secondo la DFSG) e dove gli sviluppi e decisioni sono presi in maniera comunitaria. Data l'impossibilit\`a di adottare solamente software libero, la Debian ha messo a disposizione (anche se non supporta ufficialmente) software non-free (ovvero costituito da software proprietario) e contrib (software libero che per funzionare ha bisogno di software proprietario). Diversi tipi di distribuzioni:
\begin{itemize}
  
\item Stable: versione rilasciata per il pubblico utilizzo.
\item Testing: entrano tutti i pacchetti che dopo 15 giorni di permanenza in unstable non hanno bug critici.
\item Unstable: versione per sviluppatori altamente instabile.

\end{itemize}

\subsection{Open Source}

Libro: \textit{Opensources: voices from the Open Source Revolution}

Nel 1997 Raymond scrive il libro \textit{The Cathedral and the Bazaar}, dove viene analizzato dal punto di vista architetturale Linux, che dall'esterno sembra destinato al collasso in quanto i collaboratori sono dei terzi che quindi hanno la possibilit\`a di sviluppare e dirottare il progetto (per questo viene associato al Bazaar). Analizzando tutto ci\`o si riesce a dettare delle linee guida sullo sviluppo di una comunit\`a open source:
\begin{itemize}

\item Ogni progetto deve partire da un ``prurito'' del programmatore
\item I buoni programmatori sanno cosa scrivere. I grandi programmatori sanno cosa riscrivere
\item Tratta i tuoi utenti come co-sviluppatori
\item Rilascia presto e spesso e ascolta i tuoi utenti
\item Dato un sufficiente numero di beta tester ogni problema verr\`a identificato e risolto
\item Riconoscere le buone idee degli utenti \`e importante come averne di proprie

\end{itemize}

Grazie a questi principi, nel 1998 Netscape (il primo browser grafico) annuncia la volont\`a di liberare il proprio codice sorgente, rendendo Raymond una celebrit\`a. Dopo il rilascio si cerca una strategia di lungo termine per mettere in vista l'Open Source, e si decide di lanciare una campagna di marketing, denominata ``Open Source'' sostituendo la nomenclatura ``Free Software''. Torvalds appoggi\`o l'iniziativa Open Source, con l'idea di usare DFSG come definizione di Open Source, e nel 1998 nasce la Open Source Initiative (1998) fondata da Raymond e Perens, causando il coinvolgimento di persone e di programmi. Strategie del movimento open Source:
\begin{itemize}

\item Approccio top-down
\item Puntare su Linux come dimostrazione $\to$ in quanto \`e molto software libero, e Linux aveva il fatto che poteva affascinare molte persone, dimostrando la forza del software Open Souce.
\item Puntare sulle Fortune 500 $\to$ i volontari tentano di puntare sulle piccole aziendine
\item Puntare sui media che influenzano le grosse aziende
\item Educare gli hacker sulle tattiche di promozione da seguire $\to$ troppa campagna portava ad avere effetti negativi
\item Usare un programma di certificazione $\to$ vengono rilasciate delle certificazioni per poter dichiarare se un software \`e open source o meno.

\end{itemize}

Il 7 marzo 1998 al Free Software Summit una ventina di leader del movimento appoggiano l'iniziativa, e si cerca di spingere altre aziende a seguire l'esempio di Netscape (Corel annuncia computer basati su Linux seguita da Oracle e Informix che annunciano il porting su linux).

I documenti Halloween sono dei documenti rilasciati erroneamente dalla Microsoft dove vengono sottolineati tutti i pericoli strategici che potevano venir dal movimento Open Source, e il fatto che colossi come Microsoft fossero preoccupati sottoline\`o l'importanza che questo movimento aveva preso.

\paragraph*{OSI} Il movimento Open Source prese una strategia diversa dal software libero, ovvero partendo da una serie di principi cercare una strategia per aumentare la visibilit\`a e raggiungere persone e aziende prima non possibile. La filosofia consisteva in:
\begin{itemize}

\item Licenze libere e permissive $\to$ per poter contribuire allo sviluppo di software in maniera comunitaria
\item Costruzione di una comunit\`a attorno al software $\to$ ci\`o permette alla comunit\`a di comprendere determinate scelte, facendo sentire la comunit\`a parte del processo decisionale
\item Trasparenza del processo di sviluppo $\to$ senza la possibilit\`a di modificarlo non \`e possibile interagire con il software e con la comunit\`a di utenti
\item Codice sorgente liberamente disponibile
\item Codice sorgente liberamente modificabile
\item Libera redistribuzione $\to$ insieme di linee guida che dovevano incanalare queste idee di base, costruite in modo di essere uno strumento di visione del software. Il movimento Open Source pubblica la sua definizione quasi come fosse un manifesto, spiegando il perch\`e di ogni punto. Imponendo la libera redistribuzione, si elimina la tentazione di rinunciare a importanti guadagni a lungo termine in cambio di un guadagno materiale a breve termine, ottenuto con il controllo delle vendite
\item Vincoli su altro software: la licenza non deve porre restrizioni su altro software distribuito insieme al software licenziato
\item Neutralit\`a rispetto alle tecnologie: la licenza non deve contenere clausole che dipendano o si basino su particolari tecnologie o tipi di interfacce

\end{itemize}

\paragraph*{Perl artistic license} La Perl artistic license sancisce che non \`e redistribuibile il software per scopi commerciali, mentre pu\`o essere incluso in una distribuzione pi\`u grande per essere venduto commercialmente. I software open source bloccano la redistribuzione di software commercialmente, ma permettono la redistribuzione dello stesso software ``wrapperizzato'' all'interno di qualcosa di pi\`u grande.


\input{res/fLessons/27-10-15}
\input{res/fLessons/03-11-15}
\section{Lezione del 10-11-15}

\subsubsection{Affero GPL v3}

Creata dalla Affero inc. per regolare i diritti sui servizi con codice coperto da GPL v3. La Affero GPL v3 \`e stato adottata dalla GNU.

\subsubsection{Mozilla Public License}

Originariamente nata dalla Mozilla per il rilascio del codice di Netscape, essa voleva rendere possibile l'aggiunta di software proprietario ``di contorno'' (come per esempio dei plugin), e quindi fu creata una licenza (inizialmente chiamata NPL) MPL che fu una via di mezzo tra la GPL iniziale e una BSD.

Con la Mozilla Public License \`e possibile avere una integrazione con le licenze proprietarie. Questo in quanto inizialmente voleva essere protetto il codice originario di Netscape. Con la Mozilla Public License viene consegnata una licenza brevettuale per i brevetti che sono necessari per l'utilizzo di quella versione del software.

\subsubsection{Perl Artistic License}

La Perl Artistic License non \`e pensata per diffondere, come nel caso della GPL, il software libero, ma \`e pensata per permettere all'autore del proprio software un controllo ``artistico'' sulla sua opera.

In questa licenza viene privilegiato il copyright holder. Il software in s\`e non necessita alcun pagamento, e il destinatario ha la possibilit\`a di distribuire il software agli stessi termini.

\paragraph*{Redistribuzione dei sorgenti}Il codice \`e redistribuibile mantenendo la nota di licenza originale, ma \`e possibile incorporare codice che sia o sotto dominio pubblico o sia scritto dallo stesso autore che detiene il copyright. Per la redistribuzione dei sorgenti modificabili \`e possibile rendere nel dominio pubblico le modifiche, o renderle disponibili gratuitamente. \`E possibile distribuire localmente il codice (per esempio per una propria azienda) con la licenza che si desidera. Se invece si vuole distribuire al pubblico, bisogna allora rinominare tutti i file modificati, in modo che sia chiaro che quella versione \`e stata modificata e non sia possibile confondersi con la versione originale.

\paragraph*{Redistribuzione dei binari}\`E possibile redistribuire direttamente i binari, a patto che siano disponibili anche i sorgenti originari con delle istruzioni su come prelevare la versione standard. Altres\`i \`e possibile distribuire insieme ai binari modificati anche i sorgenti modificati, oppure \`e necessario fare la rinominazione degli eseguibili e documentare le differenze.

\paragraph*{Uso commerciale}Non \`e possibile usare questa licenza per software con scopi commerciali, in quanto la vendita del software \`e vietata. \`E permesso ricevere un pagamento per il supporto e \`e possibile l'integrazione in superpacchetti commerciali. L'uso del software deve essere isolato dall'utente.

\subsection{Creative Commons}

Testi di riferimenti: \textit{Viral Spiral, How the Commoners Built a Digital Republic of Their Own - David Bollier} e \textit{Creative Commons: a user guide - Simone Aliprandi}

\subsubsection{GNU Free Documentation License}

Nata come una licenza sulla documentazione del software, in quanto problema molto sentito e non trattato dalla GPL (alcune tematiche riguardo alla trattazione dei documenti non vengono considerate).

Obiettivi della GNU Free Documentation License:
\begin{itemize}

\item Libert\`a di modifica
\item Tutela dei diritti morali dell'autore $\to$ per gli autori \`e possibile che ci siano parti importanti da non modificare, quindi \`e stata introdotta la possibilit\`a di inserire alcune sezioni non varianti, che possono essere tutte quelle sezioni che non sono l'argomento principale dell'opera.
\item Problema gestione delle copie non trasparenti $\to$ viene regolato il modo in cui viene gestita la redistribuzione per grandi quantit\`a 
\item Copia in grande quantit\`a e non

\end{itemize} 

La licenza suddivide il documento in varie parti:
\begin{itemize}
  
\item Documento
\item Titolo
\item Sezioni secondarie e invarianti
\item Testi di copertina
\item Storia del documento $\to$ changelog all'interno dell'opera
\item Licenza
  
\end{itemize}

La GNU FDL impone delle restrizioni sulle redistribuzioni senza modifica:
\begin{itemize}

\item Mantenimento della licenza
\item No misure tecnologiche di restrizione
\item Esibizioni pubbliche permesse
\item Registribuzioni voluminose:
  \begin{itemize}

  \item Obbligo di identificarsi come editore
  \item Obbligo di mantenere titolo e testi di copertina
  \item Obbligo di distribuire il sorgente

  \end{itemize}

\end{itemize}

Se invece vengono applicate delle modifiche \`e necessario:
\begin{itemize}

\item Modifica del titolo
\item Indicazione degli autori delle modifiche e del documento
\item Rimozione dei ``Riconoscimenti''
\item Preservazione della sezione delle dediche
\item Aggiornamento della sezione cronologia
\item Preservazione degli invarianti\footnote{Anche le traduzioni vengono considerate delle modifiche}
\item Preservazione della versione trasparente
\item Preservazione e/o aggiunta dei testi di copertina

\end{itemize}

\`E presente una clausola speciale, creata per Wikipedia, che sotto specifiche condizioni permettevano di convertire la licenza sotto CC-BY-SA (la licenza Creative Commons pi\`u vicina alla FDL). Questa clausola, temporanea, \`e ora scaduta.

Nei documenti ha senso trattare i vari tipi di unioni (a differenza della GPL), ed \`e possibile mantenere solo una licenza nel documento.
Nella traduzione dei documenti \`e possibile tradurre anche la licenza, ma \`e necessario mantenerne una versione in inglese.

La GNU FDL \`e un ``ponte'' tra le licenze software e le licenze sui documenti.

\subsubsection{Transizione alle Creative Commons}

Creata da un insieme di persone e dopo molti anni.

\paragraph*{Background}Negli anni '60 si ha un aumento pervasivo del copyright, fino al culmine degli anni '80 quando il giornale Nation prese un estratto dal libro del presidente Ford, causando una causa legale vinta da Ford. Negli anni successivi ci furono i cambiamenti dei mezzi d'informazione, coinvolgendo nella copia anche le persone, causando una maggiore sensibilizzazione del copyright al grande pubblico.

\paragraph*{Inizio della crisi}Nel 1990 si ha la nscita di EFF, associazione molto attiva che si contrapponeva alle stringenti regole del copyright dettate dalle lobby. Si ha anche la nascita della National Information Infrastructure creata da parte di Clinton che cercava di portare ordine su Internet, in cui vennero proposte tutte una serie di modifiche restrittive nei confronti del pubblico. Questo caus\`o una forte reazione, con la pubblicazione di un articolo nel 1994 dal The Economy of Ideas di Barlow, accumunando molte persone. Tutto ci\`o sfoci\`o nel 1995 nella creazione della Digital Future Coalition, che si contrapponeva al DMCA.
Nel 1996 si ha la Dichiarazione d'indipendenza del cyberspazio, in cui viene dichiarata l'internazionalit\`a dei contenuti del web e dove viene affermato che gli stati non dovrebbero averne influenza.
Nel 1999 si ha la conferenza ``Private censorship/perfect choice''.
Alla fine degli anni novanta si valorizza il dominio pubblico da parte di queste comunit\`a createsi, che considerano i lavori di dominio pubblico una base per la nascita di futuri progetti.

\section{Lezione del 24-11-15}

Nel 1998 viene approvato il Sonny Bono Copyright Extension Act\footnote{Detto anche Mickey Mouse Protection Act}, che come sostenitore aveva la Disney, dove l'estensione del copyright veniva estesa di ulteriori 20 anni. Questa estensione colp\`i gli interessi di tantissime opere, come il caso di Eldricht Press di Eric Eldred che ospitava un sito web di libri con il copyiright scaduto.

%\subsection{Lessing}

\paragraph*{Lessig}Politico in Erba derivante da una famiglia repubblicana, dopo il college a Cambridge divent\`o liberale e si interess\`o di Giurisprudenza, incominciandosi a interessare dell'effetto Eldred, dove una stessa legge era stata interpretata in maniera diversa in base a pressioni sociali. Si interess\`o dell'impatto dell'architettura di internet sulla legge, dove siti grandi visitati da molti utenti finivano per ottenere un controllo sugli utenti.
Negli ultimi anni si assiste a una fusione tra il mondo reale e il mondo cybernetico.
Con il patto Lessig-Eldred si ha la denuncia dell'incostituzionalit\`a del SBCEA. Si ebbe l'apertura del sito \textit{openlaw}, dove persone avevano la possibilit\`a di confrontarsi e trasmettere dei valori sul modo di vedere su ci\`o che doveva essere il modo di vedere la diffusione dell'informazione. La causa fu persa 7-2 alla corte suprema nel 2003.

\subsection{Creative Commons}

Creata dall'idea di Eldred: creare una copyright conservancy. Dopo la fine della causa con Microsoft Eldred comincia a farlo fin dall'inizio dei principi, in quanto bisogna vedere l'apertura del copyright per tutte le opere, in quanto le case produttrici dopo pochi anni smettevano di vendere le opere in quanto non c'era pi\`u un vero guadagno. Era inoltre necessario abbandonare i principi sul fair use\footnote{Utilizzo dell'opere senza vincoli. Sono presenti dei paletti (\`e possibile per una piccola citazione che non sia contro l'autore stesso etc..).}. Era necessario creare un movimento tecnico-legale, senza esserne a conoscenza si veniva a creare l'inizio della \textit{Creative Commons}. Gi\`a nel 1999 era nato il \textit{Copyright Commons}, una newsletter mensile sul caso Eldred vs Reno. Da qui si ha l'idea per la creazione della compagna Counter Copyright (CC), per dare la possibilit\`a di creare opere e distribuirle anche sotto Public Domain, anche se non era possibile in quanto ogni opera \`e sempre sotto un dominio.

Nel 1999 si ha anche la nascita di Napster, un sito in cui scaricare musica illegale che fece moltissimo successo. Lessig capisce che c'\`e moltissima gente che ha bisogno di accedere ai contenuti liberamente. Da qui nasce nel 2000 dall'idea di Habelson: una fondazione che accetti donazioni di opere, anche se scartata in quanto troppa problematica. Da ci\`o si ha nel 2001 la fondazione della Creative Commons\footnote{Il nome deriva dalla tragedia di Garret Hardin ``tragedia dei commons''.}, dove era possibile adottare una licenza comune per tutte le opere che volevano essere diffuse liberamente. Gli obiettivi del progetto sono:
\begin{itemize}

\item Promuovere la diffusione di un modello ``some rights reserved''
\item Tutelare il marchio ``Creative Commons''
\item Creazione di  appropriati strumenti legali e tecnologici

\end{itemize}
Le creative commons si pongono a met\`a tra il pubblico dominio e tra il copyright assoluto.

Creative Commons non \`e un ente pubblico, n\`e un organismo per la raccolta dei diritti d'autore e non offre servizi di consulenza legale.

\paragraph*{Licenze CC}Le licenze CC sono pensate per opere creative non software. Sono disponibili per 52 sistemi giuridici, e hanno come caratteristiche comuni sulla preservazione delle note di licenza, sul permesso di copia e distribuzione dell'opera e sull'esibizione dell'opera e l'uso di restrizioni tecnologiche vietato.
\`E possibile inserire un digital code che segna la licenza che permette di facilitare la ricerca nei motori come Google, rendendo la loro visibili\`a pi\`u rapida. Ogni licenza \`e identificata dal suo commons deed:
\begin{itemize}
  
\item Attribuzione
\item No opere derivate
\item Non commerciale
\item Condividi allo stesso modo

\end{itemize}

Tutte le combinazioni formano le licenze Creative Commons. \`E da notare che ``Attribuzione'' \`e presente in tutte le licenze e che non esiste ``No-Derivate'' e ``Share-alike'' nella stessa licenza in quanto non avrebbe molto senso.
Le licenze possibili sono:
\begin{itemize}

\item Attribution
\item Attribution-SA $\to$ per certi versi la pi\`u simile alla GNU GPL.
\item Attribution-ND $\to$ non \`e permesso creare opere derivate, ma \`e sempre possibile la distribuzione (sempre sotto la stessa licenza)
\item Attribution-NC $\to$ non sar\`a mai possibile la ridistribuzione a scopo commerciale. \`E anche possibile cambiare licenza basta che rimanga la clausola NC.
\item Attribution-NC-SA $\to$ \`e necessario riconoscere l'autore, non \`e possibile utilizzarla per scopi commerciali e \`e necessario condividerla allo stesso modo
\item Attribution-NC-ND $\to$ non \`e possibile usarla per scopi commerciali senza alcuna modifica. Questo si usa per opere che vogliono avere un'ampia diffusione e veniva utilizzato recentemente per i temi Wordpress.

\end{itemize}

\subsubsection{Obblighi delle licenze CC}

Una raccolta di opere \`e da considerarsi come un lavoro collettivo. La distribuzione della licenza \`e possibile attraverso un URL. Per eseguire una citazione su un'opera protetta da CC \`e necessario specificare:
\begin{itemize}

\item Autore originario
\item Titolo dell'opera originaria
\item Se possibile URL associato all'opera
\item Uso dell'opera
\item Rimozione citazione su richiesta del licenziate\footnote{\`E necessario specificare: il licenziante \`e colui che licenzia l'opera, l'autore \`e colui che la scrive. }

\end{itemize}

Con l'attribution SA si aveva la garanzia del proprietario dei diritti di avere i diritti necessari e che l'opera rispetta i diritti d'autore, di marchio di fabbrica e altro. Questa clausola \`e stata rimossa dalla licenza 2.0 in poi.

\subsubsection{Strumenti riconoscimento licenze CC}
\begin{itemize}

\item CC plus: protocollo per esprimere licenze alternative, viene usato anche dai motori di ricerca per trovare documenti con specifiche licenze.
\item CC0: alternativa al public domain, che permette di scegliere in modo pi\`u vicino possibile in base alla giurisdizione del paese di rilasciare la vostra opera.
\item Founders' copyright: possibilit\`a di mantenere il copyright per 14 anni rinnovabili una volta. Questa funzionalit\`a \`e stata rimossa.

\end{itemize}

\subsection{RDF e RDFa}

Materiale di riferimento: \textit{Pratical RDF di Shelley Powers, RDF primer, RDFA primer, RDFA syntax al www.w3.org}

\paragraph*{Definizioni}RDF \`e un linguaggio per rappresentare informazioni semantiche sul web.
RDFa \`e un modo per esprimere le informazioni semantiche in RDF all'interno di una pagina web. \`E un'estensione xhtml.
Ccrel: vocabolario per RDFa per esprimere asserzioni sulla licenza di contenuti web.

\subsubsection{RDF}
L'obiettivo dell'RDF \`e di costruire un linguaggio per esprimere contenuti semantici in maniera semplice, estendibile attraverso moduli e interpretabile da un calcolatore.
La soluzione adottata \`e attraverso grafi: questo perch\`e pi\`u facile da elaborare da parte di un calcolatore, rimanendo comunque estendibi. Un grafo \`e formato da degli statement, che sono creati da soggetto, predicato, oggetto. \`E importante disambiguare in maniera semplice i vari significati delle parole, e questo viene attraverso degli URI o un nodo anonimo o letterale. La visualizzazione di queste opzioni avviene rispettivamente con un ovale contentente l'URI, cerchi vuoti, rettangoli contententi la rispettiva stringa.

\paragraph*{Struttura dell'URI}Le URI non rappresentano sempre dei siti web, e in genere sono composti da:
\begin{itemize}

\item Schema (http, ftp, uuid)
\item Authority (opzionale)
\item Cammino assoluto o relativo
\item Query e identificatore di frammento

\end{itemize}

L'URI esprime solo nomi e non localizzazioni.

\paragraph*{Sintassi delle URI}La sintassi generica di un URI \`e la seguente:
\begin{verbatim}

schema:localpart [?query][#frammento]

\end{verbatim}

Dove lo schema pu\`o essere http, ftp, tel...
Per indentificare l'authority dev'essere presente la doppia barra ``//''.
Ad esempio:
\begin{verbatim}

http://www.debian.org/index.html

\end{verbatim}

Se si scrive http:/index.html allora si indica la pagina index dentro l'authority www.debian.org.
Per esempio il numero di telefono non hanno un authority, come anche per la voce mailto.

\paragraph*{Utilit\`a degli URI}Gli URI permettono di creare degli identificativi unici all'interno del web, creando uno schema forse difficile da leggere ma preciso.

\paragraph*{XML Qualified Name}Un qualified name \`e identificato da un URI. \`E da notare che gli escape forniscono qualified name diversi. I Qname in RDF dipendono dalla codifica adottata. Un modo abbastanza comune \`e specificare RDF con xmlns.
Esistono alcuni namespace standard:
\begin{itemize}

\item rdf
\item rdfs
\item dc
\item ex
\item foaf

\end{itemize}

Visitando prefix.cc \`e possibile accedere alle risorse che puntano alle varie risorse (rdf).

\paragraph*{Nodi anonimi}
Con i nodi anonimi \`e possibile creare nodi senza un URI associato, ed \`e inoltre possibile inserire letterali. I letterali di un grafo RDF possono avere un \textbf{tipo associato}, esso non \`e un tipo obbligatorio. I tipi in RDF sono definiti in XML Schema, ma non tutti i tipi di XML Schema sono associabili a un letterale, ma solo a un tipo primitivo. \newline

L'insieme di RDF/XML risulta essere molto verboso e scarsamente collegato al web.

\paragraph*{Notazione N-Triples}

Ogni riga pu\`o essere un commento (che inizia con ``\#'') o una tripla del tipo: subject predicate object.
I nodi anonimi si indicano con '\_:name'.
Questo approccio presenta gli stessi problemi di replicazione di RDF/XML, ma presenta una sintassi molto semplice.

\section{Lezione del 01-12-15}
\graphicspath{ {res/data/01-12-15/} }

\paragraph*{Collezioni RDF}

I container RDF non sono ``bloccabili''. Quindi, se necessario c'\`e il bisogno dell'uso delle collezioni. Le collezioni vengono viste come liste singolarmente linkate, e si accede tramite le seguenti voci: \textit{rdf:first, rdf:rest, rdf:nil}

\paragraph*{Reification}

I reification permettono di esprimere predicati che parlano ad altri predicati. Per esprimere ci\`o esistono dei predicati standard da applicare ai nodi anonimi:
\begin{itemize}

\item rdf:Statement
\item rdf:subject
\item rdf:predicate
\item rdf:object

\end{itemize}

\subsubsection{RDFa}

Il linguaggio RDFa \`e un linguaggio di annotazione rdf per XHTML, compatibile con i namespace.

\paragraph*{Flusso RDFa}Inizialmente il soggetto corrente \`e il documento corrente, avendo esso stesso un url. Esegue una scansione depth-first (lavorando in maniera ricorsiva), cercando un tag che specifica il predicato. Se \`e presente un oggetto e un soggetto, ottiene una tripla completa, aggiungendolo al grafo. Il soggetto iniziale \`e sempre il tag ``base''. \newline

RDFa supporta una \href{http.//www.w3.org/2011/rdfa-context/rdfa-1}{lista di prefissi iniziali predefiniti}.

\newpage

\paragraph*{Property}Il tag property setta l'oggetto di una propriet\`a. Ad esempio il codice:
\begin{verbatim}

<p>Il contenuto di questo sito e' licenziato sotto
<a property="cc:license" 
href="http://creativecommons.org/licenses/by/3.0/">CC-BY</a>
</p>
\end{verbatim}

Viene schematizzata nella seguente maniera:

\begin{figure}[h]
  \centering
  \includegraphics[scale=0.6]{01-12-15-01}
\end{figure}

\paragraph*{Content}Specifica il contenuto da usarsi a fini RDFa quando il contenuto del tag corrente \textbf{non \`e desiderabile} o \textbf{non \` disponibile}. Ci\`o avviene quando si vuole sostituire il contenuto non appropriato con un alternativo per qualche ragione (es. data americana con data europea).

\paragraph*{Vocab}Setta un vocabolare di default per tutti i discendenti. Pu\`o essere mescolato con full uris, e pu\`o essere settato su diversi livelli dell'albero xhtml. Utile soprattutto quando si hanno molte propriet\`a dello stesso vocabolario.

\paragraph*{Prefix}Permette di utilizzare velocemente pi\`u di un prefisso. Questo sistema permette di essere mescolato con vocab.
Ad esempio:
\begin{verbatim}
<body prefix="p1: http://www.p1.org/ p2: http://www.p2.org">
<span property="p1:property">properieta' 1</span>
[...]
</body>
\end{verbatim}

\paragraph*{About}L'attributo about permette di settare il soggetto, che diventer\`a il soggetto corrente all'interno del tag
Es:
\begin{verbatim}
<div about="/barbecue">
[...]
</div>
\end{verbatim}

Mescolando about e property si ha il setting contemporaneo di soggetto e predicato. \`E importante notare che l'ordine con cui gli attributi vengono inseriti \`e importante. Ad esempio:
\begin{verbatim}
<p>frammento di codice rilasciato sotto
<span about="#code" property="license">
cc-by
</span>
</p>
\end{verbatim}

L'about permette non solo di settare il soggetto, ma di settare il soggetto in modo anonimo. Ad esempio:
\begin{verbatim}

<link about="[_:n1]" rel="foaf:mbox" href="mailto:john@ex.org" />
\end{verbatim}

\paragraph*{Typeof}
Typeof pu\`o essere usato in due situazioni:
\begin{itemize}
  
\item Settare un tipo per un soggetto $\to$ funziona solamente inserendo anche l'attributo about
\item Creare un nodo anonimo usato come base per i predicati $\to$ esempio:
\begin{verbatim}
  <div typeof="foaf:Person">
    <p propery="foaf:name">Alice</p>
    <p>
      Email
      <a rel="foaf:mbox" href="mailto:alice@example.com">
        alice@example.com</a>
    </p>
    <p>Phone: <a rel="foaf:phone"
      href="tel:0444123456">0444123456</a></p>
  </div>

\end{verbatim}

\end{itemize}

\paragraph*{Rel}L'attributo rel collega l'oggetto all'attributo resource, href o src o ai nodi contenuti. Rel e Property presentano delle differenze tra loro:
\begin{itemize}

\item Il Rel non si collega al contenuto interno del tag e nemmeno al valore di content
\item Rel supporta il \textit{chaining}
  
\end{itemize}

Il chaining \`e uno strumento molto potente, che pu\`o concatenare oggetti esterni e soggetti interni tra loro. Ad esempio:
\begin{verbatim}
<a about="http://www.debian.org" rel="dc:creator"
href="http://www.ian.org">
<span property="foaf:name">Ian Murdock</span>
</a>
\end{verbatim}

Come si pu\`o vedere, lo span di foaf:name \`e si lega a www.ian.org e a Ian Murdock: ovvero l'oggetto della proposizione esterna diventa soggetto di quella interna.

\`E anche possibile che il soggetto della proposizione interna diventi oggetto di quella esterna. Ad esempio:
\begin{verbatim}
<div about="#me" rel="foaf:knows">
  <div about="http://www.w3.org/People/Ivan/#me"
   property="foaf:name" content="Ivan Herman" />
</div>
\end{verbatim}

\begin{figure}[h]
  \centering
  \includegraphics[scale=0.5]{01-12-15-02}
  \caption{Schema dell'esempio sopra riportato}
\end{figure}

\`E anche possibile il \textit{chaining con intermediario anonimo}. Se non viene specificato n\'e un oggetto per la proposizione esterna n\'e un soggetto per quella interna, il collegamento \`e fornito da un nodo anonimo creato automaticamente.

\paragraph*{Datatype}Datatype pu\`e venire usato per esprimere il tipo di un letterale.
Per esempio \`e possibile dichiarare l'et\`a di un sito.

\subsubsection{RDF Schema}
Descrive il vocabolario usabili in RDF. Si dividono in vari elementi.

\paragraph*{Classi}Collezione di elementi, dette istanza collegate agli elementi attraverso rdf:type. \`E possibile creare gerarchie gerarchiche. Si ddice che una classe A sia sottoclasse di B se ogni istanza di A \`e anche istanza di B.

\paragraph*{Propriet\`a}Connessioni tra i vari elementi. \`E possibile definire delle \textbf{sottoproriet\`a}, in modo che se \`e presente una propriet\`a allora vale anche un'altra. Inoltre \`e possibile definire \textbf{range} e \textbf{domain}.

\section{Lezione del 15-12-15}
\graphicspath{ {res/data/15-12-15/} }


\paragraph*{Dublin Core}Questo viene importato con il prefisso ``dc'', e presenta le principali voci:
\begin{itemize}
\item[Contenuto]
  \begin{itemize}
    
  \item Title: Titolo dell'opera
  \item Subject: Topic dell'opera
  \item Description: Abstract
  \item Source: Opera originaria
    
  \end{itemize}

\item[Propriet\`a Intellettuale]
  \begin{itemize}

  \item Creator: autore
  \item Contributor: ulteriore contributore
  \item Publisher: per chi \`e responsabile per la pubblicazione della risorsa $\to$ per una collana pu\`o essere il rispettivo pubblicatore
    \item Rights: un link alla licenza
    
  \end{itemize}
  
\end{itemize}

I predicati double core possono essere usati per specificare la licenza, ma pu\`o essere possibile usare un altra vocabolario pi\`u specifico.

\paragraph*{CCrel} \`E un vocabolario creato per parlare delle licenze creato dai membri della creative commons. Permette di specificare le propriet\`a del lavoro e della licenza.
Voci:
\begin{itemize}

\item cc:attributionName
\item cc:attributionUrl $\to$ permette di citare direttamente la licenza di tutto il documento, semplificando il suo ritrovamento
\item dc:source
\item cc:morePermission $\to$ permette dare un link per permettere di acquistare una licenza pi\`u permissiva
  
\end{itemize}

Le altre voci vengono riprese da altri vocabolari, come ad esempio Double Core o xhtml.

\textbf{Dc:type} permette di specificare il tipo di oggetto. Alcuni possibili valori possono essere:
\begin{itemize}

\item dcmitype:Text
\item dcmitype:Sound
\item dcmitype:StillImage
\item dcmitype:MovingImage
  
\end{itemize}

Un esempio \`e lo schema creato dal ``choose license'' di Creative Commons

\begin{figure}[h]
  \centering
  \includegraphics[scale=0.4]{15-12-15-02}
\end{figure}

\subsection{Subversion}

Altre risorse esterne: \textit{Version Control with Subversion} - Rilasciato sotto licenza CC all'indirizzo \url{http://svnbook.red-bean.com}

Subversion presenta un approccio completamente diverso rispetto ad altri sistemi di versionamento come Git o Mercurial: \`e presente un server centrale, e i client possono accedere, leggere o pushare delle modifiche. Ogni modifica del repository viene creata una nuova revisione. Subversion permette di avere non solamente uno snapshot del sistema in quel momento ma anche una shadow copy del lavoro: ovvero \`e possibile avere una visione ``ibrida'' tra due revisioni. \`E possibile avere pi\`u rami, ed \`e possibile eseguire il merge dei rami.
Subversion non si occupa di eseguire alcun tipo di analisi semantica, ma lavora dal punto di vista puramente testuale.
Termini importanti:
\begin{itemize}

\item Checkout e working copy
\item Export $\to$ \`e possibile eseguire una copia del file senza i file di subversion
\item Commit $\to$ la modifica viene inviata sul server e si ottiene una nuova revisione
\item Update $\to$ vengono presi gli ultimi aggiornamenti dal repository
\item Revisioni $\to$ memorizza l'ultima modifica del file nel repository, e permette di mantenere le versioni. Funzionamento delle revisioni:
  \begin{itemize}
    
  \item Per repository
  \item Collegate al singolo file
  \item Tag e branches $\to$ i tag permettono di dare un nome alla ultima revisione. I rami (branch) sono delle revisioni multiple che hanno una vita propria, e possono essere di sviluppo, di prova o anche stabili. Nei rami \`e possibile continuare a lavorare, mentre nei tag no.
    
  \end{itemize}

\end{itemize}

In Subversion nulla vieta di avere pi\`u progetti in un unico repository. All'interno di ogni cartella si \`e liberi avere la propria gerarchia di cartelle. \`E impotante notare che un repository non contiene una copia diretta dei progetti ospitati. Il repository \`e possibile accederci via molteplici interfacce: http, https, svn, svn+ssh, file.

\subsubsection{Problema dell'Locking}Il problema del Locking si verifica quando due utenti eseguono le modifiche al server contemporaneamente. Si vuole evitare che il lavoro di un utente non venga sovrascritto dal lavoro di un altro. Esistono due approcci principali:
\begin{itemize}

\item Alla apertura di un file il repository blocca l'accesso a quel file, impedendo gli altri utenti di editare quello specifico file. Questo sistema porta ad una serie di problemi: il lock pu\`o essere lungo in modo arbitrario, viene creata una serializzazione non necessaria e si ha una falsa sicurezza.
\item Locking ottimistico: permette a pi\`u utenti di effettuare le modifiche allo stesso file contemporaneamente. Quando accadono i conflitti devono essere risolti dai programmatori, e poi le modifiche devono essere caricate sul server, mantenendo una versione consistente (merge manuale).
  
\end{itemize}

\subsubsection{Revisioni e file}
Subersion memorizza per ogni file: la revisioni del repository su cui \`e basato, un timestamp di quando \`e stato aggiornato da repository l'ultima volta, e i file originali prelevati dal repository.

Durante il lavoro un file pu\`o essere in uno stato di:
\begin{itemize}

\item Inalterato localmente e aggiornato
\item Alterato localmente e aggiornato
\item Inalterato localmente e non aggiornato
\item Alterato localmente e non aggiornato $\to$ si ottiene l'erroe ``out of date'' ed \`e necessario eseguire un merge manuale
  
\end{itemize}

\paragraph*{Esempio pratico}Per creare un repository:
\begin{verbatim}
svnadmin create repo #repo e' il nome del repository
\end{verbatim}
Dopo la creazione di un repo, per inserire i dati basta semplicemente creare i file all'esterno della cartella di svn. Per inserirli nel repository \`e necessario creare una certa struttura standard. Per importari i dati si usa la keyword \underline{import}, in cui bisogna specificare la cartella e il repository in cui bisogna aggiungere questi file. \`E necessario specificare lo schema (protocollo), l'indirizzo e il path.
\begin{verbatim}
svn import proj/ file:///home/cesco/tmp/repo
\end{verbatim}
A questo punto si otterr\`a un errore, in quanto \`e necessario specificare un editor per committare i messaggi. Con l'opzione ``-m'' \`e possibile inserire direttamente un messaggio
\begin{verbatim}
svn import proj/ file:///home/cesco/tmp/repo -m "Import Iniziale"
\end{verbatim}
Per visualizzare i file all'interno del repo \`e possibile dare:
\begin{verbatim}
svn ls file:///home/cesco/tmp/repo
\end{verbatim}
\`E importante notare che non comparr\`a la cartella, in quanto vengono aggiunti solamente i file all'interno della cartella ``proj'' specificata prima.

Per importare un repository si esegue un \textit{checkout}, ovvero:
\begin{verbatim}
svn co file:///home/cesco/tmp/repo proj #proj e' la cartella dove verranno inseriti i file
\end{verbatim}
A questo punto viene prelevato dal repository i file e vengono inseriti in una cartella con lo stesso nome.

Per controllare eventuali file modificati \`e necessario scrivere:
\begin{verbatim}
svn status #oppure svn st
\end{verbatim}

\subsubsection{Strutture e configurazioni}

\`E presente un file di configurazione (svnserve.conf) che permette di gestire le varie configurazioni, come per le password (passwd) e il login (authz). db \`e la cartella del repository, e hooks \`e la cartella dove possono essere inseriti degli script agganciabili agli eventi. \`E presente un README e un file format.



\end{document}
